%%%%%%%%%%%%%%%%%%%%%%%%%%%%%%%%%%%%%%%%%%%%%%%%%%%%%%%%%%%%%%%%%%%%%%%%%%%%%%%%%%%%%%%% To define Code Syntax Scala
%%%%%%%%%%%%%%%%%%%%%%%%%%%%%%%%%%%%%%%%%%%%%%%%%%%%%%%%%%%%%%%%%%%%%%%%%%%%%%%%%%%%%%%
% Scala
%%%%%%%%%%%%%%%%%%%%%%%%%%%%%%%%%%%%%%%%%%%%%%%%%%%%%%%%%%%%%%%%%%%%%%%%%%%%%%%%%%%%%%%
\lstdefinestyle{myScalastyle}{
	frame=tb,
	language=scala,
	aboveskip=3mm,
	belowskip=3mm,
	showstringspaces=false,
	columns=flexible,
	numbers=left,                   % where to put the line-numbers
	numberstyle=\tiny\color{gray},  % the style that is used for the line-numbers
	stepnumber=1,                   % the step between two line-numbers. If it's 1, each line will be numbered
	numbersep=5pt,                  % how far the line-numbers are from the code
	backgroundcolor=\color{white},  % choose the background color. You must add 
	keywordstyle=\color{blue},
	commentstyle=\color{mygreen},
	%  stringstyle=\color{mauve},
	stringstyle=\color{myorange},
	frame=single,
	breaklines=true,
	breakatwhitespace=true,
	breakindent=20pt,
	tabsize=3,
	frameround=tttt,
	escapeinside={\%*}{*)},        % to add a comment within your code
	emph={count,take,textFile,filter,first,collect,mkString}, % Scala functions
	emphstyle={\color{mauve}},
	morekeywords ={val,sc},        % to add more keywords to the set  
}
\renewcommand{\lstlistingname}{\textbf{\underline{Scala Code}}}
\newcommand{\myhy}[2]{\hyperref[#1]{\color{green}\setulcolor{red}\ul{#2}}}
