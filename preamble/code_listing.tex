%%%%%%%%%%%%%%%%%%%%%%%%%%%%%%%%%%%%%%%%%%%%%%%%%%%%%%%%%%%%%%%%%%%%%%%%%%%%%%%%%%%%%%%% 
%To define Code Syntax Scala
%%%%%%%%%%%%%%%%%%%%%%%%%%%%%%%%%%%%%%%%%%%%%%%%%%%%%%%%%%%%%%%%%%%%%%%%%%%%%%%%%%%%%%%% 
% Scala
%%%%%%%%%%%%%%%%%%%%%%%%%%%%%%%%%%%%%%%%%%%%%%%%%%%%%%%%%%%%%%%%%%%%%%%%%%%%%%%%%%%%%%%% 
\definecolor{mymauve}{rgb}{0.58,0,0.82}
\definecolor{dkgreen}{rgb}{0,0.6,0}
\definecolor{ltgray}{rgb}{0.5,0.5,0.5}
\usepackage{caption} % Add the caption package

% Redefine the lstlisting format to remove the unwanted prefix
\DeclareCaptionFormat{mylst}{#1#2#3}
\DeclareCaptionFont{mycolor}{\color{red}}
\renewcommand\lstlistingname{Code Snippet:}
\renewcommand\lstlistlistingname{Code Snippet:}
%\DeclareCaptionStyle{listing} [justification=raggedright,indention=0pt, labelfont=bf]{}
%\captionsetup[lstlisting]{style=listing, labelsep=none}

\captionsetup[lstlisting]{format=mylst,labelfont={color=harvardcrimson},labelsep=space,justification=raggedright}

\lstset{%
	frame=tb,
	language=scala,
	aboveskip=3mm,
	belowskip=3mm,
	showstringspaces=false,
	columns=flexible,
	numbers=left,                   % where to put the line-numbers
	numberstyle=\tiny\color{gray},  % the style that is used for the line-numbers
	stepnumber=1,                   % the step between two line-numbers. If it's 1, each line will be numbered
	numbersep=5pt,                  % how far the line-numbers are from the code
	backgroundcolor=\color{white},  % choose the background color. You must add 
	keywordstyle=\color{blue},
	commentstyle=\color{dkgreen},
	%  stringstyle=\color{mauve},
	stringstyle=\color{myorange},
	frame=single,
	breaklines=true,
	breakatwhitespace=true,
	breakindent=20pt,
	tabsize=4,
	frameround=tttt,
	escapeinside={\%*}{*)},        % to add a comment within your code
	emph={count,take,textFile,filter,first,collect,mkString}, % Scala functions
	emphstyle={\color{mauve}},
	morekeywords ={val,sc},        % to add more keywords to the set  
	showspaces=false,
	showstringspaces=false,
	keepspaces=true
}

%%%%%%%%%%%%%%%%%%%%%%%%%%%%%%%%%%%%%%%%%%%%%%%%%%%%%%%%%%%%%%%%%%%%%%%%%%%%%%%%%%%%%%%% 
\lstset{%
	language=SQL,
	backgroundcolor=\color{white},
	basicstyle=\footnotesize,
	%breakatwhitespace=false,
	breaklines=true,
	captionpos=b,
	numbers=left,                   % where to put the line-numbers
	numberstyle=\tiny\color{gray},  % the style that is used for the line-numbers
	stepnumber=1,                   % the step between two line-numbers. If it's 1, each line will be numbered
	numbersep=5pt,                  % how far the line-numbers are from the code
	commentstyle=\color{dkgreen},
	%deletekeywords={...},
	%escapeinside={\%*}{*)},
	%extendedchars=true,
	frame=tb,
	keepspaces=false,
	keywordstyle=\color{blue},
	morekeywords={modify,MODIFY,ALL, ALTER, AND, ARRAY, AS, AUTHORIZATION, BETWEEN, BIGINT, BINARY, BOOLEAN, BOTH, BY, CASE, CAST, CHAR, COLUMN, CONF, CREATE, CROSS, CUBE, CURRENT, CURRENT_DATE, CURRENT_TIMESTAMP, CURSOR, DATABASE, DATE, DECIMAL, DELETE, DESCRIBE, DISTINCT, DOUBLE, DROP, ELSE, END, EXCHANGE, EXISTS, EXTENDED, EXTERNAL, FALSE, FETCH, FLOAT, FOLLOWING, FOR, FROM, FULL, FUNCTION, GRANT, GROUP, GROUPING, HAVING, IF, IMPORT, IN, INNER, INSERT, INT, INTERSECT, INTERVAL, INTO, IS, JOIN, LATERAL, LEFT, LESS, LIKE, LOCAL, MACRO, MAP, MORE, NONE, NOT, NULL, OF, ON, OR, ORDER, OUT, OUTER, OVER, PARTIALSCAN,PARTITIONED, STORED, TERMINATED, ROW, FORMAT, PARTITION,PERCENT, PRECEDING, PRESERVE, PROCEDURE, RANGE, READS, REDUCE, REVOKE, RIGHT, ROLLUP, ROW, ROWS, SELECT, SET, SMALLINT, TABLE, TABLESAMPLE, THEN, TIMESTAMP, TO, TRANSFORM, TRIGGER, TRUE, TRUNCATE, UNBOUNDED, UNION, UNIQUEJOIN, UPDATE, USER, USING, UTC_TMESTAMP, VALUES, VARCHAR, WHEN, WHERE, WINDOW, WITH, BY},
	numbers=left,
	numbersep=15pt,
	numberstyle=\tiny,
	rulecolor=\color{ltgray},
	showstringspaces=false,
	showtabs=false,
	stepnumber=1,
	tabsize=4,
	caption=Example SQL Query,
	showspaces=false,
	xleftmargin=4.0ex,	
	keepspaces=true
}

%%%%%%%%%%YML 
\newcommand\YAMLcolonstyle{\color{red}}
\newcommand\YAMLkeystyle{\color{black}}
\newcommand\YAMLvaluestyle{\color{blue}}

\makeatletter

% here is a macro expanding to the name of the language
% (handy if you decide to change it further down the road)
\newcommand\language@yaml{yaml}

\expandafter\expandafter\expandafter\lstdefinelanguage
\expandafter{\language@yaml}
{
  keywords={true,false,null,y,n,STAGE,PLANS,STAGE PLANS,Map, Reduce,Scan, Table,TableScan,Operator,Tree,alias,predicate},
  keywordstyle=\color{darkgray}\footnotesize,
  basicstyle=\footnotesize,
  basicstyle=\YAMLkeystyle,                                 % assuming a key comes first
  sensitive=false,
  comment=[l]{\#},
  morecomment=[s]{/*}{*/},
  commentstyle=\color{purple},
  stringstyle=\YAMLvaluestyle,
  moredelim=[l][\color{orange}]{\&},
  moredelim=[l][\color{magenta}]{*},
  moredelim=**[il][\YAMLcolonstyle{:}\YAMLvaluestyle]{:},   % switch to value style at :
  morestring=[b]',
  morestring=[b]",
  literate =    {---}{{\ProcessThreeDashes}}3
                {>}{{\textcolor{red}\textgreater}}1     
                {|}{{\textcolor{red}\textbar}}1 
                {\ -\ }{{\mdseries\ -\ }}3,
}

% switch to key style at EOL
\lst@AddToHook{EveryLine}{\ifx\lst@language\language@yaml\YAMLkeystyle\fi}
\makeatother

\newcommand\ProcessThreeDashes{\llap{\color{cyan}\mdseries-{-}-}}

%%%%%%%%%%%%%%%%%%%%%%%%%%%%%%%%%%%%%%%%%%%%%%%%%%%%%%%%%%%%%%%%%%%%%%%%%%%

\newcommand\JSONnumbervaluestyle{\color{red}}
\newcommand\JSONstringvaluestyle{\color{red}}

% switch used as state variable
\newif\ifcolonfoundonthisline

\makeatletter

\lstdefinestyle{json}
{
	showstringspaces    = false,
	keywords            = {false,true},
alsoletter          = 0123456789.,
morestring          = [s]{"}{"},
framextopmargin=3pt,
stringstyle         = \ifcolonfoundonthisline\JSONstringvaluestyle\fi,
MoreSelectCharTable =%
	\lst@DefSaveDef{`:}\colon@json{\processColon@json},
basicstyle          = \ttfamily,
keywordstyle        = \ttfamily\bfseries,
}

% flip the switch if a colon is found in Pmode
\newcommand\processColon@json{%
	\colon@json%
	\ifnum\lst@mode=\lst@Pmode%
	\global\colonfoundonthislinetrue%
	\fi
}

\lst@AddToHook{Output}{%
	\ifcolonfoundonthisline%
	\ifnum\lst@mode=\lst@Pmode%
	\def\lst@thestyle{\JSONnumbervaluestyle}%
	\fi
	\fi
%override by keyword style if a keyword is detected!
	\lsthk@DetectKeywords% 
}

% reset the switch at the end of line
\lst@AddToHook{EOL}%
{\global\colonfoundonthislinefalse}
%%%%%%%%%%%%%%%%%%%%%%%%%%%%%%%%%%%%%%%%%

%%%%%%%%%%%%%%%%%%%%%%%%%%%%%%%%%%%%%%%%%%%%%%%%%%%%%%%%%%%%%%%%%%%%%%%%%%%
%%% Local Variables:
%%% mode: latex
%%% TeX-master: "../main"
% !TeX root = ../main.tex
%%% TeX-engine: xetex
%%% End: