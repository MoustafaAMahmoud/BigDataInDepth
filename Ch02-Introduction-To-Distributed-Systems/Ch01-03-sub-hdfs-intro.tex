%%%%%%%%%%%%%%%%%%%%%%%%%%%%%%%%%%%%%%%%%%%%%%%%%%%%%%
\begin{frame}[c]{ }
	\frametitle{Hadoop Core Components }
	\centering     
	
	\textcolor{offgreen}{ \large Hadoop Core Components}
\end{frame}
%%%%%%%%%%%%%%%%%%%%%%%%%%%%%%%%%%%%%%%%%%%%%%%%%%%%%%
\begin{frame}[c]{ }
	\frametitle{Hadoop Core Components }
	
	
	\begin{itemize}  [<+->]
		\item [--] HDFS.
		\item [--] Map-Reduce.
		\item [--] YARN.
		
	\end{itemize}
\end{frame}

%%%%%%%%%%%%%%%%%%%%%%%%%%%%%%%%%%%%%%%%%%%%%%%%%%%%%%
\begin{frame}[c]{ }
	\frametitle{HDFS }
	
	
	\begin{itemize}  [<+->]
		\item [--] {\footnotesize HDFS is responsible for storing the data on the Hadoop cluster.}
		\item [--] {\footnotesize Data is split into blocks with configurable block size, for example, 64MB, 128MB, and 512MB.}
		\item [--] {\footnotesize Each data block is replicated and distributed across the cluster data node. This replication is configurable, and by default, three replica (folds).}
		\item [--] {\footnotesize Each block is stored in three different nodes. It is recommended to have two nodes in the same rack and the third one in a different rack.}
		\item [--] {\footnotesize A \textit{NameNode} keeps track of the location of the blocks and which blocks make up these files. These details known as \textit{metadata}.}
	\end{itemize}
\end{frame}
%%%%%%%%%%%%%%%%%%%%%%%%%%%%%%%%%%%%%%%%%%%%%%%%%%%%%%
\begin{frame}[c]{ }
	\frametitle{HDFS }
		\begin{figure}
		\centering
		


\tikzset{every picture/.style={line width=0.75pt}} %set default line width to 0.75pt        

\begin{tikzpicture}[x=0.6pt,y=0.6pt,yscale=-1,xscale=1]
%uncomment if require: \path (0,300); %set diagram left start at 0, and has height of 300

%Shape: Rectangle [id:dp893573566833787] 
\draw  [fill=offgreen2  ,fill opacity=1 ] (3.17,163.17) -- (291.33,163.17) -- (291.33,294.5) -- (3.17,294.5) -- cycle ;
%Shape: Rectangle [id:dp46281852294984216] 
\draw  [fill=aliceblue ,fill opacity=1 ] (8.33,170.33) -- (234.33,170.33) -- (234.33,271.33) -- (8.33,271.33) -- cycle ;
%Shape: Rectangle [id:dp6261320843201246] 
\draw   (10,62) -- (80,62) -- (80,102) -- (10,102) -- cycle ;
%Snip Single Corner Rect [id:dp12418640338729592] 
\draw  [fill=brown  ,fill opacity=1 ]  (211,61) -- (137,61) -- (123.33,47) -- (123.33,21) -- (211,21) -- cycle ;
%Snip Single Corner Rect [id:dp30064468613598183] 
\draw  [fill=brown  ,fill opacity=1 ]  (211,141) -- (137,141) -- (123.33,127) -- (123.33,101) -- (211,101) -- cycle ;
%Straight Lines [id:da7812791931414549] 
\draw  [->]  (81.33,82.33) -- (123,47) ;
%Straight Lines [id:da010110675330870511] 
\draw  [->]  (81.33,82.33) -- (123,127) ;
%Rounded Rect [id:dp35954418736480753] 
\draw  [fill=offgreen  ,fill opacity=1 ] (298,81.67) .. controls (298,52.03) and (322.03,28) .. (351.67,28) -- (603.67,28) .. controls (633.31,28) and (657.33,52.03) .. (657.33,81.67) -- (657.33,242.67) .. controls (657.33,272.31) and (633.31,296.33) .. (603.67,296.33) -- (351.67,296.33) .. controls (322.03,296.33) and (298,272.31) .. (298,242.67) -- cycle ;

%Shape: Rectangle [id:dp8687762741846744] 
\draw  [fill=ballblue  ,fill opacity=1 ] (327,55) -- (478.33,55) -- (478.33,276.33) -- (327,276.33) -- cycle ;
%Shape: Rectangle [id:dp7548707221284656] 
\draw  [fill=spirodiscoball  ,fill opacity=1 ] (489,55) -- (640.33,55) -- (640.33,276.33) -- (489,276.33) -- cycle ;

%Rounded Rect [id:dp00897404914951061] 
\draw  [fill={rgb, 255:red, 237; green, 227; blue, 227 }  ,fill opacity=1 ] (355,84) .. controls (355,79.58) and (358.58,76) .. (363,76) -- (439.33,76) .. controls (443.75,76) and (447.33,79.58) .. (447.33,84) -- (447.33,108) .. controls (447.33,112.42) and (443.75,116) .. (439.33,116) -- (363,116) .. controls (358.58,116) and (355,112.42) .. (355,108) -- cycle ;

%Rounded Rect [id:dp9790517680808902] 
\draw  [fill={rgb, 255:red, 237; green, 227; blue, 227 }  ,fill opacity=1 ]  (355,144) .. controls (355,139.58) and (358.58,136) .. (363,136) -- (439.33,136) .. controls (443.75,136) and (447.33,139.58) .. (447.33,144) -- (447.33,168) .. controls (447.33,172.42) and (443.75,176) .. (439.33,176) -- (363,176) .. controls (358.58,176) and (355,172.42) .. (355,168) -- cycle ;
%Rounded Rect [id:dp013627781761065716] 
\draw  [fill={rgb, 255:red, 237; green, 227; blue, 227 }  ,fill opacity=1 ] (356,204) .. controls (356,199.58) and (359.58,196) .. (364,196) -- (438.33,196) .. controls (442.75,196) and (446.33,199.58) .. (446.33,204) -- (446.33,228) .. controls (446.33,232.42) and (442.75,236) .. (438.33,236) -- (364,236) .. controls (359.58,236) and (356,232.42) .. (356,228) -- cycle ;
%Rounded Rect [id:dp8935666136526693] 
\draw  [fill={rgb, 255:red, 237; green, 227; blue, 227 }  ,fill opacity=1 ] (520,85) .. controls (520,80.58) and (523.58,77) .. (528,77) -- (601.33,77) .. controls (605.75,77) and (609.33,80.58) .. (609.33,85) -- (609.33,109) .. controls (609.33,113.42) and (605.75,117) .. (601.33,117) -- (528,117) .. controls (523.58,117) and (520,113.42) .. (520,109) -- cycle ;

\draw  [fill={rgb, 255:red, 237; green, 227; blue, 227 }  ,fill opacity=1 ] (520,145) .. controls (520,140.58) and (523.58,137) .. (528,137) -- (601.33,137) .. controls (605.75,137) and (609.33,140.58) .. (609.33,145) -- (609.33,169) .. controls (609.33,173.42) and (605.75,177) .. (601.33,177) -- (528,177) .. controls (523.58,177) and (520,173.42) .. (520,169) -- cycle ;

%Rounded Rect [id:dp24192848422071778] 
\draw  [fill={rgb, 255:red, 237; green, 227; blue, 227 }  ,fill opacity=1 ] (521,205) .. controls (521,200.58) and (524.58,197) .. (529,197) -- (601.33,197) .. controls (605.75,197) and (609.33,200.58) .. (609.33,205) -- (609.33,229) .. controls (609.33,233.42) and (605.75,237) .. (601.33,237) -- (529,237) .. controls (524.58,237) and (521,233.42) .. (521,229) -- cycle ;


%Shape: Circle [id:dp6662144240379807] 
%%file left panel
\draw[fill={rgb, 255:red, 74; green, 144; blue, 226 }, thick] (223,28) circle (.2 cm);
\draw (218,22) node [anchor=north west][inner sep=0.75pt]  [color=white  ,opacity=1 ]  [font=\scriptsize]  [align=left] {1};

\draw[fill={rgb, 255:red, 245; green, 166; blue, 35 }, thick] (223,52) circle (.2 cm);
\draw (218,46) node [anchor=north west][inner sep=0.75pt]  [color=black  ,opacity=1 ]  [font=\scriptsize]  [align=left] {2};

\draw[fill={rgb, 255:red, 139; green, 87; blue, 42 }, thick] (223,98) circle (.2 cm);
\draw (218,92) node [anchor=north west][inner sep=0.75pt]  [color=white  ,opacity=1 ]  [font=\scriptsize]  [align=left] {3};

\draw[fill={rgb, 255:red, 126; green, 211; blue, 33 }, thick] (223,122) circle (.2 cm);
\draw (218,116) node [anchor=north west][inner sep=0.75pt]  [color=black  ,opacity=1 ]  [font=\scriptsize]  [align=left] {4};

\draw[fill={rgb, 255:red, 208; green, 2; blue, 27 }, thick] (223,146) circle (.2 cm);
\draw (218,140) node [anchor=north west][inner sep=0.75pt]  [color=white  ,opacity=1 ]  [font=\scriptsize]  [align=left] {5};


\draw[fill={rgb, 255:red, 74; green, 144; blue, 226 }, thick] (365,102) circle (.2 cm);
\draw (360,96) node [anchor=north west][inner sep=0.75pt]  [color=white  ,opacity=1 ]  [font=\scriptsize]  [align=left] {1};

\draw[fill={rgb, 255:red, 245; green, 166; blue, 35 }, thick] (390,102) circle (.2 cm);
\draw (385,96) node [anchor=north west][inner sep=0.75pt]  [color=black  ,opacity=1 ]  [font=\scriptsize]  [align=left] {2};

\draw[fill={rgb, 255:red, 126; green, 211; blue, 33 }, thick] (415,102) circle (.2 cm);
\draw (410,96) node [anchor=north west][inner sep=0.75pt]  [color=black  ,opacity=1 ]  [font=\scriptsize]  [align=left] {3};


\draw[fill={rgb, 255:red, 74; green, 144; blue, 226 }, thick] (365,163) circle (.2 cm);
\draw (360,157) node [anchor=north west][inner sep=0.75pt]  [color=white  ,opacity=1 ]  [font=\scriptsize]  [align=left] {1};

\draw[fill={rgb, 255:red, 245; green, 166; blue, 35 }, thick] (390,163) circle (.2 cm);
\draw (385,157) node [anchor=north west][inner sep=0.75pt]  [color=black  ,opacity=1 ]  [font=\scriptsize]  [align=left] {2};

\draw[fill={rgb, 255:red, 208; green, 2; blue, 27 }, thick] (415,163) circle (.2 cm);
\draw (410,157) node [anchor=north west][inner sep=0.75pt]  [color=white  ,opacity=1 ]  [font=\scriptsize]  [align=left] {5};

\draw[fill={rgb, 255:red, 126; green, 211; blue, 33 }, thick] (365,224) circle (.2 cm);
\draw (360,218) node [anchor=north west][inner sep=0.75pt]  [color=black  ,opacity=1 ]  [font=\scriptsize]  [align=left] {1};

\draw[fill={rgb, 255:red, 139; green, 87; blue, 42 }, thick] (390,224) circle (.2 cm);
\draw (385,218) node [anchor=north west][inner sep=0.75pt]  [color=white  ,opacity=1 ]  [font=\scriptsize]  [align=left] {1};


\draw[fill={rgb, 255:red, 74; green, 144; blue, 226 }, thick] (530,102) circle (.2 cm);
\draw (525,96) node [anchor=north west][inner sep=0.75pt]  [color=white  ,opacity=1 ]  [font=\scriptsize]  [align=left] {1};

\draw[fill={rgb, 255:red, 139; green, 87; blue, 42 }, thick] (555,102) circle (.2 cm);
\draw (550,96) node [anchor=north west][inner sep=0.75pt]  [color=white  ,opacity=1 ]  [font=\scriptsize]  [align=left] {3};

\draw[fill={rgb, 255:red, 208; green, 2; blue, 27 }, thick] (580,102) circle (.2 cm);
\draw (575,96) node [anchor=north west][inner sep=0.75pt]  [color=white  ,opacity=1 ]  [font=\scriptsize]  [align=left] {5};


\draw[fill={rgb, 255:red, 245; green, 166; blue, 35 }, thick] (530,163) circle (.2 cm);
\draw (525,157) node [anchor=north west][inner sep=0.75pt]  [color=black  ,opacity=1 ]  [font=\scriptsize]  [align=left] {2};

\draw[fill={rgb, 255:red, 139; green, 87; blue, 42 }, thick] (555,163) circle (.2 cm);
\draw (550,157) node [anchor=north west][inner sep=0.75pt]  [color=white  ,opacity=1 ]  [font=\scriptsize]  [align=left] {3};


\draw[fill={rgb, 255:red, 126; green, 211; blue, 33 }, thick] (530,224) circle (.2 cm);
\draw (525,218) node [anchor=north west][inner sep=0.75pt]  [color=black  ,opacity=1 ]  [font=\scriptsize]  [align=left] {4};

\draw[fill={rgb, 255:red, 208; green, 2; blue, 27 }, thick] (555,224) circle (.2 cm);
\draw (550,218) node [anchor=north west][inner sep=0.75pt]  [color=white  ,opacity=1 ]  [font=\scriptsize]  [align=left] {5};



% Text Node
\draw (25,75) node [anchor=north west][inner sep=0.75pt]   [font=\scriptsize] [align=left] {client};
% Text Node
\draw (126,24) node [anchor=north west][inner sep=0.75pt]  [font=\scriptsize] [align=left] {/sales/\\20210521.csv};
% Text Node
\draw (126,104) node [anchor=north west][inner sep=0.75pt]  [font=\scriptsize] [align=left] {/sales/\\20210522.csv};
% Text Node
\draw (145,7) node [anchor=north west][inner sep=0.75pt]  [font=\scriptsize] [align=left]  [font=\scriptsize]  {502MB};
% Text Node
\draw (145,88) node [anchor=north west][inner sep=0.75pt]  [font=\scriptsize] [align=left]  [font=\scriptsize]  {768MB};
% Text Node
\draw (358,79) node [anchor=north west][inner sep=0.75pt]  [font=\scriptsize,color=black] [align=left] {Data Node A};
% Text Node
\draw (358,139) node [anchor=north west][inner sep=0.75pt]  [font=\scriptsize,color=black] [align=left] {Data Node B};
% Text Node
\draw (358,197) node [anchor=north west][inner sep=0.75pt]  [font=\scriptsize,color=black] [align=left] {Data Node C};
% Text Node
\draw (523,79) node [anchor=north west][inner sep=0.75pt]  [font=\scriptsize,color=black] [align=left] {Data Node D};
% Text Node
\draw (523,139) node [anchor=north west][inner sep=0.75pt]  [font=\scriptsize,color=black] [align=left] {Data Node E};
% Text Node
\draw (523,197) node [anchor=north west][inner sep=0.75pt]  [font=\scriptsize,color=black] [align=left] {Data Node F};
% Text Node
\draw (430,33) node [anchor=north west][inner sep=0.75pt]  [font=\small ,color=black] [align=left] {HDFS Cluster};
% Text Node
\draw (429,256) node [anchor=north west][inner sep=0.75pt]   [font=\scriptsize,color=black]  [align=left] {RAC 1};
% Text Node
\draw (593,257) node [anchor=north west][inner sep=0.75pt]  [font=\scriptsize,color=black]   [align=left] {RAC 2};

% Text Node
\draw (236,20) node [anchor=north west][inner sep=0.75pt]  [font=\scriptsize] [align=left]  [font=\scriptsize]  {256MB};
% Text Node
\draw (237,47) node [anchor=north west][inner sep=0.75pt]  [font=\scriptsize] [align=left]  [font=\scriptsize]  {246MB};

% Text Node
\draw (235,92) node [anchor=north west][inner sep=0.75pt]  [font=\scriptsize] [align=left] {256MB};
% Text Node
\draw (235,116) node [anchor=north west][inner sep=0.75pt]  [font=\scriptsize] [align=left] {256MB};
% Text Node
\draw (235,140) node [anchor=north west][inner sep=0.75pt]  [font=\scriptsize] [align=left] {256MB};

% Text Node
\draw (19,187) node [anchor=north west][inner sep=0.75pt]  [font=\footnotesize,color={rgb, 255:red, 65; green, 117; blue, 5 }  ,opacity=1 ]  [font=\scriptsize]  [align=left] {Metadata};
% Text Node
\draw (15,208) node [anchor=north west][inner sep=0.75pt]   [align=left] [color=black] {{\tiny /sales/20210521.csv: B1, B2}\\{\tiny /sales/20210521.csv: B3, B4, B5}};
% Text Node
\draw (172,171) node [anchor=north west][inner sep=0.75pt]   [align=left] [color=black] {\tiny B1: A, B, D};
\draw (172,184) node [anchor=north west][inner sep=0.75pt]   [align=left] [color=black] {\tiny B2: A, B, E};
\draw (172,197) node [anchor=north west][inner sep=0.75pt]   [align=left] [color=black] {\tiny B3: C, D, E};
\draw (172,210) node [anchor=north west][inner sep=0.75pt]   [align=left] [color=black] {\tiny B4: A, C, F};
\draw (172,223) node [anchor=north west][inner sep=0.75pt]   [align=left] [color=black] {\tiny B5: B, D, F};

% Text Node
\draw (243.33,202.6) node [anchor=north west][inner sep=0.75pt]  [font=\scriptsize,color=black] [align=left] {Name\\Node};


\end{tikzpicture}
%%%%%%%%%%%%%%%%%%%%%%%%%%%%%%%%%%%%%%%%%%%%%%%%%%%%%%%%%%%%%%%%%%%%%%%%%%%
%%% Local Variables:
%%% mode: latex
%%% TeX-master: "../../main.tex"
% !TeX root = ../../main.tex
%%% TeX-engine: xetex
%%% End:

		\caption{HDFS } \label{fig:hdfs}
	\end{figure}
\end{frame}
%%%%%%%%%%%%%%%%%%%%%%%%%%%%%%%%%%%%%%%%%%%%%%%%%%%%%%
\begin{frame}[c]{ }
	\frametitle{HDFS }
	
	
	\begin{itemize}  [<+->]
		\item [--] {\footnotesize Hadoop cluster contains NameNodes and DataNodes.}
		\item [--] {\footnotesize NameNodes daemon must be running at all times. A daemon is simply a program running on a node. }
		\item [--] {\footnotesize Hadoop cluster contains at least two NameNodes Active/Standby nodes.}
		\item [--] {\footnotesize HDFS files are \textit{write one}, so we can't do any random writes.}
		\item [--] {\footnotesize HDFS is optimized for large files. If we have many small files we could face a problem \textit{Hadoop small files problem}}
			\end{itemize}
			\footnotetext[1]{Small file problems \href{https://blog.cloudera.com/the-small-files-problem/}{https://blog.cloudera.com/the-small-files-problem/}	} 
		
\end{frame}
%%%%%%%%%%%%%%%%%%%%%%%%%%%%%%%%%%%%%%%%%%%%%%%%%%%%%%
\begin{frame}[c]{ }
	\frametitle{Access HDFS }
	
	
	\begin{itemize}  [<+->]
		\item [--] {\footnotesize To acess HDFS we use Hadoop APIs. }
		\item [--] {\footnotesize These APIs provide various functionality over HDFS . }
		\item [--] {\footnotesize We can use the command line "FsShell" or call the API through MapReduce, Spark, or Other Restful interfaces.}
		
		
	\end{itemize}
	\footnotetext[1]{HDFS Commands \href{https://hadoop.apache.org/docs/r2.7.1/hadoop-project-dist/hadoop-common/FileSystemShell.html}{https://hadoop.apache.org/docs/r2.7.1/hadoop-project-dist/hadoop-common/FileSystemShell.html}	} 
	
\end{frame}
