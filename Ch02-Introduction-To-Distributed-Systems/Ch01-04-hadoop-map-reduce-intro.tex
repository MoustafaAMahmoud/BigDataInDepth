
%%%%%%%%%%%%%%%%%%%%%%%%%%%%%%%%%%%%%%%%%%%%%%%%%%%%%%
\begin{frame}[c]{ }
	\frametitle{Hadoop Core Concepts }
	
	
	\begin{itemize}  [<+->]
		\item [--] HDFS.
		\item [--] YARN.
		\item [--] Map-Reduce.
		
	\end{itemize}
\end{frame}
%%%%%%%%%%%%%%%%%%%%%%%%%%%%%%%%%%%%%%%%%%%%%%%%%%%%%%
\begin{frame}[c]{ }
	\frametitle{ Hadoop Map Reduce}
	\centering     
	
	\textcolor{offgreen}{ \large Introduction To Hadoop Map Reduce API}
\end{frame}
%%%%%%%%%%%%%%%%%%%%%%%%%%%%%%%%%%%%%%%%%%%%%%%%%%%%%%	

\begin{frame}
\frametitle{The basic idea of MapReduce}
We break this into three stages
\begin{itemize}  [<+->]
	\item Map.
	\item Shuffle/Group (Mapper Intermediates).
	\item Reduce			
\end{itemize}
\footnotetext[1]{This example taken from  \href{https://reberhardt.com/cs110/summer-2018/lecture-notes/lecture-14/}{https://reberhardt.com/cs110/summer-2018/lecture-notes/lecture-14/}	} 
\end{frame}
%%%%%%%%%%%%%%%%%%%%%%%%%%%%%%%%%%%%%%%%%%%%%%%%%%%%%%
\begin{frame}
\frametitle{Map}
We distribute our raw ingredients amongst the workers.
\begin{figure}
	\includegraphics[width=.5\textwidth,height=.7\textheight]{./Figures/chapter-02/map-reduce-map-side.jpeg}
\end{figure}			
\footnotetext[1]{{\tiny This example taken from  \href{https://reberhardt.com/cs110/summer-2018/lecture-notes/lecture-14/}{https://reberhardt.com/cs110/summer-2018/lecture-notes/lecture-14/}	} }
\end{frame}
%%%%%%%%%%%%%%%%%%%%%%%%%%%%%%%%%%%%%%%%%%%%%%%%%%%%%%
\begin{frame}
\frametitle{Shuffle/Group}
We will organise and group the processed ingredients into piles, so that making a sandwich becomes easy.
\begin{figure}
	\includegraphics[width=.7\textwidth,height=.64\textheight]{./Figures/chapter-02/map-reduce-shuffle.png}
\end{figure}			
\footnotetext[1]{{\tiny This example taken from  \href{https://reberhardt.com/cs110/summer-2018/lecture-notes/lecture-14/}{https://reberhardt.com/cs110/summer-2018/lecture-notes/lecture-14/}	}} 
\end{frame}
%%%%%%%%%%%%%%%%%%%%%%%%%%%%%%%%%%%%%%%%%%%%%%%%%%%%%%
\begin{frame}
\frametitle{Reduce}
we’ll combine the ingredients into a sandwich
\begin{figure}
	\includegraphics[width=.96\textwidth,height=.7\textheight]{./Figures/chapter-02/map-reduce-reduce-side.png}
\end{figure}			
\footnotetext[1]{ {\tiny This example taken from  \href{https://reberhardt.com/cs110/summer-2018/lecture-notes/lecture-14/}{https://reberhardt.com/cs110/summer-2018/lecture-notes/lecture-14/}	}} 
\end{frame}
%%%%%%%%%%%%%%%%%%%%%%%%%%%%%%%%%%%%%%%%%%%%%%%%%%%%%%
\begin{frame}[plain,c]
	\frametitle{Case Study Example 1}
	\begin{figure}
		\centering
		


\tikzset{every picture/.style={line width=0.75pt}} %set default line width to 0.75pt        

\begin{tikzpicture}[x=0.75pt,y=0.75pt,yscale=-1,xscale=1]
%uncomment if require: \path (0,354); %set diagram left start at 0, and has height of 354

%input
%Shape: Rectangle [id:dp3834116102870988] 
\draw  [color={rgb, 255:red, 74; green, 144; blue, 226 }  ,draw opacity=1 ][line width=1]  (2,152) -- (91,152) -- (91,184) -- (2,184) -- cycle ;

%Straight Lines [id:da7265782866155249] ??? -> Node1
\draw [color={rgb, 255:red, 155; green, 155; blue, 155 }  ,draw opacity=1 ][line width=0.75]    (231,152) -- (294.26,117.99) ;
\draw [shift={(296,117)}, rotate = 510.35] [color={rgb, 255:red, 155; green, 155; blue, 155 }  ,draw opacity=1 ][line width=0.75]    (10.93,-3.29) .. controls (6.95,-1.4) and (3.31,-0.3) .. (0,0) .. controls (3.31,0.3) and (6.95,1.4) .. (10.93,3.29)   ;


%Shape: Frame [id:dp3239595123755211] 
\draw  [line width=0.75,color=offwhite]  (296,77) -- (360,77) -- (360,117) -- (296,117) -- cycle(354,83) -- (302,83) -- (302,111) -- (354,111) -- cycle ;
%Shape: Trapezoid [id:dp6832413050712917] 
\draw  [line width=0.75,color=offwhite]  (286,147) -- (295,117) -- (361,117) -- (370,147) -- cycle ;
%Shape: Trapezoid [id:dp7279166981904915] 
\draw  [line width=0.75,color=offwhite]  (290,143) -- (297.5,118) -- (358.5,118) -- (366,143) -- cycle ;

%Shape: Rectangle [id:dp6681325814864821] 
\draw  [color={rgb, 255:red, 70; green, 155; blue, 36 }  ,draw opacity=1 ][line width=1]  (444,152) -- (536,152) -- (536,181) -- (444,181) -- cycle ;

%Node1->output
%Straight Lines [id:da3966973170014948] %Node1->output
\draw [color={rgb, 255:red, 155; green, 155; blue, 155 }  ,draw opacity=1 ][line width=0.75]    (361,117) -- (438,149) ;
\draw [shift={(440,150)}, rotate = 200.48] [color={rgb, 255:red, 155; green, 155; blue, 155 }  ,draw opacity=1 ][line width=0.75]    (10.93,-3.29) .. controls (6.95,-1.4) and (3.31,-0.3) .. (0,0) .. controls (3.31,0.3) and (6.95,1.4) .. (10.93,3.29)   ;


%Node2
%Shape: Frame [id:dp7623924046060552] 
\draw  [line width=0.75,color=offwhite]  (296,193) -- (360,193) -- (360,233) -- (296,233) -- cycle(354,199) -- (302,199) -- (302,227) -- (354,227) -- cycle ;
%Shape: Trapezoid [id:dp7365747253703173] 
\draw  [line width=0.75,color=offwhite]  (286,263) -- (295,233) -- (361,233) -- (370,263) -- cycle ;
%Shape: Trapezoid [id:dp5045346305423993] 
\draw  [line width=0.75,color=offwhite]  (290,259) -- (297.5,234) -- (358.5,234) -- (366,259) -- cycle ;

%Straight Lines [id:da15205683840525686]  ??? -> Node2
\draw [color={rgb, 255:red, 155; green, 155; blue, 155 }  ,draw opacity=1 ][line width=0.75]    (216.75,183) -- (292.32,231.91) ;
\draw [shift={(294,233)}, rotate = 212.91] [color={rgb, 255:red, 155; green, 155; blue, 155 }  ,draw opacity=1 ][line width=0.75]    (10.93,-3.29) .. controls (6.95,-1.4) and (3.31,-0.3) .. (0,0) .. controls (3.31,0.3) and (6.95,1.4) .. (10.93,3.29)   ;

%Straight Lines [id:da269400866080729] Node2 -> Output
\draw [color={rgb, 255:red, 155; green, 155; blue, 155 }  ,draw opacity=1 ][line width=0.75]    (359,233) -- (438,185) ;
\draw [shift={(440,184)}, rotate = 508.54] [color={rgb, 255:red, 155; green, 155; blue, 155 }  ,draw opacity=1 ][line width=0.75]    (10.93,-3.29) .. controls (6.95,-1.4) and (3.31,-0.3) .. (0,0) .. controls (3.31,0.3) and (6.95,1.4) .. (10.93,3.29)   ;

%????
%Flowchart: Data [id:dp13666823470676226] 
\draw  [line width=1,color=offred]  (169.25,152) -- (231,152) -- (216.75,183) -- (155,183) -- cycle ;
%Straight Lines [id:da5122596435530213] 
\draw [color={rgb, 255:red, 128; green, 128; blue, 128 }  ,draw opacity=1 ]   (94,170) -- (154,169.03) ;
\draw [shift={(156,169)}, rotate = 539.0799999999999] [color={rgb, 255:red, 128; green, 128; blue, 128 }  ,draw opacity=1 ][line width=0.75]    (10.93,-3.29) .. controls (6.95,-1.4) and (3.31,-0.3) .. (0,0) .. controls (3.31,0.3) and (6.95,1.4) .. (10.93,3.29)   ;

% Text Node
\draw (24,190) node [anchor=north west]  [font=\footnotesize,color={rgb, 255:red, 74; green, 144; blue, 226 }  ,opacity=1 ]  {Input};
% Text Node
\draw (12,158) node [anchor=north west] [font=\scriptsize,color={rgb, 255:red, 74; green, 144; blue, 226 }  ,opacity=1 ]  {Hello world};
% Text Node
\draw (304,155) node [anchor=north west] [font=\footnotesize,color=offwhite]  {Node 1};
% Text Node
\draw (445,158) node [anchor=north west] [font=\scriptsize,color={rgb, 255:red, 70; green, 155; blue, 36 }  ,opacity=1 ]  {HELLO WORLD};
% Text Node
\draw (465,190) node [anchor=north west]  [font=\footnotesize,color={rgb, 255:red, 70; green, 155; blue, 36 }  ,opacity=1 ]  {Output};
% Text Node
\draw (302,266) node [anchor=north west] [font=\footnotesize,color=offwhite]  {Node 2};
% Text Node
\draw (175,158) node [anchor=north west]  [font=\scriptsize,color=offred,align=left] {???};


\end{tikzpicture}

		\caption{Convert text to upper text, for example, The -> THE } \label{fig:DS3}
	\end{figure}
	
\end{frame}
%%%%%%%%%%%%%%%%%%%%%%%%%%%%%%%%%%%%%%%%%%%%%%%%%%%%%%
\begin{frame}[plain,c]
	\frametitle{Case Study Example 1}
	\begin{figure}
		\centering
		


\tikzset{every picture/.style={line width=0.75pt}} %set default line width to 0.75pt        

\begin{tikzpicture}[scale=0.95,x=0.75pt,y=0.75pt,yscale=-1,xscale=1]
	%uncomment if require: \path (0,300); %set diagram left start at 0, and has height of 300
	
	%Shape: Rectangle [id:dp6258708726014198] 
	\draw  [color={rgb, 255:red, 74; green, 144; blue, 226 }  ,draw opacity=1 ][line width=1.5]  (60,151) -- (100,151) -- (100,189.67) -- (60,189.67) -- cycle ;

    %Node1
	%Shape: Frame [id:dp45199805556426453] 
	\draw  [line width=0.75,color=offwhite]  (445,57) -- (509,57) -- (509,97) -- (445,97) -- cycle(503,63) -- (451,63) -- (451,91) -- (503,91) -- cycle ;
	%Shape: Trapezoid [id:dp6365442083669953] 
	\draw  [line width=0.75,color=offwhite]  (435,127) -- (444,97) -- (510,97) -- (519,127) -- cycle ;
	%Shape: Trapezoid [id:dp7035229008077436] 
	\draw  [line width=0.75,color=offwhite]  (439,123) -- (446.5,98) -- (507.5,98) -- (515,123) -- cycle ;
	
    %Shape: Rectangle [id:dp4075005896181748] 
	\draw  [color={rgb, 255:red, 70; green, 155; blue, 36 }  ,draw opacity=1 ][line width=1.5]  (560.75,151) -- (610,151) -- (610,189) -- (560.75,189) -- cycle ;
	
    %output
    %Straight Lines [id:da7123818425757267] 
	\draw [color={rgb, 255:red, 128; green, 128; blue, 128 }  ,draw opacity=1 ]   (101,170) -- (128.88,169.84) ;
	\draw [shift={(130.88,169.83)}, rotate = 539.6800000000001] [color={rgb, 255:red, 128; green, 128; blue, 128 }  ,draw opacity=1 ][line width=0.75]    (10.93,-3.29) .. controls (6.95,-1.4) and (3.31,-0.3) .. (0,0) .. controls (3.31,0.3) and (6.95,1.4) .. (10.93,3.29)   ;
	
    %Shape: Rectangle [id:dp9860882909394741] 
	\draw  [color=offred  ,draw opacity=1 ][line width=0.75]  (230.75,119.63) -- (300.75,119.63) -- (300.75,149.63) -- (230.75,149.63) -- cycle ;
	%Shape: Rectangle [id:dp6809508870305205] 
	\draw  [color=offred  ,draw opacity=1 ][line width=0.75]  (229.75,189.63) -- (300.75,189.63) -- (300.75,219.63) -- (229.75,219.63) -- cycle ;
	
    %Curve Lines [id:da19696995243819926] 
	\draw [color={rgb, 255:red, 128; green, 128; blue, 128 }  ,draw opacity=1 ]   (169.75,190.63) .. controls (177.42,212.05) and (183.62,223.4) .. (228.38,219.75) ;
	\draw [shift={(229.75,219.63)}, rotate = 535.03] [color={rgb, 255:red, 128; green, 128; blue, 128 }  ,draw opacity=1 ][line width=0.75]    (10.93,-3.29) .. controls (6.95,-1.4) and (3.31,-0.3) .. (0,0) .. controls (3.31,0.3) and (6.95,1.4) .. (10.93,3.29)   ;
	%Curve Lines [id:da14087388365596432] 
	\draw [color={rgb, 255:red, 128; green, 128; blue, 128 }  ,draw opacity=1 ]   (169.75,149.63) .. controls (169.75,139.73) and (185.43,114.15) .. (229.41,119.46) ;
	\draw [shift={(230.75,119.63)}, rotate = 187.59] [color={rgb, 255:red, 128; green, 128; blue, 128 }  ,draw opacity=1 ][line width=0.75]    (10.93,-3.29) .. controls (6.95,-1.4) and (3.31,-0.3) .. (0,0) .. controls (3.31,0.3) and (6.95,1.4) .. (10.93,3.29)   ;
	%Curve Lines [id:da6655261202271237] 
	\draw [color={rgb, 255:red, 128; green, 128; blue, 128 }  ,draw opacity=1 ]   (300.75,119.63) .. controls (311.59,149.18) and (302.77,169.68) .. (336.19,170.64) ;
	\draw [shift={(337.75,170.67)}, rotate = 180.6] [color={rgb, 255:red, 128; green, 128; blue, 128 }  ,draw opacity=1 ][line width=0.75]    (10.93,-3.29) .. controls (6.95,-1.4) and (3.31,-0.3) .. (0,0) .. controls (3.31,0.3) and (6.95,1.4) .. (10.93,3.29)   ;
	%Curve Lines [id:da188937268210565] 
	\draw [color={rgb, 255:red, 128; green, 128; blue, 128 }  ,draw opacity=1 ]   (300.75,219.63) .. controls (310.55,191.21) and (304.74,171.12) .. (335.8,170.67) ;
	\draw [shift={(337.75,170.67)}, rotate = 180.63] [color={rgb, 255:red, 128; green, 128; blue, 128 }  ,draw opacity=1 ][line width=0.75]    (10.93,-3.29) .. controls (6.95,-1.4) and (3.31,-0.3) .. (0,0) .. controls (3.31,0.3) and (6.95,1.4) .. (10.93,3.29)   ;
	
    %Shape: Rectangle [id:dp795964462639149] 
	\draw  [color=offyellow  ,draw opacity=1 ] (131,151) -- (201,151) -- (201,190) -- (131,190) -- cycle ;

	%Shape: Frame [id:dp7417485861438164] 
	\draw  [line width=0.75,color=offwhite]  (443,212) -- (507,212) -- (507,252) -- (443,252) -- cycle(501,218) -- (449,218) -- (449,246) -- (501,246) -- cycle ;
	%Shape: Trapezoid [id:dp7528630303015812] 
	\draw  [line width=0.75,color=offwhite]  (433,282) -- (442,252) -- (508,252) -- (517,282) -- cycle ;
	%Shape: Trapezoid [id:dp8885841590443699] 
	\draw  [line width=0.75,color=offwhite]  (437,278) -- (444.5,253) -- (505.5,253) -- (513,278) -- cycle ;

    %????
	%Shape: Rectangle [id:dp6221340422393318] 
	\draw  [color=offpurple,line width=0.75] (340,151) -- (389.75,151) -- (389.75,189.67) -- (340,189.67) -- cycle ;

	%Curve Lines [id:da17799352827092774] 
	\draw [color={rgb, 255:red, 128; green, 128; blue, 128 }  ,draw opacity=1 ]   (390,170.17) .. controls (415.99,126.11) and (392.72,97.25) .. (442.47,97) ;
	\draw [shift={(444,97)}, rotate = 180.37] [color={rgb, 255:red, 128; green, 128; blue, 128 }  ,draw opacity=1 ][line width=0.75]    (10.93,-3.29) .. controls (6.95,-1.4) and (3.31,-0.3) .. (0,0) .. controls (3.31,0.3) and (6.95,1.4) .. (10.93,3.29)   ;
	
    %Curve Lines [id:da9685723605797839] 
	\draw [color={rgb, 255:red, 128; green, 128; blue, 128 }  ,draw opacity=1 ]   (390,170.17) .. controls (415.86,237.64) and (399.75,251.26) .. (440.12,251.98) ;
	\draw [shift={(442,252)}, rotate = 180.45] [color={rgb, 255:red, 128; green, 128; blue, 128 }  ,draw opacity=1 ][line width=0.75]    (10.93,-3.29) .. controls (6.95,-1.4) and (3.31,-0.3) .. (0,0) .. controls (3.31,0.3) and (6.95,1.4) .. (10.93,3.29)   ;
	
    %Curve Lines [id:da43813209116711305] 
	\draw [color={rgb, 255:red, 128; green, 128; blue, 128 }  ,draw opacity=1 ]   (510,97) .. controls (541.27,124.58) and (518.46,169.62) .. (557,172.88) ;
	\draw [shift={(558,173)}, rotate = 182.66] [color={rgb, 255:red, 128; green, 128; blue, 128 }  ,draw opacity=1 ][line width=0.75]    (10.93,-3.29) .. controls (6.95,-1.4) and (3.31,-0.3) .. (0,0) .. controls (3.31,0.3) and (6.95,1.4) .. (10.93,3.29)   ;
	
    %Curve Lines [id:da5479643719173531] 
	\draw [color={rgb, 255:red, 128; green, 128; blue, 128 }  ,draw opacity=1 ]   (508,252) .. controls (536.96,209.1) and (511.27,191.36) .. (557,173.54) ;
	\draw [shift={(558,173)}, rotate = 520.2] [color={rgb, 255:red, 128; green, 128; blue, 128 }  ,draw opacity=1 ][line width=0.75]    (10.93,-3.29) .. controls (6.95,-1.4) and (3.31,-0.3) .. (0,0) .. controls (3.31,0.3) and (6.95,1.4) .. (10.93,3.29)   ;

    %Shape: Cloud [id:dp9284795120380853] 
    \draw  [color={rgb, 255:red, 80; green, 227; blue, 194 }  ,draw opacity=1 ][dash pattern={on 1.69pt off 2.76pt}][line width=1.5]  (313.65,190.77) .. controls (314.97,206.06) and (308.48,220.94) .. (296.96,229.1) .. controls (285.42,237.25) and (270.87,237.25) .. (259.47,229.09) .. controls (255.04,237.81) and (247.28,243.67) .. (238.53,244.91) .. controls (229.77,246.14) and (221.06,242.6) .. (215.02,235.35) .. controls (211.23,243.26) and (204.14,248.43) .. (196.27,249.02) .. controls (188.39,249.6) and (180.84,245.52) .. (176.3,238.22) .. controls (169.66,246.52) and (159.43,249.77) .. (150.03,246.56) .. controls (140.63,243.35) and (133.76,234.26) .. (132.38,223.21) .. controls (124.67,220.53) and (118.39,214.17) .. (115.15,205.77) .. controls (111.92,197.36) and (112.05,187.74) .. (115.52,179.39) .. controls (108.22,167.77) and (106.93,152.57) .. (112.13,139.47) .. controls (117.32,126.36) and (128.22,117.31) .. (140.75,115.7) .. controls (141.25,103.25) and (147.63,92.02) .. (157.43,86.35) .. controls (167.23,80.68) and (178.92,81.44) .. (187.99,88.34) .. controls (192.44,73.56) and (203.94,62.96) .. (217.53,61.12) .. controls (231.11,59.28) and (244.34,66.53) .. (251.49,79.73) .. controls (261.01,73.66) and (272.23,72.19) .. (282.63,75.66) .. controls (293.03,79.13) and (301.73,87.25) .. (306.77,98.18) .. controls (316.32,97.23) and (325.32,103.14) .. (329.31,112.98) .. controls (333.3,122.82) and (331.43,134.49) .. (324.63,142.2) .. controls (332.92,148.17) and (336.87,159.58) .. (334.42,170.5) .. controls (331.97,181.41) and (323.68,189.36) .. (313.87,190.18) ; \draw  [color={rgb, 255:red, 80; green, 227; blue, 194 }  ,draw opacity=1 ][dash pattern={on 1.69pt off 2.76pt}][line width=1.5]  (324.62,142.21) .. controls (320.71,139.39) and (316.13,138.02) .. (311.49,138.28)(306.77,98.18) .. controls (304.77,98.38) and (302.8,98.88) .. (300.92,99.66)(251.49,79.73) .. controls (252.82,82.16) and (253.9,84.76) .. (254.73,87.46)(187.99,88.34) .. controls (187.17,91.03) and (186.61,93.82) .. (186.31,96.64)(140.75,115.7) .. controls (140.22,128.95) and (146.42,141.31) .. (156.71,147.46)(115.52,179.39) .. controls (117.37,174.93) and (120.09,171.01) .. (123.48,167.94)(132.38,223.21) .. controls (132.15,221.38) and (132.08,219.52) .. (132.17,217.67)(176.3,238.22) .. controls (177.96,236.15) and (179.34,233.82) .. (180.41,231.31)(215.02,235.35) .. controls (215.93,233.45) and (216.63,231.43) .. (217.1,229.34)(259.47,229.09) .. controls (257.06,227.36) and (254.84,225.3) .. (252.87,222.97)(313.65,190.77) .. controls (313.47,188.66) and (313.15,186.57) .. (312.68,184.53) ;
    
    %Rounded Rect [id:dp4710145421837679] 
    \draw  [color={rgb, 255:red, 196; green, 158; blue, 126 }  ,draw opacity=1 ][dash pattern={on 5.63pt off 4.5pt}][line width=1.5]  (513.82,33.19) .. controls (530.18,33.2) and (543.44,46.47) .. (543.43,62.83) -- (543.32,278.58) .. controls (543.31,294.94) and (530.04,308.2) .. (513.67,308.19) -- (424.78,308.14) .. controls (408.42,308.13) and (395.16,294.86) .. (395.17,278.5) -- (395.28,62.76) .. controls (395.29,46.39) and (408.56,33.13) .. (424.93,33.14) -- cycle ;

    %Straight Lines [id:da03293914629554606] 


    %Straight Lines [id:da03293914629554606] 
    \draw [color=offpurple  ][line width=1.5]  [dash pattern={on 1.69pt off 2.76pt}]  (361.25,294.17) -- (361.25,202.17) ;
    \draw [shift={(361.25,199.17)}, rotate = 450] [color=offpurple  ][line width=1.5]    (14.21,-4.28) .. controls (9.04,-1.82) and (4.3,-0.39) .. (0,0) .. controls (4.3,0.39) and (9.04,1.82) .. (14.21,4.28)   ;




	% Text Node
	\draw (65,208) node [anchor=north west][inner sep=0.75pt]  [font=\footnotesize,color={rgb, 255:red, 74; green, 144; blue, 226 }  ,opacity=1 ]  {Input};
	% Text Node
	\draw (65,164) node [anchor=north west][inner sep=0.75pt]  [font=\scriptsize,color={rgb, 255:red, 74; green, 144; blue, 226 }  ,opacity=1 ]  {Hello};
	% Text Node
	\draw (454,131) node [anchor=north west][inner sep=0.75pt]  [font=\footnotesize,color=offwhite]  {Node 1};
	% Text Node
	\draw (564,167) node [anchor=north west][inner sep=0.75pt]  [font=\scriptsize,color={rgb, 255:red, 70; green, 155; blue, 36 }  ,opacity=1 ]  {HELLO};
	% Text Node
	\draw (564,208) node [anchor=north west][inner sep=0.75pt]  [font=\footnotesize,color={rgb, 255:red, 70; green, 155; blue, 36 }  ,opacity=1 ]  {Output};
	% Text Node
	\draw (141,164) node [anchor=north west][inner sep=0.75pt]  [font=\scriptsize,color=offyellow  ,opacity=1 ]  {File Split};
	% Text Node
	\draw (248,131) node [anchor=north west][inner sep=0.75pt]  [font=\scriptsize,color=offred  ,opacity=1 ]  {Split-1};
	% Text Node
	\draw (248,200) node [anchor=north west][inner sep=0.75pt]  [font=\scriptsize,color=offred  ,opacity=1 ]  {Split-2};
	% Text Node
	\draw (454,286) node [anchor=north west][inner sep=0.75pt]  [font=\footnotesize,color=offwhite]  {Node 2};
	% Text Node
	\draw (354,165) node [anchor=north west][inner sep=0.75pt]  [font=\scriptsize,color=offpurple]  {???};
    \draw (340,313) node [anchor=north west][inner sep=0.75pt]  [font=\scriptsize,color=offpurple]  {Mgmt Box};

    \draw (166,313) node [anchor=north west][inner sep=0.75pt]  [font=\footnotesize,color={rgb, 255:red, 80; green, 227; blue, 194 }  ,opacity=1 ] [align=left] {File System Box};

	
    % Text Node
    \draw (418,313) node [anchor=north west][inner sep=0.75pt]  [font=\footnotesize,color={rgb, 255:red, 196; green, 158; blue, 126 }  ,opacity=1 ] [align=left] {Processing Box};



\end{tikzpicture}

	\end{figure}


\end{frame}

%%%%%%%%%%%%%%%%%%%%%%%%%%%%%%%%%%%%%%%%%%%%%%%%%%%%%%
\begin{frame}[plain,c]
	\frametitle{Case Study Example 2}
	\begin{figure}
		\centering
		


\tikzset{every picture/.style={line width=0.75pt}} %set default line width to 0.75pt        

\begin{tikzpicture}[x=0.65pt,y=0.65pt,yscale=-1,xscale=1]
	%uncomment if require: \path (0,221); %set diagram left start at 0, and has height of 221
	
	
	%Shape: Rectangle [id:dp3221278753230701] 
	\draw [color=offred]  (2,60) -- (111,60) -- (111,90) -- (2,90) -- cycle ;
	%Shape: Rectangle [id:dp1271633566749788] 
	\draw [color=offred]  (2,140) -- (111,140) -- (111,170) -- (2,170) -- cycle ;
	%Curve Lines [id:da7576806433421945] 
	\draw  [line width=0.75,color=offwhite]  (112.42,59) .. controls (125,77.7) and (100.24,107.85) .. (128.62,109.05) ;

	%Curve Lines [id:da7014275949258858] 
	\draw [line width=0.75,color=offwhite]   (112,161.08) .. controls (125,137.56) and (97.32,110.2) .. (128.44,109.12) ;
	\draw [line width=0.75,color=offwhite] [shift={(130.42,109.08)}, rotate = 180] [color=offwhite  ][line width=0.75]    (10.93,-3.29) .. controls (6.95,-1.4) and (3.31,-0.3) .. (0,0) .. controls (3.31,0.3) and (6.95,1.4) .. (10.93,3.29)   ;
	%Rounded Rect [id:dp8577127496471914] 
	\draw   [color=offpurple] (131,103.47) .. controls (131,101.05) and (132.96,99.08) .. (135.38,99.08) -- (166.03,99.08) .. controls (168.45,99.08) and (170.42,101.05) .. (170.42,103.47) -- (170.42,116.62) .. controls (170.42,119.04) and (168.45,121) .. (166.03,121) -- (135.38,121) .. controls (132.96,121) and (131,119.04) .. (131,116.62) -- cycle ;
	%Rounded Rect [id:dp30434364847793793] 
	%%SPLIT 1
	\draw   [color={rgb, 255:red, 74; green, 144; blue, 226 }] 	(200,60) -- (250,60) -- (250,90) -- (200,90) -- cycle ;
	%Rounded Rect [id:dp4147495296848157] 
	%%SPLIT2
	
	\draw    [color={rgb, 255:red, 74; green, 144; blue, 226 }]  	(200,140) -- (250,140) -- (250,170) -- (200,170) -- cycle ;
	
	%%shuffle and sort
	\draw  [color={rgb, 255:red, 248; green, 231; blue, 28 }  ,draw opacity=1 ] (406,20) -- (460,20) -- (460,200) -- (406,200) -- cycle ;
	
	%Shape: Rectangle [id:dp08196853637328605] 
	\draw  [color={rgb, 255:red, 248; green, 231; blue, 28 }  ,draw opacity=1 ] (411,23) -- (455,23) -- (455,87) -- (411,87) -- cycle ;
	%Shape: Rectangle [id:dp20313277023673393] 
	\draw  [color={rgb, 255:red, 248; green, 231; blue, 28 }  ,draw opacity=1 ] (411,129) -- (455,129) -- (455,196) -- (411,196) -- cycle ;
	
	\draw (412,150) node [anchor=north west][inner sep=0.75pt]  [font=\tiny,color=orange  ,opacity=1 ] [align=left] {(The,\{1,1\}) \\(next,1)\\(very,1)};
	
	\draw (412,33) node [anchor=north west][inner sep=0.75pt]  [font=\tiny,color=orange  ,opacity=1 ] [align=left] {(back,1)\\(cat,1)\\(came,1)\\(day,1)};
	
	%Shape: Parallelogram [id:dp48792900018844987] 
	%%MAP
	\draw  [color=offyellow] (290,55.58) -- (323,55.58) -- (309,84) -- (277,84) -- cycle ;
	%Shape: Parallelogram [id:dp15766656313302363] 
	\draw [color=offyellow]  (290,140) -- (323,140) -- (309,166) -- (277,166) -- cycle ;
	
	%Straight Lines [id:da44147302793163634] 
	\draw  [line width=0.75,color=offwhite]  (251,70.79) -- (276,70.6) ;
	\draw  [line width=0.75,color=offwhite][shift={(280,70.58)}, rotate = 539.53] [color=offwhite  ][line width=0.75]    (10.93,-3.29) .. controls (6.95,-1.4) and (3.31,-0.3) .. (0,0) .. controls (3.31,0.3) and (6.95,1.4) .. (10.93,3.29)   ;
	
	
	%Straight Lines [id:da9443785238686285] 
	\draw  [line width=0.75,color=offwhite]  (251,151.79) -- (276,151.6) ;
	\draw [line width=0.75,color=offwhite] [shift={(278.42,151.58)}, rotate = 539.56] [color=offwhite  ][line width=0.75]    (10.93,-3.29) .. controls (6.95,-1.4) and (3.31,-0.3) .. (0,0) .. controls (3.31,0.3) and (6.95,1.4) .. (10.93,3.29)   ;
	%Shape: Rectangle [id:dp016273691107661636] 
	\draw [color={rgb, 255:red, 70; green, 155; blue, 36 }]  (342,47.58) -- (390,47.58) -- (390,106.58) -- (343,106.58) -- cycle ;
	%Straight Lines [id:da747872157663686] 
	\draw  [line width=0.75,color=offwhite]  (317,70) -- (338,70) ;
	\draw  [line width=0.75,color=offwhite] [shift={(340,70.58)}, rotate = 539.53] [color=offwhite  ][line width=0.75]    (10.93,-3.29) .. controls (6.95,-1.4) and (3.31,-0.3) .. (0,0) .. controls (3.31,0.3) and (6.95,1.4) .. (10.93,3.29)   ;
	%Straight Lines [id:da29503611504052674] 
	\draw  [line width=0.75,color=offwhite]  (317,156) -- (338,156) ;
\draw [line width=0.75,color=offwhite] [shift={(340.42,156.58)}, rotate = 539.53] [color=offwhite  ][line width=0.75]    (10.93,-3.29) .. controls (6.95,-1.4) and (3.31,-0.3) .. (0,0) .. controls (3.31,0.3) and (6.95,1.4) .. (10.93,3.29)   ;
	%Shape: Rectangle [id:dp4945355441436542] 
	%output
	\draw [color={rgb, 255:red, 70; green, 155; blue, 36 }]  (342,121) -- (390,121) -- (390,182.58) -- (343,182.58) -- cycle ;
	%Shape: Rectangle [id:dp7486542282421536] 
	%% outer box
	\draw   (188,20.58) -- (398.42,20.58) -- (398.42,110.58) -- (188,110.58) -- cycle ;
	%Shape: Rectangle [id:dp06285129437676396] 
	%%outerbox
	\draw   (188,110.58) -- (398.42,110.58) -- (398.42,200.58) -- (188,200.58) -- cycle ;
	%Rounded Rect [id:dp7025642442655686] 
	\draw   (414.42,98.18) .. controls (414.42,94.54) and (417.37,91.58) .. (421.02,91.58) -- (444.82,91.58) .. controls (448.46,91.58) and (451.42,94.54) .. (451.42,98.18) -- (451.42,117.98) .. controls (451.42,121.63) and (448.46,124.58) .. (444.82,124.58) -- (421.02,124.58) .. controls (417.37,124.58) and (414.42,121.63) .. (414.42,117.98) -- cycle ;
	%Shape: Rectangle [id:dp7889635565295128] 
	%% outer box
	\draw   (468.42,19.58) -- (670.42,19.58) -- (670.42,110.58) -- (468.42,110.58) -- cycle ;
	%Shape: Rectangle [id:dp8905165615601378] 
	%% outer box
	\draw   (468.42,110.58) -- (670.42,110.58) -- (670.42,200.58) -- (468.42,200.58) -- cycle ;
	%Shape: Rectangle [id:dp38010158785945036] 
	%input
	\draw [color={rgb, 255:red, 74; green, 144; blue, 226 } ]  (471,39.58) -- (523.42,39.58) -- (523.42,100.58) -- (471,100.58) -- cycle ;
	%Shape: Rectangle [id:dp7112194600890194] 
	%input
	\draw  [color={rgb, 255:red, 74; green, 144; blue, 226 } ] (471,127) -- (528.42,127) -- (528.42,187.58) -- (471,187.58) -- cycle ;
	
	%Shape: Parallelogram [id:dp8560234333811613] 
	\draw [color=offpink]  (555,53.58) -- (593,53.58) -- (580,82) -- (544,82) -- cycle ;
	%Shape: Parallelogram [id:dp03185670047818545] 
	\draw [color=offpink]  (555,132.58) -- (593,132.58) -- (580,161) -- (544,161) -- cycle ;
	
	
	%Shape: Rectangle [id:dp7083827174523002] 
	%output
	\draw [color={rgb, 255:red, 70; green, 155; blue, 36 }]  (610,36.58) -- (660,36.58) -- (660,96.58) -- (610,96.58) -- cycle ;
	%Shape: Rectangle [id:dp1906987565798669] 
	%output
	\draw [color={rgb, 255:red, 70; green, 155; blue, 36 }]  (610,130.58) -- (660,130.58) -- (660,180.58) -- (610,180.58) -- cycle ;
	
	%Straight Lines [id:da49186429007848953] 
	\draw  [line width=0.75,color=offwhite]  (523.21,69.79) -- (546.42,69.6) ;
	\draw [line width=0.75,color=offwhite] [shift={(548.42,69.58)}, rotate = 539.53] [color=offwhite  ][line width=0.75]    (10.93,-3.29) .. controls (6.95,-1.4) and (3.31,-0.3) .. (0,0) .. controls (3.31,0.3) and (6.95,1.4) .. (10.93,3.29)   ;
	%Straight Lines [id:da9429141358329266] 
	\draw  [line width=0.75,color=offwhite]  (583.21,67.79) -- (603.42,67.6) ;
	\draw [line width=0.75,color=offwhite] [shift={(605.42,67.58)}, rotate = 539.46] [color=offwhite  ][line width=0.75]    (10.93,-3.29) .. controls (6.95,-1.4) and (3.31,-0.3) .. (0,0) .. controls (3.31,0.3) and (6.95,1.4) .. (10.93,3.29)   ;
	%Straight Lines [id:da955760701625054] 
	\draw [line width=0.75,color=offwhite]   (586.21,146.79) -- (606.42,146.6) ;
	\draw [line width=0.75,color=offwhite] [shift={(608.42,146.58)}, rotate = 539.46] [color=offwhite  ][line width=0.75]    (10.93,-3.29) .. controls (6.95,-1.4) and (3.31,-0.3) .. (0,0) .. controls (3.31,0.3) and (6.95,1.4) .. (10.93,3.29)   ;
	%Straight Lines [id:da4689003855145568] 
	\draw [line width=0.75,color=offwhite]   (528.21,147.79) -- (546.42,147.6) ;
	\draw [line width=0.75,color=offwhite] [shift={(548.42,147.58)}, rotate = 539.4100000000001] [color=offwhite  ][line width=0.75]    (10.93,-3.29) .. controls (6.95,-1.4) and (3.31,-0.3) .. (0,0) .. controls (3.31,0.3) and (6.95,1.4) .. (10.93,3.29)   ;
	%Curve Lines [id:da21391955631880766] 
	\draw [line width=0.75,color=offwhite]   (454,108) .. controls (469.19,89.86) and (442.25,98.32) .. (466.28,70.86) ;
	\draw [line width=0.75,color=offwhite] [shift={(467.42,69.58)}, rotate = 491.88] [color=offwhite  ][line width=0.75]    (10.93,-3.29) .. controls (6.95,-1.4) and (3.31,-0.3) .. (0,0) .. controls (3.31,0.3) and (6.95,1.4) .. (10.93,3.29)   ;
	%Curve Lines [id:da1946353683290437] 
	\draw [line width=0.75,color=offwhite]   (454,108) .. controls (464.9,117.15) and (454.45,140.36) .. (463.94,148.53) ;
	\draw [line width=0.75,color=offwhite] [shift={(465.42,149.58)}, rotate = 210.26] [color=offwhite  ][line width=0.75]    (10.93,-3.29) .. controls (6.95,-1.4) and (3.31,-0.3) .. (0,0) .. controls (3.31,0.3) and (6.95,1.4) .. (10.93,3.29)   ;
	%Curve Lines [id:da6948926407425784] 
	\draw [line width=0.75,color=offwhite]   (398,68) .. controls (408,91.6) and (400.11,103.6) .. (413.61,110.72) ;
	\draw [line width=0.75,color=offwhite] [shift={(415.42,111.58)}, rotate = 203.63] [color=offwhite  ][line width=0.75]    (10.93,-3.29) .. controls (6.95,-1.4) and (3.31,-0.3) .. (0,0) .. controls (3.31,0.3) and (6.95,1.4) .. (10.93,3.29)   ;
	%Curve Lines [id:da5460745178214813] 
	\draw  [line width=0.75,color=offwhite]  (398.42,138.58) .. controls (413.22,124.74) and (399.6,114.44) .. (413.53,111.86) ;
	\draw  [line width=0.75,color=offwhite] [shift={(415.42,111.58)}, rotate = 533.29] [color=offwhite  ][line width=0.75]    (10.93,-3.29) .. controls (6.95,-1.4) and (3.31,-0.3) .. (0,0) .. controls (3.31,0.3) and (6.95,1.4) .. (10.93,3.29)   ;
	%Curve Lines [id:da3401941808176002] 
	\draw  [line width=0.75,color=offwhite]  (172,100) .. controls (180.25,88.81) and (159.28,65.54) .. (186.68,65.55) ;
	\draw [line width=0.75,color=offwhite] [shift={(188.42,65.58)}, rotate = 181.91] [color=offwhite  ][line width=0.75]    (10.93,-3.29) .. controls (6.95,-1.4) and (3.31,-0.3) .. (0,0) .. controls (3.31,0.3) and (6.95,1.4) .. (10.93,3.29)   ;
	%Curve Lines [id:da46994249585601144] 
	\draw  [line width=0.75,color=offwhite]  (172,100) .. controls (180.25,109.39) and (159.28,133.59) .. (185.74,135.49) ;
	\draw [line width=0.75,color=offwhite] [shift={(187.42,135.58)}, rotate = 181.97] [color=offwhite ][line width=0.75]    (10.93,-3.29) .. controls (6.95,-1.4) and (3.31,-0.3) .. (0,0) .. controls (3.31,0.3) and (6.95,1.4) .. (10.93,3.29)   ;
	
	% Text Node
	\draw (4,67) node [anchor=north west][inner sep=0.75pt]  [font=\scriptsize] [align=left,color=offred] {The cat came back};
	% Text Node
	\draw (41,90) node [anchor=north west][inner sep=0.75pt]  [font=\scriptsize] [align=left,color=offred] {split-1};
	% Text Node
	\draw (4,149) node [anchor=north west][inner sep=0.75pt]  [font=\scriptsize] [align=left,color=offred] {The very next day};
	% Text Node
	\draw (41,174) node [anchor=north west][inner sep=0.75pt]  [font=\scriptsize] [align=left,color=offred] {split-2};
	
	
	% Text Node
	\draw (133,102.08) node [anchor=north west][inner sep=0.75pt]   [align=left,color=offpurple] {{\scriptsize S \% n}};
	
	% Text Node
	\draw (205,66) node [anchor=north west][inner sep=0.75pt]  [font=\scriptsize,color=offred] [align=left] {split-1};
	% Text Node
	\draw (205,146) node [anchor=north west][inner sep=0.75pt]  [font=\scriptsize,color=offred] [align=left] {split-2};
	
	% Text Node
	\draw (285,65) node [anchor=north west][inner sep=0.75pt]  [font=\scriptsize] [align=left,color=offyellow] {map};
	% Text Node
	\draw (285,146) node [anchor=north west][inner sep=0.75pt]  [font=\scriptsize] [align=left,color=offyellow] {map};
	
	% Text Node
	\draw (270,98) node [anchor=north west][inner sep=0.75pt]  [font=\scriptsize] [align=left,color=offgreen2] {Node 1};
	% Text Node
	\draw (270,112) node [anchor=north west][inner sep=0.75pt]  [font=\scriptsize] [align=left,color=offgreen2] {Node 2};
	
	% Text Node
	\draw (342,50.58) node [anchor=north west][inner sep=0.75pt]  [font=\scriptsize] [align=left,color=offyellow] {(The,1) \\(cat,1)\\(came,1) \\(back,1)};
	% Text Node
	\draw (341,124) node [anchor=north west][inner sep=0.75pt]  [font=\scriptsize] [align=left,color=offyellow] {(The,1) \\(very,1)\\(next,1) \\(day,1)};
	
	% Text Node
	\draw (205,36) node [anchor=north west][inner sep=0.75pt]  [font=\scriptsize] [align=left,color={rgb, 255:red, 74; green, 144; blue, 226 } ] {input};
	% Text Node
	\draw (205,185.58) node [anchor=north west][inner sep=0.75pt]  [font=\scriptsize] [align=left,color={rgb, 255:red, 74; green, 144; blue, 226 } ] {input};
	
	% Text Node
	\draw (340,34) node [anchor=north west][inner sep=0.75pt]  [font=\scriptsize] [align=left,color={rgb, 255:red, 70; green, 155; blue, 36 }] {output};
	% Text Node
	\draw (340,186) node [anchor=north west][inner sep=0.75pt]  [font=\scriptsize] [align=left,color={rgb, 255:red, 70; green, 155; blue, 36 }] {output};

	% Text Node
	\draw (416,95) node [anchor=north west][inner sep=0.75pt]  [font=\tiny] [align=left] {Shuffle \\\& Soft};
	% Text Node
	
	
	\draw (470,42.58) node [anchor=north west][inner sep=0.75pt]  [font=\scriptsize] [align=left,color=offyellow] {(back,1)\\(cat,1)\\(came,1)\\(day,1)};
	% Text Node
	\draw (470,130) node [anchor=north west][inner sep=0.75pt]  [font=\scriptsize] [align=left,color=offyellow ] {(The,\{1,1\})\\(next,1)\\(very,1)};
	
	
	
	% Text Node
	\draw (551,64) node [anchor=north west][inner sep=0.75pt]  [font=\scriptsize] [align=left,color=offpink] {count};
	% Text Node
	\draw (551,142) node [anchor=north west][inner sep=0.75pt]  [font=\scriptsize] [align=left,color=offpink] {count};
	
	
	% Text Node
	\draw (610,39.58) node [anchor=north west][inner sep=0.75pt]  [font=\scriptsize] [align=left,color=offpink] {(back,1) \\(cat,1)\\(came,1)\\(day,1)};
	% Text Node
	\draw (610,132) node [anchor=north west][inner sep=0.75pt]  [font=\scriptsize] [align=left,color=offpink] {(The,2)\\(next,1)\\(very,1)};
	
	% Text Node
	\draw (286,184.58) node [anchor=north west][inner sep=0.75pt]  [font=\scriptsize] [align=left,color=offyellow] {fn};
	% Text Node
	\draw (286,34.58) node [anchor=north west][inner sep=0.75pt]  [font=\scriptsize] [align=left,color=offyellow] {fn};
	
	% Text Node
	\draw (552.22,112.58) node [anchor=north west][inner sep=0.75pt]  [font=\scriptsize] [align=left,color=offgreen2] {Node 2};
	% Text Node
	\draw (552.22,97.58) node [anchor=north west][inner sep=0.75pt]  [font=\scriptsize] [align=left,color=offgreen2] {Node 1};
	
	% Text Node
	\draw (610,24) node [anchor=north west][inner sep=0.75pt]  [font=\scriptsize] [align=left,color={rgb, 255:red, 70; green, 155; blue, 36 }] {output};
	% Text Node
	\draw (610,182) node [anchor=north west][inner sep=0.75pt]  [font=\scriptsize] [align=left,color={rgb, 255:red, 70; green, 155; blue, 36 }] {output};
	
	% Text Node
	\draw (560,186.33) node [anchor=north west][inner sep=0.75pt]  [font=\scriptsize] [align=left,color=offpink] {fn};
	% Text Node
	\draw (560,26.58) node [anchor=north west][inner sep=0.75pt]  [font=\scriptsize] [align=left,color=offpink] {fn};
	
	% Text Node
	\draw (470,27) node [anchor=north west][inner sep=0.75pt]  [font=\scriptsize] [align=left,color={rgb, 255:red, 74; green, 144; blue, 226 } ] {input};
	% Text Node
	\draw (470,188) node [anchor=north west][inner sep=0.75pt]  [font=\scriptsize] [align=left,color={rgb, 255:red, 74; green, 144; blue, 226 } ] {input};
	
	
	\draw (520.22,208) node [anchor=north west][inner sep=0.75pt]  [font=\footnotesize] [align=left] {Reduce side};
	\draw (260,208) node [anchor=north west][inner sep=0.75pt]  [font=\footnotesize] [align=left] {Map side};
	
\end{tikzpicture}

	\end{figure}
	
\end{frame}

%%%%%%%%%%%%%%%%%%%%%%%%%%%%%%%%%%%%%%%%%%%%%%%%%%%%%%
\begin{frame}

	\begin{figure}
		\includegraphics[height=.925\textheight]{./Figures/chapter-02/Map-Reduce.png}
				\caption{Map Reduce Stages } \label{fig:MRSteps}
	\end{figure}			
\end{frame}
%%%%%%%%%%%%%%%%%%%%%%%%%%%%%%%%%%%%%%%%%%%%%%%%%%%%%%
\begin{frame}
	
	\begin{figure}
		\includegraphics[height=.925\textheight]{./Figures/chapter-02/Map-Reduce_2.png}
		\caption{Map Reduce Stages } \label{fig:MRSteps2}
	\end{figure}			
\end{frame}
%%%%%%%%%%%%%%%%%%%%%%%%%%%%%%%%%%%%%%%%%%%%%%%%%%%%%%
\begin{frame}[c]{ }
	\frametitle{Map Reduce (word count) Deep Dive }
	
	The Map-Reduce consists of three "main" parts
	
	\begin{itemize}  [<+->]
		\item [--] The Driver.
		\item [--] The Mapper.
		\item [--] The Reducer.
		
	\end{itemize}
\end{frame}
%%%%%%%%%%%%%%%%%%%%%%%%%%%%%%%%%%%%%%%%%%%%%%%%%%%%%%
\begin{frame}[c]{ }
	\frametitle{ Hadoop Map Reduce API}
	\centering     
	
	\textcolor{offgreen}{ \large Hadoop Map Reduce API Deep Dive}
\end{frame}
%%%%%%%%%%%%%%%%%%%%%%%%%%%%%%%%%%%%%%%%%%%%%%%%%%%%%%
\begin{frame}[c]{ }
	\frametitle{The Driver }

	\begin{itemize}  [<+->]
		\item [--] The code that runs on the client machine configures the job details by creating an object from the \code{Job}  class, which implements the \code{JobContext} interface.
		\item [--] It submits the job to the cluster.		
		\item [--] It parses job arguments to identify job parameters, for example, input/output directories.. 
		
	\end{itemize}

\end{frame}
%%%%%%%%%%%%%%%%%%%%%%%%%%%%%%%%%%%%%%%%%%%%%%%%%%%%%%
\begin{frame}[c]{ }
	\frametitle{The Driver:  Job Configuration}
	

The \code{Job}  object allows you to set configuration for your \code{Map-Reduce} job:
			\begin{itemize}  [<+->]
				\item [--] You can configure the \code{Mapper} \& the \code{Reducer} classes.
				\item [--] Set the \code{Mapper} input/output key \& value data types.
				\item [--] Set the \code{Reducer} input/output key \& value data types.
	
		
			\end{itemize}		

	
\end{frame}
%%%%%%%%%%%%%%%%%%%%%%%%%%%%%%%%%%%%%%%%%%%%%%%%%%%%%%
\begin{frame}[c]{ }
	\frametitle{The Driver:  Job Configuration}
	
	\begin{itemize}  [<+->]
		\item [--] We can configure file input directory and output.
		\item [--] We configure the output path using \code{FileOutputFormat.setOutputPath()} to specify the reducers' directory to write the output data.
	\end{itemize}		
	
\end{frame}
%%%%%%%%%%%%%%%%%%%%%%%%%%%%%%%%%%%%%%%%%%%%%%%%%%%%%%
\begin{frame}[c]{ }
	\frametitle{The Driver:  Job Configuration}
	
	\begin{itemize}  [<+->]
		\item [--] We configure the input path using \code{FileInputFormat.setInputPaths()}, and by default, it will read all the files in the specified directories and send them to the mappers.
		
		\item [--] We can use \code{Hadoop glob patterns} to read directory patterns, for example, \textit{/warehouse/public/sales*}.
		\item [--] We can call \code{FileInputFormat.addInputPath()} to multiple times by specifying a single file or directory. 
		
	\end{itemize}		
	
	\footnotetext[1]{For more details, please read HTDG. Ch.3 File patterns and PathFilter sections.	} 
\end{frame}
%%%%%%%%%%%%%%%%%%%%%%%%%%%%%%%%%%%%%%%%%%%%%%%%%%%%%%
\begin{frame}[c]{ }
	\frametitle{ Hadoop Map Reduce API}
	\centering     
	
	\textcolor{offgreen}{ \large Please read HTDG. Ch.3 The Java Interface}
\end{frame}
%%%%%%%%%%%%%%%%%%%%%%%%%%%%%%%%%%%%%%%%%%%%%%%%%%%%%%


\begin{frame}[c]{ }
	\frametitle{The Driver:  Job Configuration }
	
		
		\begin{itemize}  [<+->]

			\item [--] You could set driver configurations globally using Hadoop configurations.
			\item [--] Any options not specified in the job configuration will use the Hadoop default values.
			\item [--] We use the \code{Job} object to specify the job name and check its state..
		
			
	\end{itemize}
	
\end{frame}
%%%%%%%%%%%%%%%%%%%%%%%%%%%%%%%%%%%%%%%%%%%%%%%%%%%%%%
\begin{frame}[c]{ }
	\frametitle{The Driver:  Job Configuration }
	

	\begin{itemize}  [<+->]
		
	\item [--] It is optional to set the mapper and reducer classes.
	\item [--] Hadoop uses its default \code{IdentityMapper} and \code{IdentityReducer}.		
		
	\end{itemize}
	
\end{frame}
%%%%%%%%%%%%%%%%%%%%%%%%%%%%%%%%%%%%%%%%%%%%%%%%%%%%%%
\begin{frame}[c]{ }
	\frametitle{The Driver:  Job Configuration }
	
	
	Lunch a Map-Reduce job:
	\begin{itemize}  [<+->]
		
		\item [--] The \code{waitForCompletion()} method in the \code{Job} class launches the job and polls for progress. In addition, it writes the logs and summarizing the Map-Reduce job progress and changes.

	\item [--] When the job completes successfully, the job counters are displayed. Otherwise, the error that caused the job to fail is logged to the console.
		
	\end{itemize}
	
\end{frame}
%%%%%%%%%%%%%%%%%%%%%%%%%%%%%%%%%%%%%%%%%%%%%%%%%%%%%%
\begin{frame}[c]{ }
	\frametitle{InputFormat}
	
	\begin{itemize}  [<+->]
		\item [--] TheThe driver defines the \code{InputFormat}; then the \code{InputFormat} creates a \code{RecordReader"} object that parses the input data into key/value pairs passed to the mapper.
		\item [--] For example: \code{TextInputFormat}:
		\begin{itemize}  [<+->]
			
			\item It is the default.
			\item It creates \code{LineRecordReader} objects.
			\item Key: is the line offest in the file.
			\item Value: is the line which terminated by "\textbackslash n".
		\end{itemize}		
		
		
	\end{itemize}		
	
\end{frame}
%%%%%%%%%%%%%%%%%%%%%%%%%%%%%%%%%%%%%%%%%%%%%%%%%%%%%%
\begin{frame}[c]{ }
	\frametitle{Keys and Values}
	
	\begin{itemize}  [<+->]
		\item [--] Keys and Values in Hadoop are java \code{Objects} not \code{Java primitives types}.
		\item [--] Values are objects which implement \code{Writable}.
		\item [--] Keys are objects which implement \code{WritableComparable}.

		
	\end{itemize}		
	
\end{frame}
%%%%%%%%%%%%%%%%%%%%%%%%%%%%%%%%%%%%%%%%%%%%%%%%%%%%%%
\begin{frame}[c]{ }
	\frametitle{What is Writable?}
	
	\begin{itemize}  [<+->]
		\item [--] \code{Writable} is an interface in Hadoop.
		\item [--] \code{Writables} are used for data type "serialization" in Hadoop to translate/serialize "primitive java data types" to "Hadoop data types", Ex: int to IntWritable and String to Text.
		\item [--] Hadoop uses the \code{Writable} interface for data transfer in the cluster and network.
		
		
	\end{itemize}		
	
\end{frame}
%%%%%%%%%%%%%%%%%%%%%%%%%%%%%%%%%%%%%%%%%%%%%%%%%%%%%%
\begin{frame}[c]{ }
	\frametitle{What is WritableComparable?}
	
	\begin{itemize}  [<+->]
		\item [--] A \code{WritableComparable} is a \code{Writable} which is also \code{Comparable}.
		\item [--] We can compare two \code{WritableComparables} against each other to determine their \textbf{\underline{\textit{order}}}, for example, we could need to compare the order of two Text "Apple vs. Cat or numbers ordering" to understand the ordering mechanism. 
		\item [--] Obviously, the reason we have Keys to be \code{WritableComparable} is that they are passed to the reducer in \underline{\textit{\textbf{sorted order}}}.
		\item [--] Note: All Hadoop implemented types are both \code{Writable} and \code{WritableComparable}.
		
		
	\end{itemize}		
	
\end{frame}
%%%%%%%%%%%%%%%%%%%%%%%%%%%%%%%%%%%%%%%%%%%%%%%%%%%%%%
\begin{frame}[c]{ }
	\frametitle{Map Reduce (word count) Deep Dive }
	
	The Map-Reduce example consists of three main parts
	
	\begin{itemize}  [<+->]
		\item [--] \sout{The Driver}.
		\item [--] The Mapper.

		
	\end{itemize}
\end{frame}
%%%%%%%%%%%%%%%%%%%%%%%%%%%%%%%%%%%%%%%%%%%%%%%%%%%%%%
\begin{frame}[c]{ }
	\frametitle{The Mapper}
		
	\begin{itemize}  [<+->]
		
		\item [--] The mapper class deals with a single input split.
		
		\item [--] All mapper classes must extend the \code{Mapper} base class.

		\item [--] All mapper must specify the key and values for input and output.		
		
		\item [--] All mappers must override the \code{map} method and pass the key, value, and \code{Context}.
		
		\item [--]  The \code{Context} is used to write intermediate data and all information about the job's configurations.
		
	\end{itemize}
	
\end{frame}
%%%%%%%%%%%%%%%%%%%%%%%%%%%%%%%%%%%%%%%%%%%%%%%%%%%%%%
\begin{frame}[c]{ }
	\frametitle{Map Reduce (word count) Deep Dive }
	
	The Map-Reduce example consists of three main parts
	
	\begin{itemize}  [<+->]
		\item [--] \sout{The Driver}.
		\item [--] \sout{The Mapper}.
		\item [--] The Reducer.
		
	\end{itemize}
\end{frame}
%%%%%%%%%%%%%%%%%%%%%%%%%%%%%%%%%%%%%%%%%%%%%%%%%%%%%%
\begin{frame}[c]{ }
	\frametitle{The Reducer}
	
	\begin{itemize}  [<+->]
		
		\item [--] The Reducer receives a Key and an Iterable collection of Writable objects. It also receives a Context object.
		
		\item [--] All reducers classes must extend the  \code{Reducer} base class.
		
		\item [--] All mapper must specify the key and values for intermediate input and final (or intermediate) output.		
		
		\item [--] All reducers must override the "reduce" method and pass the key, \code{Iterable} and "Context".

	\end{itemize}
	
\end{frame}
%%%%%%%%%%%%%%%%%%%%%%%%%%%%%%%%%%%%%%%%%%%%%%%%%%%%%%
\begin{frame}[c]{ }
	\frametitle{ Hadoop Map Reduce API}
	\centering     
	
	\textcolor{offgreen}{ \large Map Reduce Demo}
\end{frame}
