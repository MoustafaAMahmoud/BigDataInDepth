%%%%%%%%%%%%%%%%%%%%%%%%%%%%%%%%%%%%%%%%%%%%%%%%%%%%%%
%\VideoClassification[column=2, colour=blue]
%%%%%%%%%%%%%%%%%%%%%%%%%%%%%%%%%%%%%%%%%%%%%%%%%%%%%%
% converting from logical to physical model
\midTitle{Partitioning vs Bucketing}
%%%%%%%%%%%%%%%%%%%%%%%%%%%%%%%%%%%%%%%%%%%%%%%%%%%%%%
\begin{frame}
	\frametitle{Data Categorization}
	\begin{itemize}[<+->]
		\item The database consists of some schemas.
		\item Each schema contains tables.
		\item Each table contains partitions.
		\item Each partition contains buckets (or files)
	\end{itemize}
	
\end{frame}
%%%%%%%%%%%%%%%%%%%%%%%%%%%%%%%%%%%%%%%%%%%%%%%%%%%%%%
%%%%%%%%%%%%%%%%%%%%%%%%%%%%%%%%%%%%%%%%%%%%%%%%%%%%%%
\begin{frame}
	\frametitle{Data Categorization}
		\begin{tikzpicture}[every label/.append style={font=\tiny},regentonne/.style={cylinder,aspect=.7,draw,shape border rotate=90}]

		
		%Database
		\filldraw[draw=blue,thick,rounded corners,fill=white] (-2,0) rectangle (-0.5,6);
  	    \node[font=\small] at (-1.25,5.6) {Database};

		%Tables

		\filldraw[draw=Maroon,thick,rounded corners,fill=white] (-0.25,0) rectangle (1.5,6);
     	\node[font=\small ] at (0.5,5.6) {Schema};
		
		%Partitions
		\filldraw[draw=OliveGreen,thick,rounded corners,fill=white] (1.75,-1) rectangle (7.55,6);
     	\node[font=\small ] at (4.5,5.6) {Table};
   	    
   	    
		%\draw[step=.5cm,gray,very thin] (-2,-1) grid (7.5,6);   	    
   	    %%%%%%%%%%%%%%%%%%%%%%%%%%%%%%%%%%%%%%%%%%%%%
		%schemas
		\node[database,label=below:schema-1, database middle segment={draw=black,fill=aliceblue}, database bottom segment={draw=black, fill=aliceblue}, database top segment={draw=black,fill=aliceblue} ] (s1) at (-1.3,5) {};
		
		\node[database,label=below:schema-2, database middle segment={draw=black,fill=aliceblue}, database bottom segment={draw=black, fill=aliceblue}, database top segment={draw=black,fill=aliceblue}] (s2) at (-1.3,4) {};
		
		\node[database,label=below:schema-3, database middle segment={draw=black,fill=aliceblue}, database bottom segment={draw=black, fill=aliceblue}, database top segment={draw=black,fill=aliceblue}] (s3) at (-1.3,3) {};		
		
		\node[database,label=below:schema-4, database middle segment={draw=black,fill=aliceblue}, database bottom segment={draw=black, fill=aliceblue}, database top segment={draw=black,fill=aliceblue}] (s4) at (-1.3,2) {};		
		
		\node[database,label=below:schema-5, database middle segment={draw=black,fill=aliceblue}, database bottom segment={draw=black, fill=aliceblue}, database top segment={draw=black,fill=aliceblue}] (s5) at (-1.3,1) {};						

   	    %%%%%%%%%%%%%%%%%%%%%%%%%%%%%%%%%%%%%%%%%%%%%
		%tables
		\node  (t1) at (.65,5) {\huge \faTable};
        \node [below=.01 cm of t1]  {\scriptsize table-1};

		\node  (t2) at (.65,3.6) {\huge \faTable};
		\node [below=.01 cm of t2]  {\scriptsize table-2};	
		
		\node  (t3) at (.65,2) {\huge \faTable};
		\node [below=.01 cm of t3]  {\scriptsize table-3};

   	    %%%%%%%%%%%%%%%%%%%%%%%%%%%%%%%%%%%%%%%%%%%%%
		%partitions & buckets
		%\draw[rectangle, rounded corners, minimum height=1cm,text centered, draw=black, fill=red!30] (2,4) grid (3,5);   	    
		\tikzstyle{startstop} = [rectangle, rounded corners, minimum width=1cm, minimum height=.5cm,text centered, draw=black, fill=red!30]
		
		\node (partition) [startstop] at (2.8,4.6) {\tiny Partition Only(s)};
		\node (both) [startstop] at (4.7,2) {\tiny Both};
		\node (bucketing) [startstop] at (6.6,4.6) {\tiny Bucketing Only};
		
		%%%%%%%%%%%%%%%%%%%%%%%%%%%%%%%%%%%%%%%%%%%%%
		\tikzstyle{partition} = [rectangle, minimum width=1cm, minimum height=.5cm,text centered, draw=black, fill=green!30]
		\node (part-dtl-1) [partition] at (2.6,3.9) {\tiny partition-1};		
		\node (part-dtl-2) [partition] at (2.6,3.3) {\tiny partition-2};		
		\node (part-dtl-3) [partition] at (2.6,2.7) {\tiny partition-3};		
		
		%%%%%%%%%%%%%%%%%%%%%%%%%%%%%%%%%%%%%%%%%%%%%
		\tikzstyle{bucket} = [rectangle, minimum width=1cm, minimum height=.2cm,text centered, draw=black, fill=blue!60,text=white]
		\node (buck-dtl-1) [bucket] at (6.8,3.9) {\tiny bucket-1};		
		\node (buck-dtl-2) [bucket] at (6.8,3.3)   {\tiny bucket-2};		
		\node (buck-dtl-3) [bucket] at (6.8,2.7) {\tiny bucket-3};	
		%%%%%%%%%%%%%%%%%%%%%%%%%%%%%%%%%%%%%%%%%%%%%		
		\node (part-buck-1) [partition] at (4.7,1.1) {\tiny partition-1};
		\node (part-buck-1-1) [bucket] at (6.7,1.6) {\tiny bucket-1-1};
		\node (part-buck-1-2) [bucket] at (6.7,1.1) {\tiny bucket-1-2};
		\node (part-buck-1-3) [bucket] at (6.7,.6) {\tiny bucket-1-3};
		\draw [->] (part-buck-1) -- (part-buck-1-1);
		\draw [->] (part-buck-1) -- (part-buck-1-2);
		\draw [->] (part-buck-1) -- (part-buck-1-3);
		
		\node (part-buck-2) [partition] at (4.7,0) {\tiny partition-2};
		\node (part-buck-2-1) [bucket] at (6.7,0) {\tiny bucket-2-1};
		\node (part-buck-2-2) [bucket] at (6.7,-.5) {\tiny bucket-2-2};
		\draw [->] (part-buck-2) -- (part-buck-2-1);
		\draw [->] (part-buck-2) -- (part-buck-2-2);
		
		
	\end{tikzpicture}

\end{frame}
%%%%%%%%%%%%%%%%%%%%%%%%%%%%%%%%%%%%%%%%%%%%%%%%%%%%%%
\begin{frame}
	\frametitle{Partitioning vs Bucketing Example}
\begin{table}
	\begin{tabular}{|c|c|c|c|}
		\hline
		cust_id & name & trns_date \\
		\hline
		12345 & Ali &  2020-10-10 \\
		12345 & Ali &  2020-10-10\\
		\hline 
		8765 & Amr &  2020-10-11 \\
		45365 & Omar &  2020-10-11 \\
		86896 & Zid &  2020-10-11 \\
		\hline
	\end{tabular}
\quad

	\caption{Table data example}\label{eval_table}
\end{table}

\begin{table}
	\begin{tabular}{|c|c|c|c|}
		\hline
		partition & cust_id  \\
		\hline
		2020-10-10 & 12345 \\
				   & 12345\\
		\hline 
		2020-10-11 &  8765\\
				   & 45365  \\
		           & 86896 \\
		\hline
	\end{tabular}
	\quad
		\begin{tabular}{|c|c|c|c|}
		\hline
		partition & bucket & cust_id(s)  \\
		\hline
		2020-10-10 & 1 & [12345] \\
		\hline 
		2020-10-11 & 1 & [8765,45365]\\
				   & 2 & [86896] \\
		\hline
	\end{tabular}
	\caption{Table with only partition vs partition with bucketing example}\label{eval_table}
\end{table}

\end{frame}


%%%%%%%%%%%%%%%%%%%%%%%%%%%%%%%%%%%%%%%%%%%%%%%%%%%%%%
\begin{frame}
	\frametitle{Partitioning vs Bucketing Example}
	
	\begin{itemize}[<+->]
		\item Example of table creation in \textbf{hive}
		\item Bucketing also named as \textit{clustering }
		\item /user/hive/warehouse/cust_trans_hist/trns_date=2020-10-10/
	\end{itemize}
	
			\lstinputlisting[language=sql,caption={Example how to create partition with bucketing in hive}]{Ch01-Introduction-data-management/12-Data-Model/07-partitioning-vs-bucketing/table_creation.sql}
\end{frame}


%%%%%%%%%%%%%%%%%%%%%%%%%%%%%%%%%%%%%%%%%%%%%%%%%%%%%%
\begin{frame}
	\frametitle{Partitioning Pros vs Cons}
	
Partition: Split huge amount of data into parts(directories for example) based on specified table column (s).

	\begin{itemize}[<+->]
		\item Pros
		\begin{itemize}[<+->]
			\item Distribute execution load horizontally.
			\item Number of partitions are based on the number of unique partition column.
			\item Faster execution of queries in case filter based on partition value instead of scan the entire table.
		\end{itemize}
	
		\item Cons
			\begin{itemize}[<+->]
			\item Limitation for number of partition.
			\item It can lead to performance issue in case too many small partitions.
		\end{itemize}
			
	\end{itemize}
\vspace{1.3cm}
Reference: https://stackoverflow.com/a/33645660/2516356	
\end{frame}


%%%%%%%%%%%%%%%%%%%%%%%%%%%%%%%%%%%%%%%%%%%%%%%%%%%%%%
\begin{frame}
	\frametitle{Bucketing Pros vs Cons}
	
Bucketing split data into equal parts based on column value. 
	\begin{itemize}[<+->]
		\item Pros
		\begin{itemize}[<+->]
			\item Map side join will be faster because all partitions are equal.
			\item Faster query response like partitioning
		\end{itemize}
		
		\item Cons
		\begin{itemize}[<+->]
			\item Number of buckets needs to be defined manual by the developers.
		\end{itemize}
		
	\end{itemize}
\vspace{3.6cm}
	Reference: https://stackoverflow.com/a/33645660/2516356
\end{frame}

