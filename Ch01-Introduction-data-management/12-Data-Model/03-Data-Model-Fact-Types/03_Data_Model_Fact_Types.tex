%! Author = Mostafa Alaa
%! Date = 4/7/2020
%%%%%%%%%%%%%%%%%%%%%%%%%%%%%%%%%%%%%%%%%%%%%%%%%%%%%
\VideoClassification[column=1, colour=red]

% %TODO:
% Review Text


\subsubsection{Fact Table}

\begin{frame}
	\frametitle{Fact Table Recap}
	What is the fact table?
    	\begin{itemize}
    		\item It is the foundation of the data warehouse.
            \item It consists of facts and measurements of a particular business aspect and processes ex: daily revenue for a product.
    		\item It is the target of queries in most of DWH analysis and reports.
            \item It contains measurements/facts and foreign keys to \textit{dimensions table}. 
            \item It located at the center of the schema and surrounded by dimension tables.
    	\end{itemize}
\end{frame}
%--------------------------------------------------
\begin{frame}
	\frametitle{Fact Table Recap}

          \begin{fquote}[Ralph Kimball][kimballgroup.com]
        ``There is no point in hoisting fact tables up the flagpole unless they have been chosen to reflect urgent business priorities``
          \end{fquote}

\end{frame}
%--------------------------------------------------
\midTitle{How to design a fact table?}
\begin{frame}
	\frametitle{How to design a fact table?}
    
    	\begin{itemize}
    		\item Choose the business process.
            \item Identify the grain.
            \item Identify the dimensions.
            \item Identify the fact.
    	\end{itemize}
\end{frame}

%--------------------------------------------------
\begin{frame}
	\frametitle{Fact Granularity}
    	\begin{itemize}
                \item The grain is the definition of what a single row in the fact table will represent or contains.
                \item The grain describes the physical event which needs to be measured.
                \item Grain controls the dimensions which are available in fact.
                \item Grain represents the level of information we need to represent. It is not always time; it could be the physical business measurement level.
                \item Design from the lowest possible grain.
        	\end{itemize}
\end{frame}
%--------------------------------------------------
\subsubsection{Fact Table Types}
\begin{frame}
	\frametitle{Fact Types}
	There are three types of fact tables:
	\begin{itemize}
		\item Transaction.
		\item Periodic.
		\item Accumulated Snapshot.
	\end{itemize}
\end{frame}

%--------------------------------------------------
\begin{frame}
	\frametitle{Fact Types: Transaction Fact Table}
	\begin{itemize}
        \item Fact grain set at a single transaction
        \item It has one row per transaction.
        \item For each transaction, we add a new single record.
        \item The transaction fact table is known to grow very fast as the number of transactions increases.
        
	\end{itemize}
    
\end{frame}
%--------------------------------------------------
\begin{frame}
	\frametitle{Fact Types:Transaction Example}
 \begin{table}
    		\begin{tabular}{| l | l | l|l|l|}
           			\hline
                      customer\_id & trns\_date & trns\_time & call\_type & duration\\
                  	  \hline
                      1234 & 2020-01-01 & 12:22:45.9 & Incoming & 29\\
                      1234 & 2020-01-01 & 12:22:45.9 & Incoming & 3134\\
                      1234 & 2020-01-02 & 15:22:45.0 & Outgoing & 890 \\
                      1234 & 2020-01-02 & 15:22:45.0 & International& 119 \\
                      1234 & 2020-01-03 & 23:22:45.0 & Incoming & 145 \\
                      1234 & 2020-01-03 & 23:22:45.0 & Outgoing & 124 \\
                      1234 & 2020-01-03 & 23:22:45.0 & Outgoing & 1200\\
    			\hline
    		\end{tabular}
    	\caption{Transaction fact example of telecom calls data.}
    \end{table}
\end{frame}
%--------------------------------------------------
\begin{frame}
	\frametitle{Fact Types: Periodic Fact Table}
	\begin{itemize}
        \item A periodic fact table contains one row for a \textit{group} of transactions over a period. 
		\item It must be from lower granularity to higher granularity hourly, daily, monthly, and quertrly, then yearly.
	\end{itemize}
\end{frame}
%--------------------------------------------------
\begin{frame}
	\frametitle{Fact Types: Periodic Fact Table Example}
     \begin{table}
        		\begin{tabular}{| l | l | l|l|l|}
             	\hline    
                 cust\_id & month\_id  & incoming     & outgoing & international\\
                 		\hline
                 1234       & 20200131        & 3308           & 2124      & 119 \\
    			\hline
    		\end{tabular}            
    	\caption{Periodic fact example of telecom calls data.}
    \end{table}

\end{frame}
%--------------------------------------------------


%--------------------------------------------------
\begin{frame}
	\frametitle{Fact Types: Accumulated Snapshot Fact Table}
	\begin{itemize}
		\item An accumulating fact table stores one row for the entire process.
		\item It does not accumulate time it accumulates business process.
		\item A row in an accumulating snapshot fact table summarizes the measurement events occurring at predictable steps between the beginning and the end of a process
		\item Accumulating Fact tables are used to show the activity of progress through a well-defined process and are most often used to research the time between milestones.
		\item These fact tables are updated as the business process unfolds, and each milestone is completed.
        
	\end{itemize}
\end{frame}
%--------------------------------------------------
\begin{frame}
    \frametitle{Fact Types: Accumulated Snapshot Fact Table Example}
    \begin{itemize}
        \item Accumulated Snapshot use cases are engaged when we need to report the entire process life-cycle.Fact Types: Accumulated Snapshot Use Cases.        
        \item It also uses to measure the process performance life-cycle.
            \begin{itemize}
                \item Order life-cycle.
                \item Insurance processing.
                \item Hiring process.
            \end{itemize}       
    \end{itemize}
\end{frame}
%--------------------------------------------------
\begin{frame}
    \frametitle{Fact Types: Accumulated Snapshot Fact Table Example}
    An insurance company
    \begin{itemize}
        \item It has a fact table named: \textit{fact\_claim\_processing}.
        \item This fact represents the claim life-cycle inside the company.
        \item It contains detail related to claim.
        \item This fact update after each stage finished.
    \end{itemize}
\end{frame}
%--------------------------------------------------
\begin{frame}
    \frametitle{Fact Types: Accumulated Snapshot Fact Table Example}
    Example of Accumlated Snapshot: An insurance company
    \begin{itemize}
        \item It fact table named: fact\_claim\_processing.
        \item This fact represents the claim life-cycle inside the company.
        \item It contains detail related to claim.
        \item This fact update after each stage finished.
        \item The requirement it to report the number of days (lag) between stages (milestone) and the claim data (starting).
    \end{itemize}
    \vspace{5 mm}
    \scalebox{0.85}{
\smartdiagramset{back arrow disabled=true}
\smartdiagram[flow diagram:horizontal]{Request,Investigation,Review,Decision,Payment}
}
\vspace{1 mm}
\captionof{figure}{Claim Life-Cycle}

%%%%%%%%%%%%%%%%%%%%%%%%%%%%%%%%%%%%%%%%%%%%%%%%%%%%%%%%%%%%%%%%%%%%%%%%%%%
%%% Local Variables:
%%% mode: latex
%%% TeX-master: "../../../main.tex"
% !TeX root = ../../../main.tex
%%% TeX-engine: xetex
%%% End:

\end{frame}

%--------------------------------------------------
\begin{frame}
    \frametitle{Fact Types:Accumulated Snapshot Example}
    \setlength{\leftmargini}{2em}    
    \begin{columns}[T,onlytextwidth]
            \column{.4\textwidth}
                \setlength{\partopsep}{3pt}%AND EVEN REMOVING EXTRA itemize SPACE
                \begin{itemize}\vspace{-5mm}
%                   \itemsep 1.5em
                    \item One solution to implement the requirement is to use SCD.
                    \item In this case, we will have stages and dates, and we will calculate the difference between stages and dates using complex sub-query.
                    \item Another solution is to implement an accumulated snapshot fact.
                \end{itemize}
                \column{.4\textwidth}
                \setlength{\partopsep}{3pt}
                %\resizebox{\columnwidth}{!}

%%%%%%%%%%%%%%%%%%%%%%%%%%%%%%%%%%%%%%%%%%%%%%%%%%%%%%%%%%%%%%%%%%%%%%%%%%%
%%% Local Variables:
%%% mode: latex
%%% TeX-master: "../../main.tex"
% !TeX root = ../../main.tex
%%% TeX-engine: xetex
%%% End:

   \end{columns}    
\end{frame}

%--------------------------------------------------
\begin{frame}
    \frametitle{Fact Types:Accumulated Snapshot Example}
    \setlength{\leftmargini}{2em}    
    \setlength{\headsep}{2em}
    \begin{columns}[T,onlytextwidth]                       
                \column{.4\textwidth}
                    \setlength{\headsep}{2em}
                \setlength{\partopsep}{1pt}
                %\resizebox{\columnwidth}{!}

%%%%%%%%%%%%%%%%%%%%%%%%%%%%%%%%%%%%%%%%%%%%%%%%%%%%%%%%%%%%%%%%%%%%%%%%%%%
%%% Local Variables:
%%% mode: latex
%%% TeX-master: "../../main.tex"
% !TeX root = ../../main.tex
%%% TeX-engine: xetex
%%% End:

                \column{.5\textwidth}
                    \setlength{\headsep}{2em}
                \setlength{\partopsep}{1pt}
                %\resizebox{\columnwidth}{!}

%%%%%%%%%%%%%%%%%%%%%%%%%%%%%%%%%%%%%%%%%%%%%%%%%%%%%%%%%%%%%%%%%%%%%%%%%%%
%%% Local Variables:
%%% mode: latex
%%% TeX-master: "../../../main.tex"
% !TeX root = ../../../main.tex
%%% TeX-engine: xetex
%%% End:

   \end{columns}    
\end{frame}
%--------------------------------------------------
\begin{frame}
    \frametitle{Fact Types: Accumulated Snapshot Table Example}
    
    \begin{table}
    		\begin{tabular}{| l | l |}
           			\hline
                      column\_name & column\_value\\
                  	  \hline
                      \hline
                      claim\_key & 123\\
                      customer\_key& 5235326\\
                      policy\_key& 23632623\\
                      claim\_date& 2020-01-01\\
                      investigation\_date& 2020-01-03\\
                      day\_to\_investigate& 2\\
                      review\_date& 2020-01-07\\
                      day\_to\_review& 6\\
                      decision\_date& 2020-01-08\\
                      day\_to\_decision& 7\\
                      payment\_date& 2020-01-11\\
                      day\_to\_payment& 10\\
                      process\_completed\_flag& 10\\                      
    			\hline
    		\end{tabular}
    	\caption{Accumulated Snapshot Fact Example on Claim Process Data.}
    \end{table}
    
\end{frame}

%--------------------------------------------------
\begin{frame}
    \frametitle{Fact Table Types: Comparison}

    \begin{table}
        \resizebox{.9\textwidth}{!}{% <------ Don't forget this %
        \centering\renewcommand\cellalign{lc}
        \setcellgapes{1pt}\makegapedcells
    		\begin{tabular}{| l | l | l | l |}
           		\hline                
                   Feature & Transaction & Periodic & Accumulating\\
                \hline
           		\hline
                   Grain & 1 row/transaction & 1 row/time-period & \makecell{1 row/entire\\ event stages}\\
                \hline   
                   Date Dimension &  Lowest granularity& End-of-period granularity& Multiple date\\
                \hline   
                   Facts & Transaction activities & Periodic activities & \makecell{Defined lifetime\\ activities}\\
                \hline   
                   Size & Largest & Medium & Smallest\\
        		\hline
                   Update & No &  No &\makecell{Yes, after\\stage finished}\\
    			\hline
    		\end{tabular}
            }
    	\caption{Fact tables types comparison.}
    \end{table}
\end{frame}
%--------------------------------------------------
\midTitle{Fact types}
\begin{frame}
    \frametitle{Fact types}
    Each fact table includes facts and it has different types:
                \begin{itemize}
                    \item Additive facts.
                    \item Semi-additive facts.
                    \item Non-additive facts.
                    \item Derived facts.
                    \item Textual facts.
                    \item Factless fact.
                \end{itemize}
\end{frame}
%--------------------------------------------------

\begin{frame}
    \frametitle{Additive facts}
                \begin{itemize}
                    \item It is the most flexible and useful facts.
                    \item Its measures can be summed across any of the dimensions associated with the fact table.
                \end{itemize}

\end{frame}
%--------------------------------------------------

\begin{frame}
    \frametitle{Additive facts}
                \begin{itemize}
                    \item It is the most flexible and useful facts.
                    \item It can be summed across any of the dimensions associated with the fact table.
                \end{itemize}    
                \vspace{1cm}
                \centering
                %\resizebox{\columnwidth}{!}

%%%%%%%%%%%%%%%%%%%%%%%%%%%%%%%%%%%%%%%%%%%%%%%%%%%%%%%%%%%%%%%%%%%%%%%%%%%
%%% Local Variables:
%%% mode: latex
%%% TeX-master: "../../main.tex"
% !TeX root = ../../main.tex
%%% TeX-engine: xetex
%%% End:


\end{frame}
%--------------------------------------------------
\begin{frame}
    \frametitle{Semi-additive facts}
    
    \begin{itemize}
        \item It can be added across some dimensions but not all also known as (partially-additive).
    \end{itemize}
    \vspace{1cm}
    \centering
    %\resizebox{\columnwidth}{!}

%%%%%%%%%%%%%%%%%%%%%%%%%%%%%%%%%%%%%%%%%%%%%%%%%%%%%%%%%%%%%%%%%%%%%%%%%%%
%%% Local Variables:
%%% mode: latex
%%% TeX-master: "../../main.tex"
% !TeX root = ../../main.tex
%%% TeX-engine: xetex
%%% End:

    \begin{itemize}
        \item what's the total current balance for all accounts in the bank?
        \item What's the current balances for a given account for each day of the month does not give us any useful information?
    \end{itemize}
    
\end{frame}
%--------------------------------------------------
\begin{frame}
    \frametitle{Non-additive facts}
    
    \begin{itemize}
        \item It can't be added for any of the dimensions.
        \item Non-additive facts are usually the result of ratios (percentage) or other mathematical calculations.
        \item \textbf{Profit\_Margin} is an example non-additive.
    \end{itemize}

    \vspace{1cm}
    \centering
    %\resizebox{\columnwidth}{!}

%%%%%%%%%%%%%%%%%%%%%%%%%%%%%%%%%%%%%%%%%%%%%%%%%%%%%%%%%%%%%%%%%%%%%%%%%%%
%%% Local Variables:
%%% mode: latex
%%% TeX-master: "../../main.tex"
% !TeX root = ../../main.tex
%%% TeX-engine: xetex
%%% End:


\end{frame}

%--------------------------------------------------

\begin{frame}
    \frametitle{Derived facts}
    
    \begin{itemize}
        \item Derived facts are created by performing a mathematical calculation
        on a number of other facts, and are sometimes referred to as
        calculated facts. Derived facts may or may not be stored inside the
        fact table.
        \item Total\_sales = Qty\_Sold * ( Unit\_price - Discount)
    \end{itemize}

    \centering
    %\resizebox{\columnwidth}{!}

%%%%%%%%%%%%%%%%%%%%%%%%%%%%%%%%%%%%%%%%%%%%%%%%%%%%%%%%%%%%%%%%%%%%%%%%%%%
%%% Local Variables:
%%% mode: latex
%%% TeX-master: "../../main.tex"
% !TeX root = ../../main.tex
%%% TeX-engine: xetex
%%% End:



\end{frame}
%--------------------------------------------------
\begin{frame}
    \frametitle{Textual facts}
    \begin{itemize}
        \item A textual fact consists of one or more characters such as flags and indicators.
        \item \red{It should be avoided in the fact table}.
    \end{itemize}
    
\end{frame}
%--------------------------------------------------
\begin{frame}
\frametitle{Factless fact}
    \begin{itemize}
        \item A fact table with only foreign keys and no facts is called a factless
        fact table.
    \end{itemize}
\end{frame}
%--------------------------------------------------
\begin{frame}
    \frametitle{References}
        \begin{itemize}
            \item https://www.nuwavesolutions.com/accumulating-snapshot-fact-tables/
            \item https://www.kimballgroup.com/2008/11/fact-tables/
            \item https://www.1keydata.com/datawarehousing/fact-table-types.html
        \end{itemize}
\end{frame}
%%%%%%%%%%%%%%%%%%%%%%%%%%%%%%%%%%%%%%%%%%%%%%%%%%%%%%%%%%%%%%%%%%%%%%%%%%%%
%%% Local Variables:
%%% mode: latex
%%% TeX-master: "../main"
% !TeX root = ../main.tex
%%% TeX-engine: xetex
%%% End: