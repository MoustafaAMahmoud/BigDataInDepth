%%%%%%%%%%%%%%%%%%%%%%%%%%%%%%%%%%%%%%%%%%%%%%%%%%%%%%
\VideoClassification[column=2, colour=red]
%%%%%%%%%%%%%%%%%%%%%%%%%%%%%%%%%%%%%%%%%%%%%%%%%%%%%%

\midTitle{Schema Types}
\begin{frame}
\frametitle{Schema Types}
	\begin{itemize}[<+->]
		\item Star Schema.
		\item Snowflake Schema.
	\end{itemize}

\end{frame}
%%%%%%%%%%%%%%%%%%%%%%%%%%%%%%%%%%%%%%%%%%%%%%%%%%%%%%
%%%%%%%%%%%%%%%%%%%%%%%%%%%%%%%%%%%%%%%%%%%%%%%%%%%%%%
\midTitle{Schema Types: Star Schema}
\begin{frame}
	\frametitle{Star Schema Characteristics}
	\begin{itemize}[<+->]
		\item \textbf{Simplicity}: It is the simplest type of DWH schemas.
		\item \textbf{Query effectiveness}: Because of simplicity, It needs less join to query the data (It is optimized to query large dataset).
		\item \textbf{Data Redundancy and Large Table Size}: Due to de-normalization it has a data redundancy and the table size is very huge.
		\item \textbf{Most} used and \textbf{widely} supported.
	\end{itemize}
\end{frame}
%%%%%%%%%%%%%%%%%%%%%%%%%%%%%%%%%%%%%%%%%%%%%%%%%%%%%%
\begin{frame}
\frametitle{Star Schema Characteristics}
	\begin{itemize}[<+->]
		\item Dimensions represented by one one-dimension table.
		\item The dimension table are not joined to each other
		\item Fact table would contains key and measure.
		\item Data integrity is not enforced due de-normalized structure.
	\end{itemize}
\end{frame}
%%%%%%%%%%%%%%%%%%%%%%%%%%%%%%%%%%%%%%%%%%%%%%%%%%%%%%
\begin{frame}
\frametitle{Schema Types: Star Schema Example}

\resizebox{\columnwidth}{!}{%
\begin{tikzpicture}[every node/.style={font=\ttfamily}, node distance=1.4in,scale=.75, every node/.style={scale=0.75}]
%https://tex.stackexchange.com/questions/133754/creating-crows-foot-style-e-r-diagrams-rather-than-chen-style-ones
\matrix  [entity=Usage, entity anchor=Usage-id]  {
	\properties{
		id,
		cust-id (FK),
		cal-id (FK), 
		loc-id (FK),
		promo-id (FK),
		date-id (FK),
		TotalInCalls (agg),
		TotalOutCalls (agg),
		TotalAmount (agg)
	}
};


\matrix  [entity=CellLookup, above left=of Usage-id, entity anchor=CellLookup-id]  {
	\properties{
		id,
		celltype,
		vendorname,
		street,
		city,
		state,
		zip
	}
};
\matrix  [entity=Promotion, below left=of Usage-id,yshift=10ex, entity anchor=Promotion-id]  {
	\properties{
		id,
		promotype,
		promodesc,
		value,
		startdate,
		enddate
	}
};

\matrix [entity=CustomerProfile, below right=of Usage-id,yshift=10ex, entity anchor=CustomerProfile-id]  {
	\properties{
		id, 
		gender, 
		age, 
		nationality,
		firstname,
		lastname
	}
};


\matrix  [entity=Calendar, above right=of Usage-id, entity anchor=Calendar-id]  {
	\properties{
		id,
		date,
		day,
		week,
		month,
		qtr,
		year
	}
};

\draw [one to one] (Usage-id)  to (CustomerProfile-id);
\draw [one to one] (Usage-id)  to (Calendar-id);
\draw [one to one] (Usage-id)  to (CellLookup-id);
\draw [one to one] (Usage-id)  to (Promotion-id);

\end{tikzpicture}
}
%%%%%%%%%%%%%%%%%%%%%%%%%%%%%%%%%%%%%%%%%%%%%%%%%%%%%%%%%%%%%%%%%%%%%%%%%%%
%%% Local Variables:
%%% mode: latex
%%% TeX-master: "../../main.tex"
% !TeX root = ../../main.tex
%%% TeX-engine: xetex
%%% End:

\end{frame}
%%%%%%%%%%%%%%%%%%%%%%%%%%%%%%%%%%%%%%%%%%%%%%%%%%%%%%%
\midTitle{Schema Types: Snowflake Schema}
\begin{frame}
	\frametitle{What is Snowflake?}
	\begin{figure}
	\includegraphics[scale=0.25]{Ch01-Introduction-data-management/12-Data-Model/04-Data-Model-Schema-Types/Figures/snowflake-real.jpg}
	\caption{Snowflake Photo taken from  \href{https://earthsky.org/earth/best-snowflakes-photos-from-earthsky-friends}{https://earthsky.org}}
	\end{figure}
\end{frame}
%%%%%%%%%%%%%%%%%%%%%%%%%%%%%%%%%%%%%%%%%%%%%%%%%%%%%%%
\begin{frame}
	\frametitle{What is Snowflake?}
	\begin{figure}
		\includegraphics[scale=0.5]{Ch01-Introduction-data-management/12-Data-Model/04-Data-Model-Schema-Types/Figures/Snowflakes-PNG-File.png}
		\caption{Snowflake Simple Design}
	\end{figure}
\end{frame}
%%%%%%%%%%%%%%%%%%%%%%%%%%%%%%%%%%%%%%%%%%%%%%%%%%%%%%%
\begin{frame}
	\frametitle{What is Snowflake?}
	\begin{figure}
		\includegraphics[scale=0.2]{Ch01-Introduction-data-management/12-Data-Model/04-Data-Model-Schema-Types/Figures/Frozen-Snowflake-PNG-File.png}
		\caption{Snowflake Final Design}
	\end{figure}
\end{frame}
%%%%%%%%%%%%%%%%%%%%%%%%%%%%%%%%%%%%%%%%%%%%%%%%%%%%%%%
\begin{frame}
\frametitle{Star Schema Characteristics}
	\begin{itemize}
		\item \textbf{Extension}: Snowflake is an extension of the Star Schema.
		\item \textbf{Normalized}: Dimension tables are normalized this means every dimension could be expanded into additional tables.
		\item \textbf{Disk Space Efficiency}: Due to its normalization methodology it uses less desk space which enhance the query as we scan less data size.
		\item \textbf{Complicated}: Due to the normalization query needs to join more table in some cases to get the data which reduces the performance.
	\end{itemize}

\end{frame}

%%%%%%%%%%%%%%%%%%%%%%%%%%%%%%%%%%%%%%%%%%%%%%%%%%%%%%
\begin{frame}
	\frametitle{Schema Types: Star schema (Example)}
	\begin{tikzpicture}[every node/.style={font=\ttfamily}, node distance=1.4in,scale=.75, every node/.style={scale=0.75}]
%https://tex.stackexchange.com/questions/133754/creating-crows-foot-style-e-r-diagrams-rather-than-chen-style-ones
\matrix  [entity=Usage, entity anchor=Usage-id]  {
	\properties{
		id,
		cust-id (FK),
		cal-id (FK), 
		loc-id (FK),
		promo-id (FK),
		date-id (FK),
		TotalInCalls (agg),
		TotalOutCalls (agg),
		TotalAmount (agg)
	}
};


\matrix [entity=CustomerProfile, below right=of Usage-id,yshift=10ex, entity anchor=CustomerProfile-id]  {
	\properties{
		id, 
		gender, 
		age, 
		NationalityID,
		firstname,
		lastname
	}
};

\matrix [entity=Nationality, below right=of CustomerProfile-id,yshift=10ex, entity anchor=Nationality-id]  {
	\properties{
		id, 
		name
	}
};


\matrix  [entity=Calendar, above right=of Usage-id, entity anchor=Calendar-id]  {
	\properties{
		id,
		date,
		day,
		week,
		month,
		qtr,
		year
	}
};

\draw [one to one] (Usage-id)  to (CustomerProfile-id);
\draw [one to one] (Usage-id)  to (Calendar-id);
\draw [one to one] (CustomerProfile-id)  to (Nationality-id);

\end{tikzpicture}
%%%%%%%%%%%%%%%%%%%%%%%%%%%%%%%%%%%%%%%%%%%%%%%%%%%%%%%%%%%%%%%%%%%%%%%%%%%
%%% Local Variables:
%%% mode: latex
%%% TeX-master: "../../main.tex"
% !TeX root = ../../main.tex
%%% TeX-engine: xetex
%%% End:

\end{frame}

%%%%%%%%%%%%%%%%%%%%%%%%%%%%%%%%%%%%%%%%%%%%%%%%%%%%%%%
\begin{frame}
\frametitle{Star Vs Snowflake Schema}
\vspace{.1cm}
	\begin{tabular}{| l | l |}
		\hline
		Star & Snowflake\\
		\hline
		 \makecell{Dimension represented by one-table} &  \makecell{Dimension tables are expanded\\ into multi-tables }\\
		 		\hline
		\makecell{Fact table surrounded\\ by dimension tables} & 
		\makecell{Fact table surrounded by\\Hierarchy of dimension tables} \\
				\hline
		 \makecell{Less join}
		 & \makecell{Requires many joins}\\
 		\hline
		Simple Design & Very Complex Design\\ %easy for understanding vs difficult
		\hline
		De-normalized Data structure & Normalized Data Structure\\
		\hline
		High level of Data redundancy & Very low-level data redundancy\\
		\hline
		Maintenance is difficult & Maintenance is easier\\
		\hline
		\makecell{Good for datamarts with\\ simple relationships\\ (1:1 or 1:many)} & \makecell{Good for core \\to simplify (many:many)}\\
		\hline

	\end{tabular}
\end{frame}

%%%%%%%%%%%%%%%%%%%%%%%%%%%%%%%%%%%%%%%%%%%%%%%%%%%%%%%%%%%%%%%%%%%%%%%%%%%%
%%% Local Variables:
%%% mode: latex
%%% TeX-master: "../main"
% !TeX root = ../main.tex
%%% TeX-engine: xetex
%%% End: