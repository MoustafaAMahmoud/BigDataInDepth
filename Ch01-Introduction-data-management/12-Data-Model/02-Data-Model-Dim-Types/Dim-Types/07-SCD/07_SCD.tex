%%%%%%%%%%%%%%%%%%%%%%%%%%%%%%%%%%%%%%%%%%%%%%%%%%%%%%
\VideoClassification[column=1, colour=red]
%%%%%%%%%%%%%%%%%%%%%%%%%%%%%%%%%%%%%%%%%%%%%%%%%%%%%%
\midTitle{Dimensions Types: Slowly changing Dimensions}
\begin{frame}
    \frametitle{Slowly changing Dimensions}
    %https://www.guru99.com/dimensional-model-data-warehouse.html
    \begin{itemize}[<+->]
		\item It the dimension which changes over time. So, for a specific date we have different value.
		\item It has different types as following
		    \begin{itemize}[<+->]
        \item Type 0 (Fixed Dimension): We don't change the current even the source changes. 
        \item Type 1 (No History): No history is maintained only the latest replace the current.
        \item Type 2 (History): Series of history of records are maintained.
        \item Type 3 (Hybrid): Only the last Change and the Current new change is stored
        \item Type 4 : We split the data into two tables, first the current record and second is the historical (most common usage).
    \end{itemize}   


        %slowly changing dim assume customer was located in dubai then he changed his location
        % We don't care about the change. just update the latest
        % we can have two version new, old.
        % we can have sergeate key.
        %with start and end data.
        %start and end is null
        %join with sergate key
    \end{itemize}
\end{frame}
%%%%%%%%%%%%%%%%%%%%%%%%%%%%%%%%%%%%%%%%%%%%%%%%%%%%%%
\begin{frame}
	\frametitle{Slowly changing Dimensions}
\begin{block}{Note}
	\textit{There are some other types which is a combination between the above similar than type 3 combined between 1 \& 2. \\ You can check the chapter resources for more information about the other types.}
\end{block}
%https://www.kimballgroup.com/2013/02/design-tip-152-slowly-changing-dimension-types-0-4-5-6-7/     
\end{frame}
%%%%%%%%%%%%%%%%%%%%%%%%%%%%%%%%%%%%%%%%%%%%%%%%%%%%%%
\begin{frame}
	\frametitle{Slowly changing Dimensions}
	
	\begin{itemize}
		\item Type 0.
	\end{itemize}
	\begin{table}[t]
		\centering
		\sffamily
		\begin{tabular}{|l | l | a |}
			\hline
			CustomerID & Name & City\\
			\hline
			\hline			
			123456789 & Ronaldo  & Madrid\\
			\hline
		\end{tabular}
		\quad
		\begin{tabular}{|l | l| a|}
			\hline
			CustomerID & Name & City\\
			\hline
			\hline			
			123456789 & Ronaldo  & Turin\\
			\hline
		\end{tabular}
		\caption{Source System Old vs New}
	\end{table}
	
	\begin{table}[t]
		\centering
		\sffamily
		\begin{tabular}{|l | l | l |a|}
			\hline
			ID & CustomerID & Name & City\\
			\hline
			\hline		
			1 & 123456789 & Ronaldo  & Madrid\\
			\hline
		\end{tabular}
		\caption{Customer Profile Dimension}
	\end{table}
		
\end{frame}
%%%%%%%%%%%%%%%%%%%%%%%%%%%%%%%%%%%%%%%%%%%%%%%%%%%%%%
\begin{frame}
	\frametitle{Slowly changing Dimensions}
	
	\begin{itemize}
		\item Type 1.
	\end{itemize}
	\begin{table}[t]
		\centering
		\sffamily
		\begin{tabular}{|l | l | a |}
			\hline
			CustomerID & Name & City\\
			\hline
			\hline			
			123456789 & Ronaldo  & Madrid\\
			\hline
		\end{tabular}
		\quad
		\begin{tabular}{|l | l| a|}
			\hline
			CustomerID & Name & City\\
			\hline
			\hline			
			123456789 & Ronaldo  & Turin\\
			\hline
		\end{tabular}
		\caption{Source System Old vs New}
	\end{table}
	
	\begin{table}[t]
		\centering
		\sffamily
		\begin{tabular}{|l | l | l |a|}
			\hline
			ID & CustomerID & Name & City\\
			\hline
			\hline		
			1 & 123456789 & Ronaldo  & Turin\\
			\hline
		\end{tabular}
		\caption{Customer Profile Dimension}
	\end{table}	
\end{frame}
%%%%%%%%%%%%%%%%%%%%%%%%%%%%%%%%%%%%%%%%%%%%%%%%%%%%%%
\begin{frame}
	\frametitle{Slowly changing Dimensions}
	\begin{itemize}
		\item Type 2.
	\end{itemize}
	\begin{table}[t]
		\centering
		\sffamily
		\begin{tabular}{|l | l | a |l|}
			\hline
			CustomerID & Name & City & UpdatedDt \\
			\hline
			\hline			
			123456789 & Ronaldo  & Madrid & 2018-12-12\\
			\hline
			\hline			
			123456789 & Ronaldo  & Turin & 2019-06-12\\
			\hline
			\hline			
			123456789 & Ronaldo  & London & 2019-08-12\\
			\hline
			\hline						
			123456789 & Ronaldo  & Porto & 2019-12-12\\		
			\hline
		\end{tabular}	
		\caption{Source System Old vs New}
	\end{table}%
	\vspace{-.8cm}

	\begin{table}[t]
		\centering
		\sffamily
		  \begin{adjustbox}{max width=\textwidth}			
		\begin{tabular}{|l | l | l | a | l | l |a|}
			\hline
			ID & CustomerID & Name & City & effectiveDt & TerminationDt & isCurrent\\
			\hline
			\hline		
			1 & 123456789 & Ronaldo  & Madrid & 2018-12-12 & 2019-06-12 & false\\
			2 & 123456789 & Ronaldo  & Turin & 2019-06-12 & 2019-08-12 & false\\
			3 & 123456789 & Ronaldo  & London & 2019-08-12 & 2019-12-12 & false\\
			4 & 123456789 & Ronaldo  & Porto  & 2019-12-12 & null & true\\
			\hline
		\end{tabular}
		\end{adjustbox}

		\caption{Customer Profile Dimension {\scriptsize We can replace null with a finite date (9999-12-31) but it needs to be consistent}}
	\end{table}
	
\end{frame}
%%%%%%%%%%%%%%%%%%%%%%%%%%%%%%%%%%%%%%%%%%%%%%%%%%%%%%
\begin{frame}
	\frametitle{Slowly changing Dimensions}
	\begin{itemize}
		\item Type 3.
	\end{itemize}
	\begin{table}[t]
		\centering
		\sffamily
		\begin{tabular}{|l | l | a |l|}
			\hline
			CustomerID & Name & City & UpdatedDt \\
			\hline
			\hline			
			123456789 & Ronaldo  & Madrid & 2018-12-12\\
			\hline
			\hline			
			123456789 & Ronaldo  & Turin & 2019-06-12\\
			\hline
			\hline			
			123456789 & Ronaldo  & London & 2019-08-12\\
			\hline
			\hline						
			123456789 & Ronaldo  & Porto & 2019-12-12\\		
			\hline
		\end{tabular}	
		\caption{Source System Old vs New}
		
	\end{table}
	
	\begin{table}[t]
		\centering
		\sffamily
		\begin{tabular}{|l | l | l | a | l | l |}
			\hline
			ID & CustomerID & Name & City & UpdatedDate  & previousCity\\
			\hline
			\hline		
			1 & 123456789 & Ronaldo  & Porto  & 2019-12-12 & London\\
			\hline
		\end{tabular}
		\caption{Customer Profile Dimension}
	\end{table}
\end{frame}

%%%%%%%%%%%%%%%%%%%%%%%%%%%%%%%%%%%%%%%%%%%%%%%%%%%%%%
\begin{frame}
	\frametitle{Slowly changing Dimensions}
	\begin{itemize}
		\item Type 4 (Split current and Historical).
	\end{itemize}

	\begin{table}[t]
	\centering
	\sffamily
	\begin{tabular}{|l | l | l | a | l | l |a|}
		\hline
		ID & CustomerID & Name & City & effectiveDt & TerminationDt\\
		\hline
		\hline		
		1 & 123456789 & Ronaldo  & Madrid & 2018-12-12 & 2019-06-12\\
		2 & 123456789 & Ronaldo  & Turin & 2019-06-12 & 2019-08-12\\
		3 & 123456789 & Ronaldo  & London & 2019-08-12 & 2019-12-12\\
		4 & 123456789 & Ronaldo  & Porto  & 2019-12-12 & null\\
		\hline
	\end{tabular}
	\caption{Customer Profile Dimension Hist}
\end{table}

	
	\begin{table}[t]
		\centering
		\sffamily
		\begin{tabular}{|l | l | l | a | l |}
			\hline
			ID & CustomerID & Name & City & UpdatedDate\\
			\hline
			\hline		
			1 & 123456789 & Ronaldo  & Porto  & 2019-12-12\\
			\hline
		\end{tabular}
		\caption{Customer Profile Dimension}
	\end{table}

\end{frame}

%%%%%%%%%%%%%%%%%%%%%%%%%%%%%%%%%%%%%%%%%%%%%%%%%%%%%%
\begin{frame}[fragile]
	\frametitle{Slowly changing Dimensions}
	\begin{itemize}
		\item How does the Facts join SCD? We have two scenarios as following:
		\begin{itemize}
			\item Getting the current customer information (Join with the latest).
			\item Getting the historical customer information (Join with the historical table based on \textbf{\textit{cust id \& date}}).
		\end{itemize}
	\end{itemize}

	\begin{table}[t]
		\centering
		\sffamily
		\begin{tabular}{|l | a | l | l |}
			\hline
			ID & CustomerID & TotalCalls & CallDate \\
			\hline
			\hline		
			1 & 123456789 & 30 & 2018-12-12 \\
			2 & 123456789 & 30 & 2019-12-12 \\
			\hline
		\end{tabular}
		\caption{Customer Usage}
	\end{table}

	
\end{frame}
%%%%%%%%%%%%%%%%%%%%%%%%%%%%%%%%%%%%%%%%%%%%%%%%%%%%%%

%%%%%%%%%%%%%%%%%%%%%%%%%%%%%%%%%%%%%%%%%%%%%%%%%%%%%%%
\begin{frame}[fragile]
	\frametitle{Slowly changing Dimensions}
	\lstinputlisting[language=sql,caption={Example to show how to use SCD}]{Ch01-Introduction-data-management/12-Data-Model/02-Data-Model-Dim-Types/Dim-Types/07-SCD/Code/SCD_Examply.sql}
	
\end{frame}




%%%%%%%%%%%%%%%%%%%%%%%%%%%%%%%%%%%%%%%%%%%%%%%%%%%%%%%%%%%%%%%%%%%%%%%%%%%%
%%% Local Variables:
%%% mode: latex
%%% TeX-master: "../../../../../main"
% !TeX root = ../../../../../main.tex
%%% TeX-engine: xetex