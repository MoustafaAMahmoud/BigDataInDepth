
%%%%%%%%%%%%%%%%%%%%%%%%%%%%%%%%%%%%%%%%%%%%%%%%%%%%%
\VideoClassification[column=1, colour=blue]
%%%%%%%%%%%%%%%%%%%%%%%%%%%%%%%%%%%%%%%%%%%%%%%%%%%%%
\midTitle{Dimensions Types: Fast (Rapidly) Changing Dimension (Mini Dimension) }
\begin{frame}
	\frametitle{Fast Changing Dimension (Mini Dimension)}
	\begin{itemize}[<+->]
		\item When we have a dimension with one or more of its attributes changing very fast. 
		
		\item \blue{\textit{\faBullhorn \space It causes a performance issue if we tried to handle this case similar SCD Type 2 because of the rapidly changing  in this dimension and the table will includes a lot of rows for this dimension}}.

		\item We solve this case by separation the attributes into one or more dimensions. This technique also called \textit{\textbf{mini-dimensions}}.		
	
	\end{itemize}
\end{frame}
%%%%%%%%%%%%%%%%%%%%%%%%%%%%%%%%%%%%%%%%%%%%%%%%%%%%%
\begin{frame}
	\frametitle{Fast Changing Dimension (Mini Dimension)}
	\begin{itemize}[<+->]
		\item How to implement FCD (Mini Dimension)? \underline{{\footnotesize \textit{Hint: Search for the mini-dimension  relation table.}}}
	\end{itemize}

\begin{table}
	\begin{adjustbox}{max width=.95\textwidth}			
		\begin{tabular}{| l | l | l | l | a | a | l|}
			\hline
			Patient\_id & Name & Gender & BirthDate & Weight & B\_Presaure & UpdateDt\\
			\hline
			\hline		
			123 & Anna   & F & 1968-01-12 & 50 & 110.0 &2019-01-01\\
			123 & Anna   & F & 1968-01-12 & 55 & 130.0 &2019-01-07\\
			123 & Anna   & F & 1968-01-12 & 59 & 115.0 &2019-01-14\\
			123 & Anna   & F & 1968-01-12 & 65 & 120.0 &2019-01-21\\
			\hline
		\end{tabular}
	\end{adjustbox}
	\caption{Patient Profile Dimension}
\end{table}

\begin{table}
	\resizebox{.97\columnwidth}{!}{%
		
		\begin{tabular}{| l | l | l | l |}
			\hline
			Patient\_id & Name & Gender & BirthDate \\
			\hline 
			\hline		
			123 & Anna   & F & 1968-01-12 \\
			\hline
		\end{tabular}

\quad

		\begin{tabular}{| l | l | l | l |}
			\hline
			Patient\_Key & Weight & B\_Presure \\
			\hline
			\hline		
			1 & 50 & 110.0\\
			2 & 55 & 130.0\\
			3 & 59 & 115.0\\
			4 & 65 & 120.0\\
			\hline
		\end{tabular}
	}
	\caption{Patient Profile Dimension After Removing FCD and Split it into Junk-Dimension table}
\end{table}

\end{frame}

%%%%%%%%%%%%%%%%%%%%%%%%%%%%%%%%%%%%%%%%%%%%%%%%%%%%%
\begin{frame}
	\frametitle{Fast Changing Dimension (Mini Dimension)}

\begin{table}
	\begin{adjustbox}{max width=.7\textwidth}
		\begin{tabular}{| l | l | l | l |}
			\hline
			Patient\_id & Patient\_Key & Start\_Date & End\_Date\\
			\hline
			\hline		
			123 & 1   & 2019-01-01 & 2019-01-07\\
			123 & 2   & 2019-01-07 & 2019-01-14\\
			123 & 3   & 2019-01-14 & 2019-01-21\\
			123 & 4   & 2019-01-21 & null\\
			\hline
		\end{tabular}
	\end{adjustbox}
	\caption{Patient Mini Dimension}
\end{table}
\begin{tikzpicture}[every node/.style={font=\ttfamily}, node distance=1.4in,scale=.7, every node/.style={scale=0.7}]
  
    \matrix  [entity=Patient, entity anchor=Patient-ID]  {
	    \properties{
	    ID,
	    Name,
	    Gender,
	    BirthDate
	    }
    };
    \matrix  [entity=PatientRelationDim, left=of Patient-ID,xshift=9ex,entity anchor=PatientRelationDim-PatientID]  {
		\properties{
			PatientID,
			PatientKey,
			StartDate,
			EndDate
		}
	};
%
	\matrix  [entity=PatientMeasureDetails, left=of PatientRelationDim-PatientID,xshift=6ex, entity anchor=PatientMeasureDetails-PatientKey]  {
		\properties{
			PatientKey,
			Weight,
			BPresure
		}
	};

\draw [one to one] (Patient-ID)  to (PatientRelationDim-PatientID);
\draw [one to one] (PatientRelationDim-PatientID)  to (PatientMeasureDetails-PatientKey);

\end{tikzpicture}

%%%%%%%%%%%%%%%%%%%%%%%%%%%%%%%%%%%%%%%%%%%%%%%%%%%%%%%%%%%%%%%%%%%%%%%%%%%
%%% Local Variables:
%%% mode: latex
%%% TeX-master: "../../main.tex"
% !TeX root = ../../main.tex
%%% TeX-engine: xetex
%%% End:

\end{frame}



%%%%%%%%%%%%%%%%%%%%%%%%%%%%%%%%%%%%%%%%%%%%%%%%%%%%%%%%%%%%%%%%%%%%%%%%%%%%
%%% Local Variables:
%%% mode: latex
%%% TeX-master: "../../../../../main"
% !TeX root = ../../../../../main.tex
%%% TeX-engine: xetex