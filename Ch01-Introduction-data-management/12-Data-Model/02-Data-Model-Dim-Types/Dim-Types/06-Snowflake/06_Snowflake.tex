%%%%%%%%%%%%%%%%%%%%%%%%%%%%%%%%%%%%%%%%%%%%%%%%%%%%%%
\VideoClassification[column=1, colour=blue]
%%%%%%%%%%%%%%%%%%%%%%%%%%%%%%%%%%%%%%%%%%%%%%%%%%%%%%
\midTitle{Dimensions Types: Snowflake Dimensions}
\begin{frame}
    \frametitle{Snowflake Dimensions}
    \begin{itemize}[<+->]
        \item Snowflake Dimension is a dimension that has a hierarchy of attributes. This attribute is normalized, and each dimension has a relationship with another hierarchy dimension table.\\
        \item \red{\textit{\faBug This dimension design not recommended as it has much complexity to the model and query performance. Also, it complicates the ETL process and makes too many dimensions without needs}}.
    \end{itemize}
\end{frame}
%%%%%%%%%%%%%%%%%%%%%%%%%%%%%%%%%%%%%%%%%%%%%%%%%%%%%%
\begin{frame}
    \frametitle{Snowflake Dimensions}
    \begin{itemize}
        \item Snowflake Dimension is a dimension that has a hierarchy of attributes. This attribute is normalized, and each dimension has a relationship with another hierarchy dimension table.\\
		\item \red{\textit{\faBug This dimension design not recommended as it has much complexity to the model and query performance. Also, it complicates the ETL process and makes too many dimensions without needs}}.
    \end{itemize}
    \centering
    \resizebox{.9\columnwidth}{!}{%
\begin{tikzpicture}[every node/.style={font=\ttfamily}, node distance=1.4in,scale=.75, every node/.style={scale=0.75}]
    \matrix  [entity=Usage, entity anchor=Usage-id]  {
        \properties{
        id,
        cust-id (FK),
        promo-id (FK),
        usgdate-id (FK)
        }
    };
    \matrix  [entity=Promotion, below left=of Usage-id,yshift=10ex, entity anchor=Promotion-id]  {
        \properties{
        id,
        promotype,
        promodesc,
        subcategory-id (FK),
        value
        }
    };
    \matrix  [entity=PromoSubCategory, below left=of Promotion-id,yshift=10ex, entity anchor=PromoSubCategory-id]  {
        \properties{
        id,
        subcategory,
        category-id (FK),
        }
    };
    \matrix  [entity=PromoCategory, below left=of PromoSubCategory-id,yshift=10ex, entity anchor=PromoCategory-id]  {
        \properties{
        id,
        category,
        }
    };


    \draw [one to one] (Usage-id)  to (Promotion-id);
    \draw [one to one] (Promotion-id)  to (PromoSubCategory-id);
    \draw [one to one] (PromoSubCategory-id) to (PromoCategory-id);
\end{tikzpicture}
}

%%%%%%%%%%%%%%%%%%%%%%%%%%%%%%%%%%%%%%%%%%%%%%%%%%%%%%%%%%%%%%%%%%%%%%%%%%%
%%% Local Variables:
%%% mode: latex
%%% TeX-master: "../../main.tex"
% !TeX root = ../../main.tex
%%% TeX-engine: xetex
%%% End:

\end{frame}



%%%%%%%%%%%%%%%%%%%%%%%%%%%%%%%%%%%%%%%%%%%%%%%%%%%%%%%%%%%%%%%%%%%%%%%%%%%%
%%% Local Variables:
%%% mode: latex
%%% TeX-master: "../../../../../main"
% !TeX root = ../../../../../main.tex
%%% TeX-engine: xetex