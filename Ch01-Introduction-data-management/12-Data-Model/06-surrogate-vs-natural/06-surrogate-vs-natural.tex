%%%%%%%%%%%%%%%%%%%%%%%%%%%%%%%%%%%%%%%%%%%%%%%%%%%%%%
%\VideoClassification[column=2, colour=blue]
%%%%%%%%%%%%%%%%%%%%%%%%%%%%%%%%%%%%%%%%%%%%%%%%%%%%%%
% converting from logical to physical model
\midTitle{Surrogate vs Natural Key}
%%%%%%%%%%%%%%%%%%%%%%%%%%%%%%%%%%%%%%%%%%%%%%%%%%%%%%
\begin{frame}
\frametitle{Entity Identification}
	\begin{itemize}[<+->]
		\item Each entity needs an identifier column.
		\item Identifier attribute for each instance (row) helps to distinguish between the instances or row of the same entity.		
	\end{itemize}

\end{frame}
%%%%%%%%%%%%%%%%%%%%%%%%%%%%%%%%%%%%%%%%%%%%%%%%%%%%%%
\begin{frame}
	\frametitle{Entity Identification}
	\begin{itemize}
		\item Each entity needs to an identifier column.
		\item Identifier attribute for each instance (row) helps to distinguish between the instances or row of the same entity.		
	\end{itemize}

	\centering
	\begin{tikzpicture}[every node/.style={font=\normalsize}, node distance=1.4in,text=blue]

    \matrix  [entity=Cust_Profile, entity anchor=Cust_Profile-Cust_ID]  {
        \properties{
        Cust_ID,
        Cust_Name,
        Cust_BirthDate,
        Cust_Gender
        }
    };

\end{tikzpicture}
%%%%%%%%%%%%%%%%%%%%%%%%%%%%%%%%%%%%%%%%%%%%%%%%%%%%%%%%%%%%%%%%%%%%%%%%%%%
%%% Local Variables:
%%% mode: latex
%%% TeX-master: "../../../main.tex"
% !TeX root = ../../../main.tex
%%% TeX-engine: xetex
%%% End:

	
\end{frame}

%%%%%%%%%%%%%%%%%%%%%%%%%%%%%%%%%%%%%%%%%%%%%%%%%%%%%%
\begin{frame}
	\frametitle{Natural vs Surrogate}
	\hspace{.5cm}
	  \resizebox{\textwidth}{!}{
	\begin{tabular}{  | c | c | c |  }
	\hline
	Metrics &	Surrogate Key & Natural Key \\
	\hline
	Uniqueness & Yes & Yes\\
	\hline
	Name &	\makecell{Artificial,\\ System generated} & Business key \\ 
	\hline
	\makecell{Business \\Meaning} &  \makecell{Doesn't have a \\business meaning} & \makecell{has a business\\ meaning}\\
	\hline
	\makecell{Conceptual\\ Relation} & \makecell{Doesn't relate} & \makecell{Part of \\conceptual model}\\
	\hline
	Creation & System (Database) & \makecell{Set of column(s)\\ from the data}\\		
	\hline    
	\end{tabular}
}
\end{frame}
%%%%%%%%%%%%%%%%%%%%%%%%%%%%%%%%%%%%%%%%%%%%%%%%%%%%%%


%%%%%%%%%%%%%%%%%%%%%%%%%%%%%%%%%%%%%%%%%%%%%%%%%%%%%%
\begin{frame}
	\frametitle{Natural or Surrogate?}
	\vspace{2.6cm}
\begin{center}
	\hspace{1.5cm}	{\Huge It depends \\
	\hspace{1.5cm}			:):}
\end{center}
\end{frame}
%%%%%%%%%%%%%%%%%%%%%%%%%%%%%%%%%%%%%%%%%%%%%%%%%%%%%%
\begin{frame}
	\frametitle{Natural or Surrogate?}
	\vspace{1cm}
	It depends on several factors:
		\begin{itemize}[<+->]
			\item The nature of the data.
			\item Database (DWH) platform.
			\item The group who uses this data from development or DBA team.		
		\end{itemize}		
\end{frame}
%%%%%%%%%%%%%%%%%%%%%%%%%%%%%%%%%%%%%%%%%%%%%%%%%%%%%%
\begin{frame}
	\frametitle{Natural Key Pros and Cons}

	\hspace{.5cm}
	\begin{itemize}
		\item	Pros
		\begin{itemize}
			\item Key has a business meaning.
			\item No extra space or computation needed.
		\end{itemize}	
	\end{itemize}

\begin{itemize}
		\item	Cons
		\begin{itemize}
			\item Could change over time.
			\item Difficult for maintenance.
			\item It required more space or a combination of multi-column
		\end{itemize}	
	\end{itemize}	

\end{frame}
%%%%%%%%%%%%%%%%%%%%%%%%%%%%%%%%%%%%%%%%%%%%%%%%%%%%%%
%%%%%%%%%%%%%%%%%%%%%%%%%%%%%%%%%%%%%%%%%%%%%%%%%%%%%
\begin{frame}
	\frametitle{Surrogate Key Pros and Cons}

\hspace{.5cm}
\begin{itemize}
	\item	Pros
	\begin{itemize}
		\item Stable.
		\item Easy to maintain.
		\item Less disk IO is required.
		\item Guaranteed to be unique.
		\item Simple key structure.
	\end{itemize}	
\end{itemize}

\begin{itemize}
	\item	Cons
	\begin{itemize}
		\item Has no business meaning (could require more joins)
		\item Extra column added (space).
		\item Extra IO when insert/update data.
	\end{itemize}	
\end{itemize}	
\end{frame}
%%%%%%%%%%%%%%%%%%%%%%%%%%%%%%%%%%%%%%%%%%%%%%%%%%%%%

\begin{frame}
\frametitle{Surrogate Key Pros and Cons}
\vspace{.5cm}
Just a few reasons for using surrogate keys for your dimension and fact tables in DWH implementation:
	\begin{itemize}[<+->]
	\item Stability: Changing a key because of a business or natural need will negatively affect related tables. Surrogate keys rarely, if ever, need to be changed because there is no meaning tied to the value.
	
	\item Convention: Allows you to have a standardized Primary Key column naming convention rather than having to think about how to join tables with various names for their PKs.
	
	\item Speed: Depending on the PK value and type, a surrogate key of an integer may be smaller, faster to index and search.
	
\end{itemize}	

\href{https://stackoverflow.com/a/63189/2516356}{https://stackoverflow.com/a/63189/2516356}

\end{frame}
%%%%%%%%%%%%%%%%%%%%%%%%%%%%%%%%%%%%%%%%%%%%%%%%%%%%%

\begin{frame}
	\frametitle{Surrogate Key AND Natural Key}
	\vspace{.5cm}
	Do we need to remove the natural key to use the surrogate key?
	\begin{itemize}[<+->]
		\item NO, we will keep both in the table and treat the surrogate key as a primary key. 
	\end{itemize}	
	
\end{frame}