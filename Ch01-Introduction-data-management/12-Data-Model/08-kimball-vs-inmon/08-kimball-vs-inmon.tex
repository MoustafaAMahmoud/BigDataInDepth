%%%%%%%%%%%%%%%%%%%%%%%%%%%%%%%%%%%%%%%%%%%%%%%%%%%%%%
%\VideoClassification[column=2, colour=blue]
%%%%%%%%%%%%%%%%%%%%%%%%%%%%%%%%%%%%%%%%%%%%%%%%%%%%%%
\midTitle{Kimball vs Inmon}
%%%%%%%%%%%%%%%%%%%%%%%%%%%%%%%%%%%%%%%%%%%%%%%%%%%%%%
\begin{frame}
	\frametitle{How to architect the data warehouse?}
	\begin{itemize}[<+->]
		\item Kimball's paradigm (Dimensional Modeling).
		\item Inmon's paradigm (Enterprise Warehouse).
	\end{itemize}
\end{frame}
%%%%%%%%%%%%%%%%%%%%%%%%%%%%%%%%%%%%%%%%%%%%%%%%%%%%%%
\begin{frame}
	\frametitle{How to architect the data warehouse? }
	\begin{itemize}[<+->]
	\item Architecture focus on the way we construct the data model.
	\item We choose the architecture based on the use case and the project.
	\item The type of business and project time frame are considered before choosing the architecture. 
\end{itemize}
	
\end{frame}
%%%%%%%%%%%%%%%%%%%%%%%%%%%%%%%%%%%%%%%%%%%%%%%%%%%%%%
%%%%%%%%%%%%%%%%%%%%%%%%%%%%%%%%%%%%%%%%%%%%%%%%%%%%%%
\begin{frame}
	\frametitle{Commons between Kimball and Inmon}
		\begin{itemize}[<+->]
		\item \textbf{Both} consider DWH as the central data repository for the enterprise.
		\item \textbf{Both} are using ETL to load DWH.
		\item \textbf{Both} are serving enterprise reporting needs.
		\item \textbf{Both} combine the DWH elements with build their model.
	\end{itemize}

\end{frame}
%%%%%%%%%%%%%%%%%%%%%%%%%%%%%%%%%%%%%%%%%%%%%%%%%%%%%%
\begin{frame}
	\frametitle{Kimball's paradigm}
	\begin{itemize}[<+->]
		\item Kimball focuses on business (organization department).
		\item It is suitable for organizations which have changes (best choice for startups or mid-organizations).
		\item It works well in the agile and quick delivery for the business.
		\item Its focus to make the user access the data easily.
	\end{itemize}
	
\end{frame}
%%%%%%%%%%%%%%%%%%%%%%%%%%%%%%%%%%%%%%%%%%%%%%%%%%%%%%

\begin{frame}
	\frametitle{Kimball's Paradigm Architecture Overview}
	\input{Ch01-Introduction-data-management/12-Data-Model/08-kimball-vs-inmon/fig_kimball_arch}
\end{frame}
%%%%%%%%%%%%%%%%%%%%%%%%%%%%%%%%%%%%%%%%%%%%%%%%%%%%%%

\begin{frame}
	\frametitle{Kimball's paradigm}
	\begin{itemize}[<+->]
		\item  The data warehouse is a set of small data marts for each department. 
		\item These data mart stored in a dimensional model (star or snowflake schema).
		\item According to this approach, the data warehouse is essentially a union of all data marts.
		\item Ralph Kimball, in 1997, stated that “...\textbf{\textit{the data warehouse is nothing more than the union of all the data marts}}“
	\end{itemize}
	
\end{frame}

%%%%%%%%%%%%%%%%%%%%%%%%%%%%%%%%%%%%%%%%%%%%%%%%%%%%%%
\begin{frame}
	\frametitle{Kimball's paradigm}
	Dimentional Model Architecture Approach:
	\begin{itemize}[<+->]
		\item After identifying a business process or metrics, It starts to design the data mart, which serves this business.
		\item Identify dimensions and facts to achieve business metrics.
		\item It is okay to have duplicates in the design.
		\item It doesn't focus on normalization.
	\end{itemize}
	
\end{frame}
%%%%%%%%%%%%%%%%%%%%%%%%%%%%%%%%%%%%%%%%%%%%%%%%%%%%%%
\begin{frame}
	\frametitle{Disadvantages of Kimball's paradigm}
	\begin{itemize}
		\item Because of Kimball's focus on individual business (data marts), This leads to the following points:
		\begin{itemize}[<+->]
			\item Lose the idea of a single source of truth because the entire data warehouse is not fully integrated.
			\item Redundant data added to the model.
			\item The model is not helping to take the strategic decisions, or it needs a too complicated process for getting the BI reports from multiple data marts. 
		\end{itemize}
	\end{itemize}
	
\end{frame}
%%%%%%%%%%%%%%%%%%%%%%%%%%%%%%%%%%%%%%%%%%%%%%%%%%%%%%
\begin{frame}
	\frametitle{Inmon's Paradigm}
	\begin{itemize}[<+->]
		\item Inmon’s paradigm, is to have "\textbf{one version of the truth}," an enterprise has all the information in one place named data warehouse, and data marts source their information from the data warehouse.
	
		\item Bill Inmon responded in 1998 by saying,  “\textbf{\textit{You can catch all the minnows in the ocean and stack them together and they still do not make a whale}}“. 
	\end{itemize}
	
\end{frame}

%%%%%%%%%%%%%%%%%%%%%%%%%%%%%%%%%%%%%%%%%%%%%%%%%%%%%%
\begin{frame}
	\frametitle{Inmon's Paradigm Architecture Overview}
	\input{Ch01-Introduction-data-management/12-Data-Model/08-kimball-vs-inmon/fig_inmon_arch}
\end{frame}
%%%%%%%%%%%%%%%%%%%%%%%%%%%%%%%%%%%%%%%%%%%%%%%%%%%%%%
\begin{frame}
	\frametitle{Inmon's paradigm}
	\begin{itemize}[<+->]
		\item Inmon’s paradigm is strategic (visionary), and has enterprise-wide reporting.
		\item It has very low data redundancy. So, it is easy for maintenance (change) and well-integrated.

	\end{itemize}
	
\end{frame}
%%%%%%%%%%%%%%%%%%%%%%%%%%%%%%%%%%%%%%%%%%%%%%%%%%%%%%
\begin{frame}
	\frametitle{Inmon's paradigm}
	\begin{itemize}[<+->]
		\item Inmon's paradigm has a significant edge and impact on enterprise especial in known industries.
		\item There are lots of mature models ready for enterprises, and most of the cases they following Inmon strategy. For example, If we plan to build a banking, telecom, or insurance data model. 
		
	\end{itemize}
	
\end{frame}
%%%%%%%%%%%%%%%%%%%%%%%%%%%%%%%%%%%%%%%%%%%%%%%%%%%%%%
\begin{frame}
	\frametitle{Disadvantages of Inmon's paradigm}
	\begin{itemize}[<+->]
		\item The initial build is costly.
		\item It is time-consuming at the beginning. 
		\item It requires highly skilled data modelling team.
		\item It requires more effort in ETL to build data marts after the creation of a data warehouse. 
	\end{itemize}
	
\end{frame}
%%%%%%%%%%%%%%%%%%%%%%%%%%%%%%%%%%%%%%%%%%%%%%%%%%%%%%
\begin{frame}
	\frametitle{Kimball vs Inmon}
\begin{table}
    \resizebox{\textwidth}{!}{
	\begin{tabular}{|c|c|c|}
		\hline
		Paradigm & Kimball & Inmon \\
		\hline
		Data integration & Focus on the individual business & enterprise-wide\\
		\hline
		Data orientation &  Business process & Subject \\
		\hline
		Approach & Bottom-Up Approach & Top-Down Approach \\
    	\hline	
		Building time & Less time &  lot of time \\
    	\hline			
		Cost & Iterative steps of effecient cost &  \makecell{High initial cost but\\ development costs are low} \\
    	\hline	
		Maintenance & difficult &  easy \\
    	\hline			
		Model & De-normalized &  Normalized \\
    	\hline			
		Modeling Complicated & generalist team to implement &  It needs specialized team \\
		\hline			
		Organization Size & Startups or Mid-Level & Enterprise \\
		\hline
	\end{tabular}
	}
\vspace{.1cm}
	\caption{Kimball vs Inmon}\label{eval_table}
\end{table}

\end{frame}

%%%%%%%%%%%%%%%%%%%%%%%%%%%%%%%%%%%%%%%%%%%%%%%%%%%%%%
\begin{frame}
	\frametitle{References}

\vspace{1.5cm}

\begin{itemize}
    	\item The Data Warehouse Toolkit: The Definitive Guide to Dimensional Modeling, 3rd Edition by Ralph Kimball. 
		\item https://www.1keydata.com/datawarehousing/inmon-kimball.html
		\item https://www.geeksforgeeks.org/difference-between-kimball-and-inmon/
		\item http://www.datamartist.com/data-warehouse-vs-data-mart
		\item https://www.astera.com/type/blog/data-warehouse-concepts/
\end{itemize}


\end{frame}

