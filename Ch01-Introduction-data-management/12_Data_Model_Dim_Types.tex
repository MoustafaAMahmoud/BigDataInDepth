%%%%%%%%%%%%%%%%%%%%%%%%%%%%%%%%%%%%%%%%%%%%%%%%%%%%%%%
%\VideoClassification[column=1, colour=blue]
%%%%%%%%%%%%%%%%%%%%%%%%%%%%%%%%%%%%%%%%%%%%%%%%%%%%%%%
%\midTitle{Dimensions Types: Conformed Dimension}
%\begin{frame}
%    \frametitle{Conformed Dimensions}
%    %https://www.guru99.com/dimensional-model-data-warehouse.html
%    %The dimension can also contain one or more hierarchical relationships
%    \begin{description}[<+->]
%        \item[Conformed Dimensions]    the dimension which is \underline{\textit{identical}} and has the \underline{\textit{same meaning}} across many fact tables which it relates and used in different areas of the warehouse.
%        \begin{example}
%            \begin{itemize}[<+->]
%                \item \underline{\textbf{(Date as a Key)}}: if we have a date column across many facts, we could use the date as key in all tables. So, it should be a unified format.
%                \item \underline{\textbf{(Product-Id as a Key)}}: if we have a product name which could vary between systems
%                \forexample (upper/lower) We can create a dimension table for the product details and use product id unified across fact tables.
%            \end{itemize}
%        \end{example}
%    \end{description}
%\end{frame}
%%%%%%%%%%%%%%%%%%%%%%%%%%%%%%%%%%%%%%%%%%%%%%%%%%%%%%%
%\VideoClassification[column=1, colour=blue]
%%%%%%%%%%%%%%%%%%%%%%%%%%%%%%%%%%%%%%%%%%%%%%%%%%%%%%%
%\midTitle{Dimensions Types: Degenerate Dimension}
%\begin{frame}
%	\frametitle{Degenerate Dimensions}
%	%https://www.guru99.com/dimensional-model-data-warehouse.html
%	\begin{itemize}[<+->]
%		\item Degenerate Dimensions
%		\begin{itemize}[<+->]
%			\item Dimension Key without corresponding dimension table.% (Not a fact and not an attribute)
%			\item Stored in fact table.
%			%This kind of dimension does not have its dimension as it is derived from the fact table.
%			\item It used to provide a grouping for business cases.
%		\end{itemize}
%	\end{itemize}
%\end{frame}
%%%%%%%%%%%%%%%%%%%%%%%%%%%%%%%%%%%%%%%%%%%%%%%%%%%%%%%
%%\VideoClassification[column=1, colour=blue]
%%%%%%%%%%%%%%%%%%%%%%%%%%%%%%%%%%%%%%%%%%%%%%%%%%%%%%%
%\begin{frame}
%	\frametitle{Degenerate Dimension}
%	%https://www.guru99.com/dimensional-model-data-warehouse.html
%	\begin{itemize}
%		\item Degenerate Dimensions
%		\begin{itemize}
%			\item Dimension Key without corresponding dimension table.% (Not a fact and not an attribute)
%			\item Stored in fact table.
%			\item It used to provide a grouping for business cases.
%			%This kind of dimension does not have its dimension as it is derived from the fact table.
%		\end{itemize}
%	\end{itemize}
%	\centering
%	\begin{tikzpicture}[every node/.style={font=\ttfamily}, node distance=1.4in,scale=.6, every node/.style={scale=0.6}]
    \matrix  [entity=OrderDetial, entity anchor=OrdersDetial-OrderID]  {
    \properties{
    OrderID,
    OrderDate (FK),
    ProductID (FK),
    Quantity,
    Amount
    }
    };
\end{tikzpicture}

%	
%	\begin{table}[t]
%		\centering
%		\sffamily
%		\begin{tabular}{|a | l | l | l | l|}
%			\hline
%			OrderID  & OrderDate & ProductID & Quantity & Amount\\
%			\hline
%			\hline
%			%\rowcolor{LightCyan}
%			123 & 123456789 & 111 & 2 & 120.45\\
%			123 & 123456789 & 222 & 5 & 10.45\\
%			%\rowcolor{myorange}
%			\hline
%			\hline
%			431 & 98765122 & 333 & 1 & 15.45\\
%			431 & 98765122 & 555 & 6 & 4.45\\
%			\hline
%		\end{tabular}
%	\end{table}
%\end{frame}
%
%%%%%%%%%%%%%%%%%%%%%%%%%%%%%%%%%%%%%%%%%%%%%%%%%%%%%%%
%\VideoClassification[column=1, colour=blue]
%%%%%%%%%%%%%%%%%%%%%%%%%%%%%%%%%%%%%%%%%%%%%%%%%%%%%%%
%\midTitle{Dimensions Types: Junk Dimension (Garbage Dimension)}
%\begin{frame}
%    \frametitle{Junk Dimension}
%    \begin{itemize}[<+->]
%        %junk assume we have multi flags we will collect them as one id
%        \item It used to reduce the number of dimensions (low-cardinality columns) in the dimensional model and reduce the number of columns in the fact table. It is a collection of random transnational codes, flags, or text attributes.
%        \item It optimizes space as fact tables should not include low-cardinality or text fields. It mainly includes measures, foreign keys, and degenerate dimension keys.
%    \end{itemize}
%\end{frame}
%%%%%%%%%%%%%%%%%%%%%%%%%%%%%%%%%%%%%%%%%%%%%%%%%%%%%%%
%\begin{frame}
%    \frametitle{Junk Dimension}
%    %\resizebox{\columnwidth}{!}{%
\begin{tikzpicture}[every node/.style={font=\ttfamily}, node distance=1.4in,scale=.6, every node/.style={scale=0.6}]
\matrix  [entity=Car, entity anchor=Car-id]  {
	\properties{
		id,
		colour-id (FK),
		body-id (FK)
	}
};


\matrix  [entity=Colour, left=of Car-id, entity anchor=Colour-id]  {
	\properties{
		id,
		colourname
	}
};
\matrix  [entity=Body, right=of Car-id,xshift=-15ex,entity anchor=Body-id]  {
	\properties{
		id,
		bodyname
	}
};
\draw (-10,1) -- (8,1) node[blue,draw=red, ultra thin, minimum size=1cm] [above,pos=0.5] {Design without junk DIM};
\draw [one to one] (Car-id)  to (Colour-id);
\draw [one to one] (Car-id)  to (Body-id);
\end{tikzpicture}
%}

%\resizebox{\columnwidth}{!}{%
	\begin{tikzpicture}[every node/.style={font=\ttfamily}, node distance=1.4in,scale=.6, every node/.style={scale=0.6}]
	
	\matrix  [entity=Car, entity anchor=Car-id]  {
		\properties{
			id,
			car-attirbute-id (FK),
		}
	};	

	\matrix  [entity=Car-Attributes, left=of Car-id, entity anchor=Car-Attributes-id]  {
		\properties{
			id,
			colourname,
			bodyname
		}
	};
	\draw (-10,1) -- (8,1) node[blue,draw=red, ultra thin, minimum size=1cm] [above,pos=0.49] {Design with junk DIM};
	\draw [one to one] (Car-id)  to (Car-Attributes-id);
	\end{tikzpicture}
%}
%%%%%%%%%%%%%%%%%%%%%%%%%%%%%%%%%%%%%%%%%%%%%%%%%%%%%%%%%%%%%%%%%%%%%%%%%%%
%%% Local Variables:
%%% mode: latex
%%% TeX-master: "../../main.tex"
% !TeX root = ../../main.tex
%%% TeX-engine: xetex
%%% End:

%\end{frame}
%%%%%%%%%%%%%%%%%%%%%%%%%%%%%%%%%%%%%%%%%%%%%%%%%%%%%%%
%\begin{frame}
%	\frametitle{Junk Dimension}
%
%	\begin{block}{Junk Dimension Table Size}
%		\begin{itemize}
%			\item We must split the Junk dimension into more dimensions in case the size grows by the time.
%			\item It is easy to calculate the expected number of rows as it is the total number of combinations between the low-cardinality attributes; \forexample 3 columns each have 3 values total = 3 * 3 = 9.
%		\end{itemize}
%	\end{block}
%	
%\end{frame}
%%%%%%%%%%%%%%%%%%%%%%%%%%%%%%%%%%%%%%%%%%%%%%%%%%%%%%%
%\VideoClassification[column=1, colour=blue]
%%%%%%%%%%%%%%%%%%%%%%%%%%%%%%%%%%%%%%%%%%%%%%%%%%%%%%%
%\midTitle{Dimensions Types: Role-Playing Dimension}
%\begin{frame}
%    \frametitle{Role-Playing Dimension}
%    \begin{description}
%        \item [Role-Playing Dimensions (Re-usable Dimension)] A single physical dimension helps to reference multiple times in a fact table as each reference linking to a logically distinct role for the dimension.
%    \end{description}
%    \centering
%    \begin{tikzpicture}[every node/.style={font=\ttfamily}, node distance=1.4in,scale=.6, every node/.style={scale=0.6}]

    \matrix  [entity=OrderDetail, entity anchor=OrderDetail-OrderID]  {
        \properties{
        OrderID,
        OrderDate (FK),
        ShippedDate (FK),
        DeliveryDate (FK),
        ExpiryDate (FK)
        }
    };
    \matrix  [entity=Calendar, right=of OrderDetail-OrderID, entity anchor=Calendar-id] {
        \properties{
        id,
        date,
        day,
        week,
        month,
        qtr,
        year
        }
    };
    %\draw [one to one] (OrderDetail-OrderID)  to (Calendar-id);
    \draw[densely dotted] (2.2,-1.9)  -- (5.9,-1.9);
    \draw[densely dotted] (2.2,-1.2)  -- (5.9,-1.2);
    \draw[densely dotted] (2.2,-.6 )  -- (5.9,-.6);
    \draw[densely dotted] (2.2,-2.5)  -- (5.9,-2.5);
\end{tikzpicture}
%%%%%%%%%%%%%%%%%%%%%%%%%%%%%%%%%%%%%%%%%%%%%%%%%%%%%%%%%%%%%%%%%%%%%%%%%%%
%%% Local Variables:
%%% mode: latex
%%% TeX-master: "../../main.tex"
% !TeX root = ../../main.tex
%%% TeX-engine: xetex
%%% End:

%\end{frame}
%%%%%%%%%%%%%%%%%%%%%%%%%%%%%%%%%%%%%%%%%%%%%%%%%%%%%%%
%\begin{frame}
%    \frametitle{Conformed vs Role-Playing Dimension}
%    \begin{block}{Conformed vs Role-Playing}
%        \begin{itemize}
%            \item \textbf{Conformed} is the same dimension used in different facts and has \textit{\underline{the same meaning}} \forexample CustomerID.
%            \item \textbf{Role-Playing} is the same dimension which used multiple times within the same fact but \textit{\underline{with different meanings}} \forexample Date.
%        \end{itemize}
%    \end{block}
%\end{frame}
%
%%%%%%%%%%%%%%%%%%%%%%%%%%%%%%%%%%%%%%%%%%%%%%%%%%%%%%%
%\VideoClassification[column=1, colour=blue]
%%%%%%%%%%%%%%%%%%%%%%%%%%%%%%%%%%%%%%%%%%%%%%%%%%%%%%%
%\midTitle{Dimensions Types: Outrigger Dimensions}
%\begin{frame}
%    \frametitle{Outrigger Dimensions}
%    \begin{itemize}[<+->]
%        \item A dimension which has a reference to another dimension table. The secondary dimension called outrigger dimension.
%        \item \blue{\textit{\faBullhorn This dimension design should be used carefully without limited cases}}.
%    \end{itemize}
%\end{frame}
%%%%%%%%%%%%%%%%%%%%%%%%%%%%%%%%%%%%%%%%%%%%%%%%%%%%%%%
%\begin{frame}
%    \frametitle{Outrigger Dimensions}
%    \centering
%    \resizebox{.9\columnwidth}{!}{%
\begin{tikzpicture}[every node/.style={font=\ttfamily}, node distance=1.4in,scale=.75, every node/.style={scale=0.75}]

    \matrix  [entity=Usage, entity anchor=Usage-id]  {
        \properties{
        id,
        cust-id (FK),
        promo-id (FK),
        usgdate-id (FK)
        }
    };

    \matrix  [entity=Promotion, below left=of Usage-id, entity anchor=Promotion-id]  {
        \properties{
        id,
        promotype,
        promodesc,
        promocategory,
        promosubcategory,
        value,
        startdate-id (FK),
        enddate-id (FK)
        }
    };
    \matrix  [entity=Calendar, above left=of Promotion-id, entity anchor=Calendar-id]  {
        \properties{
        id,
        date,
        day,
        week,
        month,
        qtr,
        year
        }
    };

    \draw [one to one] (Usage-id)  to (Promotion-id);
    \draw [one to one] (Promotion-id)  to (Calendar-id);
    \draw [one to one] (Usage-id)  to (Calendar-id);
\end{tikzpicture}
}

%\end{frame}
%
%%%%%%%%%%%%%%%%%%%%%%%%%%%%%%%%%%%%%%%%%%%%%%%%%%%%%%%
%\VideoClassification[column=1, colour=blue]
%%%%%%%%%%%%%%%%%%%%%%%%%%%%%%%%%%%%%%%%%%%%%%%%%%%%%%%
%\midTitle{Dimensions Types: Snowflake Dimensions}
%\begin{frame}
%    \frametitle{Snowflake Dimensions}
%    \begin{itemize}[<+->]
%        \item Snowflake Dimension is a dimension that has a hierarchy of attributes. This attribute is normalized, and each dimension has a relationship with another hierarchy dimension table.\\
%        \item \red{\textit{\faBug This dimension design not recommended as it has much complexity to the model and query performance. Also, it complicates the ETL process and makes too many dimensions without needs}}.
%    \end{itemize}
%\end{frame}
%%%%%%%%%%%%%%%%%%%%%%%%%%%%%%%%%%%%%%%%%%%%%%%%%%%%%%%
%\begin{frame}
%    \frametitle{Snowflake Dimensions}
%    \begin{itemize}
%        \item Snowflake Dimension is a dimension that has a hierarchy of attributes. This attribute is normalized, and each dimension has a relationship with another hierarchy dimension table.\\
%		\item \red{\textit{\faBug This dimension design not recommended as it has much complexity to the model and query performance. Also, it complicates the ETL process and makes too many dimensions without needs}}.
%    \end{itemize}
%    \centering
%    \resizebox{.9\columnwidth}{!}{%
\begin{tikzpicture}[every node/.style={font=\ttfamily}, node distance=1.4in,scale=.75, every node/.style={scale=0.75}]
    \matrix  [entity=Usage, entity anchor=Usage-id]  {
        \properties{
        id,
        cust-id (FK),
        promo-id (FK),
        usgdate-id (FK)
        }
    };
    \matrix  [entity=Promotion, below left=of Usage-id,yshift=10ex, entity anchor=Promotion-id]  {
        \properties{
        id,
        promotype,
        promodesc,
        subcategory-id (FK),
        value
        }
    };
    \matrix  [entity=PromoSubCategory, below left=of Promotion-id,yshift=10ex, entity anchor=PromoSubCategory-id]  {
        \properties{
        id,
        subcategory,
        category-id (FK),
        }
    };
    \matrix  [entity=PromoCategory, below left=of PromoSubCategory-id,yshift=10ex, entity anchor=PromoCategory-id]  {
        \properties{
        id,
        category,
        }
    };


    \draw [one to one] (Usage-id)  to (Promotion-id);
    \draw [one to one] (Promotion-id)  to (PromoSubCategory-id);
    \draw [one to one] (PromoSubCategory-id) to (PromoCategory-id);
\end{tikzpicture}
}

%%%%%%%%%%%%%%%%%%%%%%%%%%%%%%%%%%%%%%%%%%%%%%%%%%%%%%%%%%%%%%%%%%%%%%%%%%%
%%% Local Variables:
%%% mode: latex
%%% TeX-master: "../../main.tex"
% !TeX root = ../../main.tex
%%% TeX-engine: xetex
%%% End:

%\end{frame}
%%%%%%%%%%%%%%%%%%%%%%%%%%%%%%%%%%%%%%%%%%%%%%%%%%%%%%%
%\VideoClassification[column=1, colour=red]
%%%%%%%%%%%%%%%%%%%%%%%%%%%%%%%%%%%%%%%%%%%%%%%%%%%%%%%
%\midTitle{Dimensions Types: Slowly changing Dimensions}
%\begin{frame}
%    \frametitle{Slowly changing Dimensions}
%    %https://www.guru99.com/dimensional-model-data-warehouse.html
%    \begin{itemize}[<+->]
%		\item It the dimension which changes over time. So, for a specific date we have different value.
%		\item It has different types as following
%		    \begin{itemize}[<+->]
%        \item Type 0 (Fixed Dimension): We don't change the current even the source changes. 
%        \item Type 1 (No History): No history is maintained only the latest replace the current.
%        \item Type 2 (History): Series of history of records are maintained.
%        \item Type 3 (Hybrid): Only the last Change and the Current new change is stored
%        \item Type 4 : We split the data into two tables, first the current record and second is the historical (most common usage).
%    \end{itemize}   
%
%
%        %slowly changing dim assume customer was located in dubai then he changed his location
%        % We don't care about the change. just update the latest
%        % we can have two version new, old.
%        % we can have sergeate key.
%        %with start and end data.
%        %start and end is null
%        %join with sergate key
%    \end{itemize}
%\end{frame}
%%%%%%%%%%%%%%%%%%%%%%%%%%%%%%%%%%%%%%%%%%%%%%%%%%%%%%%
%\begin{frame}
%	\frametitle{Slowly changing Dimensions}
%\begin{block}{Note}
%	\textit{There are some other types which is a combination between the above similar than type 3 combined between 1 \& 2. \\ You can check the chapter resources for more information about the other types.}
%\end{block}
%%https://www.kimballgroup.com/2013/02/design-tip-152-slowly-changing-dimension-types-0-4-5-6-7/     
%\end{frame}
%%%%%%%%%%%%%%%%%%%%%%%%%%%%%%%%%%%%%%%%%%%%%%%%%%%%%%%
%\begin{frame}
%	\frametitle{Slowly changing Dimensions}
%	
%	\begin{itemize}
%		\item Type 0.
%	\end{itemize}
%	\begin{table}[t]
%		\centering
%		\sffamily
%		\begin{tabular}{|l | l | a |}
%			\hline
%			CustomerID & Name & City\\
%			\hline
%			\hline			
%			123456789 & Ronaldo  & Madrid\\
%			\hline
%		\end{tabular}
%		\quad
%		\begin{tabular}{|l | l| a|}
%			\hline
%			CustomerID & Name & City\\
%			\hline
%			\hline			
%			123456789 & Ronaldo  & Turin\\
%			\hline
%		\end{tabular}
%		\caption{Source System Old vs New}
%	\end{table}
%	
%	\begin{table}[t]
%		\centering
%		\sffamily
%		\begin{tabular}{|l | l | l |a|}
%			\hline
%			ID & CustomerID & Name & City\\
%			\hline
%			\hline		
%			1 & 123456789 & Ronaldo  & Madrid\\
%			\hline
%		\end{tabular}
%		\caption{Customer Profile Dimension}
%	\end{table}
%		
%\end{frame}
%%%%%%%%%%%%%%%%%%%%%%%%%%%%%%%%%%%%%%%%%%%%%%%%%%%%%%%
%\begin{frame}
%	\frametitle{Slowly changing Dimensions}
%	
%	\begin{itemize}
%		\item Type 1.
%	\end{itemize}
%	\begin{table}[t]
%		\centering
%		\sffamily
%		\begin{tabular}{|l | l | a |}
%			\hline
%			CustomerID & Name & City\\
%			\hline
%			\hline			
%			123456789 & Ronaldo  & Madrid\\
%			\hline
%		\end{tabular}
%		\quad
%		\begin{tabular}{|l | l| a|}
%			\hline
%			CustomerID & Name & City\\
%			\hline
%			\hline			
%			123456789 & Ronaldo  & Turin\\
%			\hline
%		\end{tabular}
%		\caption{Source System Old vs New}
%	\end{table}
%	
%	\begin{table}[t]
%		\centering
%		\sffamily
%		\begin{tabular}{|l | l | l |a|}
%			\hline
%			ID & CustomerID & Name & City\\
%			\hline
%			\hline		
%			1 & 123456789 & Ronaldo  & Turin\\
%			\hline
%		\end{tabular}
%		\caption{Customer Profile Dimension}
%	\end{table}	
%\end{frame}
%%%%%%%%%%%%%%%%%%%%%%%%%%%%%%%%%%%%%%%%%%%%%%%%%%%%%%%
%\begin{frame}
%	\frametitle{Slowly changing Dimensions}
%	\begin{itemize}
%		\item Type 2.
%	\end{itemize}
%	\begin{table}[t]
%		\centering
%		\sffamily
%		\begin{tabular}{|l | l | a |l|}
%			\hline
%			CustomerID & Name & City & UpdatedDt \\
%			\hline
%			\hline			
%			123456789 & Ronaldo  & Madrid & 2018-12-12\\
%			\hline
%			\hline			
%			123456789 & Ronaldo  & Turin & 2019-06-12\\
%			\hline
%			\hline			
%			123456789 & Ronaldo  & London & 2019-08-12\\
%			\hline
%			\hline						
%			123456789 & Ronaldo  & Porto & 2019-12-12\\		
%			\hline
%		\end{tabular}	
%		\caption{Source System Old vs New}
%	\end{table}%
%	\vspace{-.8cm}
%
%	\begin{table}[t]
%		\centering
%		\sffamily
%		  \begin{adjustbox}{max width=\textwidth}			
%		\begin{tabular}{|l | l | l | a | l | l |a|}
%			\hline
%			ID & CustomerID & Name & City & effectiveDt & TerminationDt & isCurrent\\
%			\hline
%			\hline		
%			1 & 123456789 & Ronaldo  & Madrid & 2018-12-12 & 2019-06-12 & false\\
%			2 & 123456789 & Ronaldo  & Turin & 2019-06-12 & 2019-08-12 & false\\
%			3 & 123456789 & Ronaldo  & London & 2019-08-12 & 2019-12-12 & false\\
%			4 & 123456789 & Ronaldo  & Porto  & 2019-12-12 & null & true\\
%			\hline
%		\end{tabular}
%		\end{adjustbox}
%
%		\caption{Customer Profile Dimension {\scriptsize We can replace null with a finite date (9999-12-31) but it needs to be consistent}}
%	\end{table}
%	
%\end{frame}
%%%%%%%%%%%%%%%%%%%%%%%%%%%%%%%%%%%%%%%%%%%%%%%%%%%%%%%
%\begin{frame}
%	\frametitle{Slowly changing Dimensions}
%	\begin{itemize}
%		\item Type 3.
%	\end{itemize}
%	\begin{table}[t]
%		\centering
%		\sffamily
%		\begin{tabular}{|l | l | a |l|}
%			\hline
%			CustomerID & Name & City & UpdatedDt \\
%			\hline
%			\hline			
%			123456789 & Ronaldo  & Madrid & 2018-12-12\\
%			\hline
%			\hline			
%			123456789 & Ronaldo  & Turin & 2019-06-12\\
%			\hline
%			\hline			
%			123456789 & Ronaldo  & London & 2019-08-12\\
%			\hline
%			\hline						
%			123456789 & Ronaldo  & Porto & 2019-12-12\\		
%			\hline
%		\end{tabular}	
%		\caption{Source System Old vs New}
%		
%	\end{table}
%	
%	\begin{table}[t]
%		\centering
%		\sffamily
%		\begin{tabular}{|l | l | l | a | l | l |}
%			\hline
%			ID & CustomerID & Name & City & UpdatedDate  & previousCity\\
%			\hline
%			\hline		
%			1 & 123456789 & Ronaldo  & Porto  & 2019-12-12 & London\\
%			\hline
%		\end{tabular}
%		\caption{Customer Profile Dimension}
%	\end{table}
%\end{frame}
%
%%%%%%%%%%%%%%%%%%%%%%%%%%%%%%%%%%%%%%%%%%%%%%%%%%%%%%%
%\begin{frame}
%	\frametitle{Slowly changing Dimensions}
%	\begin{itemize}
%		\item Type 4 (Split current and Historical).
%	\end{itemize}
%
%	\begin{table}[t]
%	\centering
%	\sffamily
%	\begin{tabular}{|l | l | l | a | l | l |a|}
%		\hline
%		ID & CustomerID & Name & City & effectiveDt & TerminationDt\\
%		\hline
%		\hline		
%		1 & 123456789 & Ronaldo  & Madrid & 2018-12-12 & 2019-06-12\\
%		2 & 123456789 & Ronaldo  & Turin & 2019-06-12 & 2019-08-12\\
%		3 & 123456789 & Ronaldo  & London & 2019-08-12 & 2019-12-12\\
%		4 & 123456789 & Ronaldo  & Porto  & 2019-12-12 & null\\
%		\hline
%	\end{tabular}
%	\caption{Customer Profile Dimension Hist}
%\end{table}
%
%	
%	\begin{table}[t]
%		\centering
%		\sffamily
%		\begin{tabular}{|l | l | l | a | l |}
%			\hline
%			ID & CustomerID & Name & City & UpdatedDate\\
%			\hline
%			\hline		
%			1 & 123456789 & Ronaldo  & Porto  & 2019-12-12\\
%			\hline
%		\end{tabular}
%		\caption{Customer Profile Dimension}
%	\end{table}
%
%\end{frame}
%
%%%%%%%%%%%%%%%%%%%%%%%%%%%%%%%%%%%%%%%%%%%%%%%%%%%%%%%
%\begin{frame}[fragile]
%	\frametitle{Slowly changing Dimensions}
%	\begin{itemize}
%		\item How does the Facts join SCD? We have two scenarios as following:
%		\begin{itemize}
%			\item Getting the current customer information (Join with the latest).
%			\item Getting the historical customer information (Join with the historical table based on \textbf{\textit{cust id \& date}}).
%		\end{itemize}
%	\end{itemize}
%
%	\begin{table}[t]
%		\centering
%		\sffamily
%		\begin{tabular}{|l | a | l | l |}
%			\hline
%			ID & CustomerID & TotalCalls & CallDate \\
%			\hline
%			\hline		
%			1 & 123456789 & 30 & 2018-12-12 \\
%			2 & 123456789 & 30 & 2019-12-12 \\
%			\hline
%		\end{tabular}
%		\caption{Customer Usage}
%	\end{table}
%
%	
%\end{frame}
%%%%%%%%%%%%%%%%%%%%%%%%%%%%%%%%%%%%%%%%%%%%%%%%%%%%%%%
%
%%%%%%%%%%%%%%%%%%%%%%%%%%%%%%%%%%%%%%%%%%%%%%%%%%%%%%%%
%\begin{frame}[fragile]
%	\frametitle{Slowly changing Dimensions}
%	
%	\lstinputlisting[language=sql,caption={Example to show how to use SCD}]{./Ch01-Introduction-data-management/Code/SCD_Examply.sql}
%	
%\end{frame}
%
%%%%%%%%%%%%%%%%%%%%%%%%%%%%%%%%%%%%%%%%%%%%%%%%%%%%%%
%\VideoClassification[column=1, colour=blue]
%%%%%%%%%%%%%%%%%%%%%%%%%%%%%%%%%%%%%%%%%%%%%%%%%%%%%%
%\midTitle{Dimensions Types: Fast (Rapidly) Changing Dimension (Mini Dimension) }
%\begin{frame}
%	\frametitle{Fast Changing Dimension (Mini Dimension)}
%	\begin{itemize}[<+->]
%		\item When we have a dimension with one or more of its attributes changing very fast. 
%		
%		\item \blue{\textit{\faBullhorn \space It causes a performance issue if we tried to handle this case similar SCD Type 2 because of the rapidly changing  in this dimension and the table will includes a lot of rows for this dimension}}.
%
%		\item We solve this case by separation the attributes into one or more dimensions. This technique also called \textit{\textbf{mini-dimensions}}.		
%	
%	\end{itemize}
%\end{frame}
%%%%%%%%%%%%%%%%%%%%%%%%%%%%%%%%%%%%%%%%%%%%%%%%%%%%%%
%\begin{frame}
%	\frametitle{Fast Changing Dimension (Mini Dimension)}
%	\begin{itemize}[<+->]
%		\item How to implement FCD (Mini Dimension)? \underline{{\footnotesize \textit{Hint: Search for the mini-dimension  relation table.}}}
%	\end{itemize}
%
%\begin{table}
%	\begin{adjustbox}{max width=.95\textwidth}			
%		\begin{tabular}{| l | l | l | l | a | a | l|}
%			\hline
%			Patient\_id & Name & Gender & BirthDate & Weight & B\_Presaure & UpdateDt\\
%			\hline
%			\hline		
%			123 & Anna   & F & 1968-01-12 & 50 & 110.0 &2019-01-01\\
%			123 & Anna   & F & 1968-01-12 & 55 & 130.0 &2019-01-07\\
%			123 & Anna   & F & 1968-01-12 & 59 & 115.0 &2019-01-14\\
%			123 & Anna   & F & 1968-01-12 & 65 & 120.0 &2019-01-21\\
%			\hline
%		\end{tabular}
%	\end{adjustbox}
%	\caption{Patient Profile Dimension}
%\end{table}
%
%\begin{table}
%	\resizebox{.97\columnwidth}{!}{%
%		
%		\begin{tabular}{| l | l | l | l |}
%			\hline
%			Patient\_id & Name & Gender & BirthDate \\
%			\hline 
%			\hline		
%			123 & Anna   & F & 1968-01-12 \\
%			\hline
%		\end{tabular}
%
%\quad
%
%		\begin{tabular}{| l | l | l | l |}
%			\hline
%			Patient\_Key & Weight & B\_Presure \\
%			\hline
%			\hline		
%			1 & 50 & 110.0\\
%			2 & 55 & 130.0\\
%			3 & 59 & 115.0\\
%			4 & 65 & 120.0\\
%			\hline
%		\end{tabular}
%	}
%	\caption{Patient Profile Dimension After Removing FCD and Split it into Junk-Dimension table}
%\end{table}
%
%\end{frame}
%
%%%%%%%%%%%%%%%%%%%%%%%%%%%%%%%%%%%%%%%%%%%%%%%%%%%%%%
%\begin{frame}
%	\frametitle{Fast Changing Dimension (Mini Dimension)}
%
%\begin{table}
%	\begin{adjustbox}{max width=.7\textwidth}
%		\begin{tabular}{| l | l | l | l |}
%			\hline
%			Patient\_id & Patient\_Key & Start\_Date & End\_Date\\
%			\hline
%			\hline		
%			123 & 1   & 2019-01-01 & 2019-01-07\\
%			123 & 2   & 2019-01-07 & 2019-01-14\\
%			123 & 3   & 2019-01-14 & 2019-01-21\\
%			123 & 4   & 2019-01-21 & null\\
%			\hline
%		\end{tabular}
%	\end{adjustbox}
%	\caption{Patient Mini Dimension}
%\end{table}
%\begin{tikzpicture}[every node/.style={font=\ttfamily}, node distance=1.4in,scale=.7, every node/.style={scale=0.7}]
  
    \matrix  [entity=Patient, entity anchor=Patient-ID]  {
	    \properties{
	    ID,
	    Name,
	    Gender,
	    BirthDate
	    }
    };
    \matrix  [entity=PatientRelationDim, left=of Patient-ID,xshift=9ex,entity anchor=PatientRelationDim-PatientID]  {
		\properties{
			PatientID,
			PatientKey,
			StartDate,
			EndDate
		}
	};
%
	\matrix  [entity=PatientMeasureDetails, left=of PatientRelationDim-PatientID,xshift=6ex, entity anchor=PatientMeasureDetails-PatientKey]  {
		\properties{
			PatientKey,
			Weight,
			BPresure
		}
	};

\draw [one to one] (Patient-ID)  to (PatientRelationDim-PatientID);
\draw [one to one] (PatientRelationDim-PatientID)  to (PatientMeasureDetails-PatientKey);

\end{tikzpicture}

%%%%%%%%%%%%%%%%%%%%%%%%%%%%%%%%%%%%%%%%%%%%%%%%%%%%%%%%%%%%%%%%%%%%%%%%%%%
%%% Local Variables:
%%% mode: latex
%%% TeX-master: "../../main.tex"
% !TeX root = ../../main.tex
%%% TeX-engine: xetex
%%% End:

%\end{frame}
%%%%%%%%%%%%%%%%%%%%%%%%%%%%%%%%%%%%%%%%%%%%%%%%%%%%%%%%
%\VideoClassification[column=1, colour=blue]
%%%%%%%%%%%%%%%%%%%%%%%%%%%%%%%%%%%%%%%%%%%%%%%%%%%%%%%
%\midTitle{Dimensions Types: Shrunken Rollup Dimensions}
%\begin{frame}
%    \frametitle{Shrunken Rollup Dimensions}%منكمش
%    \begin{itemize}[<+->]    	
%		\item Shrunken Rollup dimension is used for developing aggregate (higher level of summary) fact tables. 
%		\item It required that the data model has a lower level of granularity.
%	\end{itemize}
%		\begin{example}
%		    \begin{itemize}[<+->]    	
%			\item We have a daily usage fact table, and we need to have a higher level of monthly usage. So, we use the monthly dimension to get a summary of the daily.
%			\item We have a daily usage fact table aggregated on area-id, and we need to create another summary table aggregated based on city id. So, the new grain level here is the new dimension for the city.
%			\end{itemize}
%		\end{example}
%\end{frame}
%%%%%%%%%%%%%%%%%%%%%%%%%%%%%%%%%%%%%%%%%%%%%%%%%%%%%%%
%\begin{frame}
%	\frametitle{Shrunken Rollup Dimensions Cont.}
%
%	\begin{table}[t]
%		\centering
%		\sffamily
%		\begin{tabular}{|l | a | l |}
%			\hline
%			OrderDate & AreaID & TotalOrders\\
%			\hline
%			\hline
%			123456789 & 123  & 20\\
%			123456789 & 123  & 30\\
%			\hline
%			\hline
%			123456789 & 678  & 10\\
%			123456789 & 678  & 12\\
%			\hline
%		\end{tabular}
%
%	\end{table}
%	
%	\begin{table}[t]
%		\centering
%		\sffamily
%		\begin{tabular}{|a | l | l |}
%			\hline
%			AreaID & AreaName & CityID\\
%			\hline
%			\hline			
%			123 & Al-Matareya  & 1\\
%			678 & Ain shams    & 1\\
%			\hline
%		\end{tabular}
%		\quad
%		\begin{tabular}{|a | l|}
%			\hline
%			CityID & CityName\\
%			\hline
%			\hline			
%			1 & Cairo \\
%			\hline
%		\end{tabular}
%	\end{table}
%
%	\begin{table}[t]
%	\centering
%	\sffamily
%		\begin{tabular}{|l | a | l |}
%			\hline
%			OrderDate & CityID & TotalOrders\\
%			\hline
%			\hline		
%			123456789 & 1  & 72\\
%			\hline
%		\end{tabular}
%	\end{table}
%\end{frame}
%%%%%%%%%%%%%%%%%%%%%%%%%%%%%%%%%%%%%%%%%%%%%%%%%%%%%%
\VideoClassification[column=1, colour=blue]
%%%%%%%%%%%%%%%%%%%%%%%%%%%%%%%%%%%%%%%%%%%%%%%%%%%%%%
\midTitle{Dimensions Types: Multi-valued dimensions (Many-To-Many Dimension)}
\begin{frame}
	\frametitle{Multi-valued dimensions}
	\begin{itemize}[<+->]
		\item When the relationships between the dimension member and the fact are many to many which means the dimension members are lower granularity than the facts. 
		\item Fact table should contains one-to-one relationship with the dimension. So, we introduce the \textbf{\textit{Bridge table}} when we need to related multiple dimensions values with one record.
	\end{itemize}
	
	\begin{example}
		\begin{itemize}[<+->]
			\item Patients can have multiple diagnoses.
			\item Students can have multiple majors.
			\item customers can have multiple account.
			\item Authors can have multiple publications.
		\end{itemize}
	\end{example}		
	
\end{frame}
%%%%%%%%%%%%%%%%%%%%%%%%%%%%%%%%%%%%%%%%%%%%%%%%%%%%%%
\begin{frame}
	\frametitle{Multi-valued dimensions}
	\begin{example}[Sales of Articles]
		\begin{itemize}[<+->]
			\item Assume we need to report the sales of article and we have some articles has more than one author.
			\item According to the report we need to check each author and associate with the articles they have authored. How can we model this case?
		\end{itemize}
	\end{example}
	\begin{tikzpicture}[every node/.style={font=\ttfamily}, node distance=1.4in,scale=.7, every node/.style={scale=0.7}]
  
    \matrix  [entity=Author, entity anchor=Author-ID]  {
	    \properties{
	    ID,
	    Name,
	    Email,
	    Bio
	    }
    };
    \matrix  [entity=AuthorGroupRelation, left=of Author-ID,xshift=9ex,entity anchor=AuthorGroupRelation-ID]  {
		\properties{
			ID,
			AuthorKey,
			WieghtingFactor
		}
	};
%
	\matrix  [entity=AuthorGroup, left=of AuthorGroupRelation-ID,xshift=6ex, entity anchor=AuthorGroup-ID]  {
		\properties{
			ID
		}
	};

	\matrix  [entity=ArticleSales, below right =of AuthorGroup-ID,yshift=7ex,xshift=-6ex, entity anchor=ArticleSales-ID]  {
	\properties{
		ID,
		AuthorGroupID,
		SalesDt,
		Quantity,
		UnitPrice
	}
	};

	\matrix  [entity=Article,  right =of ArticleSales-ID, entity anchor=Article-ID]  {
	\properties{
		ID,
		Title,
		Journal,
		Volume,
		Price
	}
};


\draw [one to many] (Author-ID)  to (AuthorGroupRelation-ID);
\draw [many to one] (AuthorGroupRelation-ID) to (AuthorGroup-ID);
\draw [many to one] (ArticleSales-ID)  to (AuthorGroup-ID);
\draw [many to one] (ArticleSales-ID)  to (Article-ID);
\end{tikzpicture}

%%%%%%%%%%%%%%%%%%%%%%%%%%%%%%%%%%%%%%%%%%%%%%%%%%%%%%%%%%%%%%%%%%%%%%%%%%%
%%% Local Variables:
%%% mode: latex
%%% TeX-master: "../../main.tex"
% !TeX root = ../../main.tex
%%% TeX-engine: xetex
%%% End:

\end{frame}

%multi-valued attributes is to create a relationship between the dimension table and a secondary dimensional table (outrigger table).
%
\textbf{}
%%%%%%%%%%%%%%%%%%%%%%%%%%%%%%%%%%%%%%%%%%%%%%%%%%%%%%
\VideoClassification[column=1, colour=blue]
%%%%%%%%%%%%%%%%%%%%%%%%%%%%%%%%%%%%%%%%%%%%%%%%%%%%%%
\midTitle{Dimensions Types: Heterogeneous Dimensions}
\begin{frame}
\frametitle{Heterogeneous Dimensions}
\begin{itemize}[<+->]
	
	\item This type works when we have a case that a company selling different product to the same base of customer. Every product has it different attributes. 
	\item One famous example of this type assume an insurance company has two types of product like health and car. In this case Car insurance has different attributes than the health insurance.
	\item If we tried to model this two different products this type name Heterogeneous dimensions. 
\end{itemize}
\end{frame}
%%%%%%%%%%%%%%%%%%%%%%%%%%%%%%%%%%%%%%%%%%%%%%%%%%%%%%
\begin{frame}
	\frametitle{Heterogeneous Dimensions}
	\begin{itemize}[<+->]		
		\item There are different scenario to implement this type
		\begin{description}
			\item [Separate Dimensions] Split each one in separate dimensions and facts. It will be less data and business will do this analysis from two separate facts.
			\item [Merge Attributes] We will merge all the attributes in one single table and we will add the common attributes and null for un related attributes. Implementing this scenarios when we have less different of attributes. However, this implementation is not recommended because of the table size, performance, and maintenance.
			\item [Generic Design] In this approach we will create a single fact table and single dimension with the common attributes. The problem of this design we will report or care about the common attributes only.
		\end{description}		
\end{itemize}
\end{frame}
%%%%%%%%%%%%%%%%%%%%%%%%%%%%%%%%%%%%%%%%%%%%%%%%%%%%%%