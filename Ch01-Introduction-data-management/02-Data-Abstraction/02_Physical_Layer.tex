%! Author = moustafa
%! Date = 26/04/2020
%---------------------------------------------------------
\VideoClassification
%%%%%%%%%%%%%%%%%%%%%%%%%%%%%%%%%%%%%%%%%%%%%%%%%%%%%%
\begin{frame}
    \frametitle{Physical level}
    \begin{itemize}[<+->]
        \item \textbf{Physical level (Internal)}:
        \begin{itemize}[<+->]
            \item Lowest level.
            \item Describes \textbf{\underline{\blue{how}}} data is stored.
            \item Describes the data structure.
            \item It allows you to modify the lowest level (Physical part) without any change in the logical schema. These change could be
            \begin{itemize}[<+->]
                \item Using a new storage device
                \item Change the structure of the data used for storage
                \item Change the file type or use a different storage structure
                \item Chang the access method
                \item Modify indexes
                \item Change the compression algorithm or hashing technique.
            \end{itemize}
        \end{itemize}
    \end{itemize}
\end{frame}
%%%%%%%%%%%%%%%%%%%%%%%%%%%%%%%%%%%%%%%%%%%%%%%%%%%%%%
\begin{frame}
    \frametitle{Physical level}
    \begin{example}
        \begin{itemize}[<+->]
            \item Database contains product information.
            \item Physical layer describes
            \begin{itemize}[<+->]
                \item Storage mechanism and the blocks (bytes, gigabytes, terabytes, etc.).
                \item The amount of memory used.
                \item Usually this layer abstracted from the programmers.
            \end{itemize}
        \end{itemize}
    \end{example}

\end{frame}
%---------------------------------------------------------


%%%%%%%%%%%%%%%%%%%%%%%%%%%%%%%%%%%%%%%%%%%%%%%%%%%%%%%%%%%%%%%%%%%%%%%%%%%%
%%% Local Variables:
%%% mode: latex
%%% TeX-master: "../main"
% !TeX root = ../main.tex
%%% TeX-engine: xetex
%%% End:
