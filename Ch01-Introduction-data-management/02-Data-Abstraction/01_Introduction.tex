%! Author = moustafa
%! Date = 26/04/2020

%---------------------------------------------------------
\VideoClassification[column=2, colour=blue]
%%%%%%%%%%%%%%%%%%%%%%%%%%%%%%%%%%%%%%%%%%%%%%%%%%%%%%
\subsection{Data Abstraction}
\begin{frame}
    \frametitle{Motivation to Data Layers (Use Case)}
        \begin{figure}[H]
    	\smartdiagramset{
    		%descriptive items y sep = 3em,
    		description font = \scriptsize\sffamily,
    		description title font=\scriptsize\sffamily,
    	}
%	\centering
	\begin{subfigure}[t]{0.475\textwidth}
%		\centering
		\scalebox{0.5}{
		\smartdiagram[descriptive diagram]{
			{App, Application UI},
			{FS, CSV Data}}
		}
		\vspace{-.6\baselineskip}
		\caption{{\tiny Two layers Arch. (Data \& UI)}}
		\label{fig:ch_1_data_abstraction_1}
	\end{subfigure}
	\hfill
	\begin{subfigure}[t]{0.475\textwidth}
%		\centering 
		\scalebox{0.5}{
			\smartdiagram[descriptive diagram]{
				{App, Application UI},
				{BL, CSV Data Loader (Reporting)},
				{FS, CSV Data}}
			}
		\vspace{-.6\baselineskip}
		\caption{{\tiny Three layers Arch. (Data \& BL \& UI)}}
		\label{fig:ch_1_data_abstraction_2}
	\end{subfigure}
%	\vskip\baselineskip
	\begin{subfigure}[t]{0.475\textwidth}   
%		\centering 
		\scalebox{0.5}
		{
			\smartdiagram[descriptive diagram]{
			{App, Application UI},
			{BL, (JSON/CSV) Data Loader (Reporting)},
			{FS, JSON Data}}
		}
		\vspace{-.6\baselineskip}
		\caption{{\tiny Three layers Arch. (Data (multi-sources) \& BL \& UI)}}
		\label{fig:ch_1_data_abstraction_3}
	\end{subfigure}
	\quad
	\begin{subfigure}[t]{0.475\textwidth}   
%		\centering 
		\scalebox{0.5}
		{
			\smartdiagram[descriptive diagram]{
				{App, Application UI},
				{DBMS-H, Ready prepared layer for each department (Reporting)},
				{DBMS-M, Logical part to prepare for the data structure and the relation between the data},
				{DBMS-L, Storage and Data format related stuff + Data indexing and searching algorithms}}
		}
		\vspace{-.6\baselineskip}
		\caption{{\tiny Four layers Arch. (DB (L, M, H) \& UI)}}
		\label{fig:ch_1_data_abstraction_4}
	\end{subfigure}
	\vspace{-.6\baselineskip}
	\caption {\tiny Data Abstraction Journey} 
	\label{fig:ch_1_data_abstraction}
	\end{figure}
	


\end{frame}
%%%%%%%%%%%%%%%%%%%%%%%%%%%%%%%%%%%%%%%%%%%%%%%%%%%%%%
\begin{frame}
    \frametitle{Motivation to Data Layers (Solution Thinking)}

    \begin{itemize}[<+->]
        \item How can we think about a data solution or challenges in the data products?
        \begin{itemize}[<+->]
            \item Requirements analysis.
            \item Identify the problem (challenges).
            \item Think about how to overcome the challenges.
            \item Ask your self the following questions:
            \begin{itemize}[<+->]
                \item Can we solve the problem using the current data structure by adding new features?
                \item What if we enhance/change the data structure or modeling?
                \item Could it help if we change the backend engine \forexample (DBMS system)?
            \end{itemize}
        \end{itemize}
        \item To answer these questions you need to understand the \textbf{\underline{data layers}}.
    \end{itemize}

\end{frame}
%%%%%%%%%%%%%%%%%%%%%%%%%%%%%%%%%%%%%%%%%%%%%%%%%%%%%%
\begin{frame}
    \frametitle{Data Layers (Abstraction)}
    \begin{itemize}[<+->]
        \item Any data product (database) contains multi-layers.
        \item Each layer responsible for different tasks and operations.
        \item Each layer interacts with (hardware or software or mixed).
        \item Eliminate the complexity of data interactions; not all internal processes are shared or available for the user.
        \item The developer for each layer hides irrelevant internal details from the developer (users).
        \item The process of \textbf{\underline{\blue{hiding}}} irrelevant details from the developer (user) is called data \textbf{\underline{\blue{abstraction}}}.
    \end{itemize}
\end{frame}
%%%%%%%%%%%%%%%%%%%%%%%%%%%%%%%%%%%%%%%%%%%%%%%%%%%%%%
\begin{frame}
    \frametitle{Data Layers (Abstraction)}
    \begin{definition}
        \textbf{Data Abstraction and Data Independence}: DBMS comprises complex data-structures. To make the system efficient in terms of retrieval of data and reduce complexity in terms of usability of users, developers use abstraction i.e., hide irrelevant details from the users. This approach simplifies database design.

    \end{definition}
    %Capacity of changing in one level without affecting the other levels. Copied but forget from where!!!
    \begin{itemize}[<+->]
        \item There are 3 levels of data abstraction.
        \begin{itemize}[<+->]
            \item Physical Level
            \item Logical/Conceptual Level.
            \item View Level.
        \end{itemize}
    \end{itemize}
    %TOP TIER, MIDDLE TIER, BOTTOM TIER
\end{frame}
%%%%%%%%%%%%%%%%%%%%%%%%%%%%%%%%%%%%%%%%%%%%%%%%%%%%%%
\begin{frame}
    \frametitle{Data Layers (Abstraction)}
    	\scalebox{0.9}{
\begin{tikzpicture}[node distance=2cm,
					every node/.style={fill=white, font=\sffamily}, align=center,
					scale=0.6, 
					every node/.style={transform shape}]

% Specification of nodes (position, etc.)
\node (view2)             [optionalETL]              {report 2 (view)};
\node (view1)     [optionalETL, right of=view2, xshift=3cm]          {report 1 (view)};
\node (view3)      [optionalETL, right of=view1, xshift=3cm]   {report 3 (view)};
\node (concept)     [required,below of=view1, yshift=-1.5cm]   {Conceptual Layer};
\node (physical)      [optionalELK, below of=concept, yshift=-1.5cm] {Physical Interaction};
\node (fs)      [optionalELK, below of=physical, yshift=-1cm] {FS};

%\node (Appendix) [startstop, above of=Arch] {Ch.13 Appendix};     
% Normal Path
\draw[<->]     (view2) -- (concept);
\draw[<->]     (view3) -- (concept);
\draw[<->]     (view1) -- (concept);
\draw[<->]     (concept) -- (physical);
\draw[<->]     (fs) -- (physical);
\draw[-]      (12,-1.5) to[out=0,in=180] (13,0)  node[right]{View Level (User View) } to[out=180,in=0] (12,1.5);
\draw[-]      (12,-5) to[out=0,in=180] (13,-3.5) node[right]{Logical/ Conceptual Level } to[out=180,in=0]  (12,-2);
\draw[-]      (12,-11) to[out=0,in=180] (13,-8.5) node[right]{Physical Level } to[out=180,in=0]  (12,-6);
\end{tikzpicture}
}

%%%%%%%%%%%%%%%%%%%%%%%%%%%%%%%%%%%%%%%%%%%%%%%%%%%%%%%%%%%%%%%%%%%%%%%%%%%
%%% Local Variables:
%%% mode: latex
%%% TeX-master: "../../main.tex"
% !TeX root = ../../main.tex
%%% TeX-engine: xetex
%%% End:

\end{frame}
%%%%%%%%%%%%%%%%%%%%%%%%%%%%%%%%%%%%%%%%%%%%%%%%%%%%%%%%%%%%%%%%%%%%%%%%%%%%
%%% Local Variables:
%%% mode: latex
%%% TeX-master: "../main"
% !TeX root = ../main.tex
%%% TeX-engine: xetex
%%% End:
