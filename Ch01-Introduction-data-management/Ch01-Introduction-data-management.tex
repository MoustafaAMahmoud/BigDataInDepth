%---------------------------------------------------------
\begin{frame}
\frametitle{Videos classification}

\begin{table}[t]
	\centering	
	\begin{tabular}{|c |c | c | c|}
		\hline
		\thead{Watching Method \\ / Audience}  & \thead{Computer} & \thead{Mobile/Tablet} &  \thead{Just 	listening} \\
		\hline
		\thead{Developer} &   &   & \bluecircled\\
		\hline
		\thead{DevOps}  &  &  &  \bluecircled\\
		\hline
		\thead{Business} &  &  & \bluecircled \\
		\hline%Watching Method \newline Audience
	\end{tabular}
	\centering
	\vspace{.6\baselineskip}
	\caption{Video classification\\ The green circle \greencircled \space means short video. \\The blue circle \bluecircled \space  means medium video.\\ The red circle \redcircled \space  means long video}\label{Tab:Data_Representation_Matrix}
\end{table}
\end{frame}

%---------------------------------------------------------


\section{Introduction To Data Management and Data Warehouse}

%%%%%%%%%%%%%%%%%%%%%%%%%%%%%%%%%%%%%%%%%%%%%%%%%%%%%%%%%%%%%%%%%%%%%%%%%%%%%%%%%%%%%%%%%

\begin{frame}
\frametitle{Chapter Objectives}

\begin{itemize}[<+->]
	\item Be familiar with data management life-cycle.
	\item Understand the data abstraction and the data layer.
	\item Motivation to DWH.
	\item What are the different types of DWH?
	\item Usecases for DWH. How is it different from the operational DB?
	\item Explain the data Encoding and Formats.
	\item Show what the challenges of building a DWH are?
	\item What are the data modeling and its design?
\end{itemize}
%https://opentextbc.ca/dbdesign01/chapter/chapter-5-data-modelling/
\end{frame}

%%%%%%%%%%%%%%%%%%%%%%%%%%%%%%%%%%%

\subsection{Data Management}

\begin{frame}
\frametitle{Data Management}

\begin{itemize}[<+->]
\item Data are a product.
\item Data product has a life-cycle as following (simplified):
\begin{itemize}[<+->]
	\item \textbf{Question}, Idea, or service.
	\item \textbf{Identify} the source of information and the data type.
	\item \textbf{Document} all details regarding the data including quality, security, efficiency, and access (consideration during the cycle).
	\item Delivery automation (Tools and Process). AKA \textbf{DevOps} cycle.
	\item Data Architecture (model design and rules).
	\item \textbf{Extraction}, \textbf{Transformation}, and \textbf{Loading} Process.
	\item Business Intelligence (\textbf{BI}) or data discovery (continues process).
	\item \textbf{Integration} and publishing.
	\item Data retention or \textbf{archiving} process ex: (Hot or Cold storage).
\end{itemize}
\end{itemize}

\end{frame}

%%%%%%%%%%%%%%%%%%%%%%%%%%%%%%%%%%%%%%%%%%%%%%%%%%%%%%%%%%%%%%%%%%%%%%%%%%%%%%%%%%%%%%%%%

\begin{frame}
\frametitle{Data Management Life-Cycle}
\scalebox{0.9}{

\smartdiagramset{circular distance=3.5cm,
font=\scriptsize,
%				text width=1cm,
module minimum width=2cm,
circular distance =3.4cm,
module minimum height=.1cm,
arrow tip=to}
\smartdiagram[circular diagram]{Archiving,Idea, Identify, Document, 				
DevOps, Arch.,ETL, BI, Integration}
}
\end{frame}

%---------------------------------------------------------
\begin{frame}
\frametitle{Videos classification}

\begin{table}[t]
	\centering	
	\begin{tabular}{|c |c | c | c|}
		\hline
		\thead{Watching Method \\ / Audience}  & \thead{Computer} & \thead{Mobile/Tablet} &  \thead{Just 	listening} \\
		\hline
		\thead{Developer} &   &  \bluecircled & \\
		\hline
		\thead{DevOps}  &  & \bluecircled &  \\
		\hline
		\thead{Business} &  & \bluecircled & \\
		\hline%Watching Method \newline Audience
	\end{tabular}
	\centering
	\vspace{.6\baselineskip}
	\caption{Video classification\\ The green circle \greencircled \space means short video. \\The blue circle \bluecircled \space  means medium video.\\ The red circle \redcircled \space  means long video}\label{Tab:Data_Representation_Matrix}
\end{table}
\end{frame}

%%%%%%%%%%%%%%%%%%%%%%%%%%%%%%%%%%%%%%%%%%%%%%%%%%%%%%
\subsection{Data Abstraction}
\begin{frame}
	\frametitle{Motivation to Data Layers (Use Case)}	
%	\begin{itemize}[<+->]
%		\item You hired at a company which doesn’t have any system.
%		\item You started to collect all the data from files into CSV file format.
%		\item You built a system which read the data from CSV files.
%		\item Your manager asked you to create some report which required to combine different component.
%		\item After successfully generate this reports your manager asked you to create 100+ reports to support other departments.
%		\item For optimization purpose you changed the files from CSV into Json and update the searching mechanism into the files.         
%	\end{itemize}
%	
	
    \begin{figure}[H]
    	\smartdiagramset{
    		%descriptive items y sep = 3em,
    		description font = \scriptsize\sffamily,
    		description title font=\scriptsize\sffamily,
    	}
%	\centering
	\begin{subfigure}[t]{0.475\textwidth}
%		\centering
		\scalebox{0.5}{
		\smartdiagram[descriptive diagram]{
			{App, Application UI},
			{FS, CSV Data}}
		}
		\vspace{-.6\baselineskip}
		\caption{{\tiny Two layers Arch. (Data \& UI)}}
		\label{fig:ch_1_data_abstraction_1}
	\end{subfigure}
	\hfill
	\begin{subfigure}[t]{0.475\textwidth}
%		\centering 
		\scalebox{0.5}{
			\smartdiagram[descriptive diagram]{
				{App, Application UI},
				{BL, CSV Data Loader (Reporting)},
				{FS, CSV Data}}
			}
		\vspace{-.6\baselineskip}
		\caption{{\tiny Three layers Arch. (Data \& BL \& UI)}}
		\label{fig:ch_1_data_abstraction_2}
	\end{subfigure}
%	\vskip\baselineskip
	\begin{subfigure}[t]{0.475\textwidth}   
%		\centering 
		\scalebox{0.5}
		{
			\smartdiagram[descriptive diagram]{
			{App, Application UI},
			{BL, (JSON/CSV) Data Loader (Reporting)},
			{FS, JSON Data}}
		}
		\vspace{-.6\baselineskip}
		\caption{{\tiny Three layers Arch. (Data (multi-sources) \& BL \& UI)}}
		\label{fig:ch_1_data_abstraction_3}
	\end{subfigure}
	\quad
	\begin{subfigure}[t]{0.475\textwidth}   
%		\centering 
		\scalebox{0.5}
		{
			\smartdiagram[descriptive diagram]{
				{App, Application UI},
				{DBMS-H, Ready prepared layer for each department (Reporting)},
				{DBMS-M, Logical part to prepare for the data structure and the relation between the data},
				{DBMS-L, Storage and Data format related stuff + Data indexing and searching algorithms}}
		}
		\vspace{-.6\baselineskip}
		\caption{{\tiny Four layers Arch. (DB (L, M, H) \& UI)}}
		\label{fig:ch_1_data_abstraction_4}
	\end{subfigure}
	\vspace{-.6\baselineskip}
	\caption {\tiny Data Abstraction Journey} 
	\label{fig:ch_1_data_abstraction}
	\end{figure}
	


\end{frame}
%%%%%%%%%%%%%%%%%%%%%%%%%%%%%%%%%%%%%%%%%%%%%%%%%%%%%%
\begin{frame}
	\frametitle{Motivation to Data Layers (Solution Thinking)}
	
	\begin{itemize}[<+->]
		\item How can we think about a data solution or challenges in the data products?
		\begin{itemize}[<+->]
			\item Requirements analysis.
			\item Identify the problem (challenges).
			\item Think about how to overcome the challenges.
			\item Ask your self the following questions:
			\begin{itemize}[<+->]
				\item Can we solve the problem using the current data structure by adding new features?
				\item What if we enhance/change the data structure or modeling?
				\item Could it help if we change the backend engine (ex: DBMS system)?
			\end{itemize}			
		\end{itemize}
		\item To answer these questions you need to understand the \textbf{\underline{data layers}}.
	\end{itemize}
	
\end{frame}
%%%%%%%%%%%%%%%%%%%%%%%%%%%%%%%%%%%%%%%%%%%%%%%%%%%%%%
\begin{frame}
	\frametitle{Data Layers (Abstraction)}
	\begin{itemize}[<+->]
		\item Any data product (database) contains multi-layers.
		\item Each layer responsible for different tasks and operations.
		\item Each layer interacts with (hardware or software or mixed).
		\item Eliminate the complexity of data interactions; not all internal processes are shared or available for the user.
		\item The developer for each layer hides irrelevant internal details from the developer (users). 
		\item The process of \textbf{\underline{\blue{hiding}}} irrelevant details from the developer (user) is called data \textbf{\underline{\blue{abstraction}}}.
	\end{itemize}	
\end{frame}
%%%%%%%%%%%%%%%%%%%%%%%%%%%%%%%%%%%%%%%%%%%%%%%%%%%%%%
\begin{frame}
	\frametitle{Data Layers (Abstraction)}
	\begin{definition}
		\textbf{Data Abstraction and Data Independence}: DBMS comprises complex data-structures. To make the system efficient in terms of retrieval of data and reduce complexity in terms of usability of users, developers use abstraction i.e., hide irrelevant details from the users. This approach simplifies database design.
		%https://www.geeksforgeeks.org/data-abstraction-and-data-independence/
	\end{definition}	
	%Capacity of changing in one level without affecting the other levels. Copied but forget from where!!!
	\begin{itemize}[<+->]
		\item There are 3 levels of data abstraction.
		\begin{itemize}[<+->]
			\item Physical Level
			\item Logical/Conceptual Level.
			\item View Level.
		\end{itemize}
	\end{itemize}	
	%TOP TIER, MIDDLE TIER, BOTTOM TIER
\end{frame}
%%%%%%%%%%%%%%%%%%%%%%%%%%%%%%%%%%%%%%%%%%%%%%%%%%%%%%
\begin{frame}
	\frametitle{Data Layers (Abstraction)}
	\scalebox{0.9}{
\begin{tikzpicture}[node distance=2cm,
					every node/.style={fill=white, font=\sffamily}, align=center,
					scale=0.6, 
					every node/.style={transform shape}]

% Specification of nodes (position, etc.)
\node (view2)             [optionalETL]              {report 2 (view)};
\node (view1)     [optionalETL, right of=view2, xshift=3cm]          {report 1 (view)};
\node (view3)      [optionalETL, right of=view1, xshift=3cm]   {report 3 (view)};
\node (concept)     [required,below of=view1, yshift=-1.5cm]   {Conceptual Layer};
\node (physical)      [optionalELK, below of=concept, yshift=-1.5cm] {Physical Interaction};
\node (fs)      [optionalELK, below of=physical, yshift=-1cm] {FS};

%\node (Appendix) [startstop, above of=Arch] {Ch.13 Appendix};     
% Normal Path
\draw[<->]     (view2) -- (concept);
\draw[<->]     (view3) -- (concept);
\draw[<->]     (view1) -- (concept);
\draw[<->]     (concept) -- (physical);
\draw[<->]     (fs) -- (physical);
\draw[-]      (12,-1.5) to[out=0,in=180] (13,0)  node[right]{View Level (User View) } to[out=180,in=0] (12,1.5);
\draw[-]      (12,-5) to[out=0,in=180] (13,-3.5) node[right]{Logical/ Conceptual Level } to[out=180,in=0]  (12,-2);
\draw[-]      (12,-11) to[out=0,in=180] (13,-8.5) node[right]{Physical Level } to[out=180,in=0]  (12,-6);
\end{tikzpicture}
}
\end{frame}

%---------------------------------------------------------
\begin{frame}
\frametitle{Videos classification}

\begin{table}[t]
	\centering	
	\begin{tabular}{|c |c | c | c|}
		\hline
		\thead{Watching Method \\ / Audience}  & \thead{Computer} & \thead{Mobile/Tablet} &  \thead{Just 	listening} \\
		\hline
		\thead{Developer} &   &   & \greencircled \\
		\hline
		\thead{DevOps}  &  &  & \greencircled  \\
		\hline
		\thead{Business} &  &  & \greencircled \\
		\hline%Watching Method \newline Audience
	\end{tabular}
	\centering
	\vspace{.6\baselineskip}
	\caption{Video classification\\ The green circle \greencircled \space means short video. \\The blue circle \bluecircled \space  means medium video.\\ The red circle \redcircled \space  means long video}\label{Tab:Data_Representation_Matrix}
\end{table}
\end{frame}
%%%%%%%%%%%%%%%%%%%%%%%%%%%%%%%%%%%%%%%%%%%%%%%%%%%%%%
\begin{frame}
	\frametitle{Physical level}
	\begin{itemize}[<+->]
		\item \textbf{Physical level (Internal)}: 
		\begin{itemize}[<+->]
			\item Lowest level.
			\item Describes \textbf{\underline{\blue{how}}} data is stored.
			\item Describes the data structure.
			\item It allows you to modify the lowest level (Physical part) without any change in the logical schema. These change could be
				\begin{itemize}[<+->]				
					\item Using a new storage device
					\item Change the structure of the data used for storage
					\item Change the file type or use a different storage structure
					\item Chang the access method
					\item Modify indexes
					\item Change the compression algorithm or hashing technique.
			\end{itemize}									
		\end{itemize}		
		%https://beginnersbook.com/2015/04/levels-of-abstraction-in-dbms/		
		%https://www.guru99.com/dbms-data-independence.html
	\end{itemize}	
\end{frame}
%%%%%%%%%%%%%%%%%%%%%%%%%%%%%%%%%%%%%%%%%%%%%%%%%%%%%%
\begin{frame}
	\frametitle{Physical level}
	\begin{example}		
		\begin{itemize}[<+->]
			\item Database contains product information.
			\item Physical layer describes
			\begin{itemize}[<+->]
				\item Storage mechanism and the blocks (bytes, gigabytes, terabytes, etc.).
				\item The amount of memory used.
				\item Usually this layer abstracted from the programmers.
			\end{itemize}
		\end{itemize}
	\end{example}
	
\end{frame}
%---------------------------------------------------------
\begin{frame}
\frametitle{Videos classification}
\begin{table}[t]
	\centering	
	\begin{tabular}{|c |c | c | c|}
		\hline
		\thead{Watching Method \\ / Audience}  & \thead{Computer} & \thead{Mobile/Tablet} &  \thead{Just 	listening} \\
		\hline
		\thead{Developer} &   &   & \greencircled \\
		\hline
		\thead{DevOps}  &  &  & \greencircled  \\
		\hline
		\thead{Business} &  &  & \greencircled \\
		\hline%Watching Method \newline Audience
	\end{tabular}
	\centering
	\vspace{.6\baselineskip}
	\caption{Video classification\\ The green circle \greencircled \space means short video. \\The blue circle \bluecircled \space  means medium video.\\ The red circle \redcircled \space  means long video}\label{Tab:Data_Representation_Matrix}
\end{table}
\end{frame}

%%%%%%%%%%%%%%%%%%%%%%%%%%%%%%%%%%%%%%%%%%%%%%%%%%%%%%
\begin{frame}
	\frametitle{Logical level}
	\begin{itemize}[<+->]
		\item \textbf{Logical level (Conceptual)}: 
		\begin{itemize}[<+->]
			\item Intermediate level
			\item Describes \textbf{\underline{\blue{what}}} data is stored
			\item Describes what the relationship between the stored data is?
			\item It allows you to change the logical view without altering the external view, API, or programs. These change could be
			\begin{itemize}[<+->]
				\item Add a new table
				\item Change the records merge or delete without affecting the running applications
				\item Change attribute (Add,delete) to the existing table
			\end{itemize}									
		\end{itemize}		
	\end{itemize}	 
\end{frame}
%%%%%%%%%%%%%%%%%%%%%%%%%%%%%%%%%%%%%%%%%%%%%%%%%%%%%%
\begin{frame}
	\frametitle{Logical level}
	\begin{example}
		\begin{itemize}[<+->]
			\item Database contains product information.
			\item Logical Layer describes
			\begin{itemize}[<+->]
				\item The product fields and their data types
				\item How this product interact with other entities in the database
				\item The programmers' design this level based on business knowledge and the requirements
			\end{itemize}
		\end{itemize}
	\end{example}
	
\end{frame}


%---------------------------------------------------------
\begin{frame}
\frametitle{Videos classification}

\begin{table}[t]
	\centering	
	\begin{tabular}{|c |c | c | c|}
		\hline
		\thead{Watching Method \\ / Audience}  & \thead{Computer} & \thead{Mobile/Tablet} &  \thead{Just 	listening} \\
		\hline
		\thead{Developer} &   &   & \greencircled \\
		\hline
		\thead{DevOps}  &  &  & \greencircled  \\
		\hline
		\thead{Business} &  &  & \greencircled \\
		\hline%Watching Method \newline Audience
	\end{tabular}
	\centering
	\vspace{.6\baselineskip}
	\caption{Video classification\\ The green circle \greencircled \space means short video. \\The blue circle \bluecircled \space  means medium video.\\ The red circle \redcircled \space  means long video}\label{Tab:Data_Representation_Matrix}
\end{table}
\end{frame}
%%%%%%%%%%%%%%%%%%%%%%%%%%%%%%%%%%%%%%%%%%%%%%%%%%%%%%
\begin{frame}
	\frametitle{View level}
	\begin{itemize}[<+->]
		\item \textbf{View level (External)}: 
		\begin{itemize}[<+->]
			\item Highest level.
			\item \textbf{\underline{\blue{View}}} of the data stored? 
			\item Designed for a category of users
			\item The final interface for the user
			\item Extended or hidden based on the user's role
			\item Not all the views is extended to all users, and there is authentication based on the category
		\end{itemize}		
	\end{itemize}	
	
\end{frame}
%%%%%%%%%%%%%%%%%%%%%%%%%%%%%%%%%%%%%%%%%%%%%%%%%%%%%%
\begin{frame}
	\frametitle{View level}
	\begin{example}
		\begin{itemize}[<+->]
			\item The database contains product information
			\item It could be designed to show the sales of the product in a specific region
			\item We might hide information about some products based on the teams or users
		\end{itemize}
	\end{example}
	
\end{frame}

%---------------------------------------------------------
\begin{frame}
\frametitle{Videos classification}

\begin{table}[t]
	\centering	
	\begin{tabular}{|c |c | c | c|}
		\hline
		\thead{Watching Method \\ / Audience}  & \thead{Computer} & \thead{Mobile/Tablet} &  \thead{Just 	listening} \\
		\hline
		\thead{Developer} &   & \bluecircled  &  \\
		\hline
		\thead{DevOps}  &  & \bluecircled &   \\
		\hline
		\thead{Business} &  &\bluecircled  &  \\
		\hline%Watching Method \newline Audience
	\end{tabular}
	\centering
	\vspace{.6\baselineskip}
	\caption{Video classification\\ The green circle \greencircled \space means short video. \\The blue circle \bluecircled \space  means medium video.\\ The red circle \redcircled \space  means long video}\label{Tab:Data_Representation_Matrix}
\end{table}
\end{frame}
%%%%%%%%%%%%%%%%%%%%%%%%%%%%%%%%%%%%%%%%%%%%%%%%%%%%%%
\begin{frame}[c]
	\frametitle{Data solution thinking (Summary) }
	% !!!!!!!!!!!!!! We need to mention that this slide is just an overview there will be a detailed one later
        \begin{center}
			Let's answer our previous question. How can we solve data challenges?
        \end{center}
    \end{frame}

%%%%%%%%%%%%%%%%%%%%%%%%%%%%%%%%%%%%%%%%%%%%%%%%%%%%%%

%%%%%%%%%%%%%%%%%%%%%%%%%%%%%%%%%%%%%%%%%%%%%%%%%%%%%% 
\begin{frame}
	\frametitle{Data solution thinking (Summary) }
	\begin{itemize}[<+->]
        \item Let's split the problem based on the data layers.
          \begin{itemize}[<+->]
          \item View layer
            \begin{itemize}[<+->]
            \item When we need to add/remove/create new reports, it is usually a view layer.
            \item We don't need to change the logical or physical layer to support the view layer.
          \end{itemize}
        \end{itemize}
       \end{itemize}
 \end{frame}

%%%%%%%%%%%%%%%%%%%%%%%%%%%%%%%%%%%%%%%%%%%%%%%%%%%%%% 
\begin{frame}
\frametitle{Data solution thinking (Summary) }
	\begin{itemize}[<+->]
        \item Let's split the problem based on the data layers.
          \begin{itemize}[<+->]
           \item Logical Layer
           \begin{itemize}[<+->]
             \item When you have missing sources into your logical layer, and you need to add this source and its structure.
             \item There is a performance issue in the existing reports, and you need to change the model. For example, reduce the join by creating a new join table (\textit{materialized view}).
             \item Update the data type or the existing relation, which could help to fix some data or performance issues.
            \end{itemize}
           \end{itemize}
        \end{itemize}
 \end{frame}

 %%%%%%%%%%%%%%%%%%%%%%%%%%%%%%%%%%%%%%%%%%%%%%%%%%%%%%%
 %%%%%%%%%%%%%%%%%%%%%%%%%%%%%%%%%%%%%%%%%%%%%%%%%%%%%% 
\begin{frame}
  \frametitle{Data solution thinking (Summary) }
  \begin{itemize}[<+->]
  \item Let's split the problem based on the data layers.
    \begin{itemize}[<+->]
    \item Physical Layer
      \begin{itemize}[<+->]
      \item When our problem is hard or impossible to fix by optimizing the query (view)/ logical layer, it is time for physical change.
      \item If we need to change your storage/compression/structure/access technique.
      \item If we need to change the data orientation structure from row to column or key-value storage, It is time to change the physical layer.
      \end{itemize}
    \end{itemize}
  \end{itemize}
 \end{frame}

%%%%%%%%%%%%%%%%%%%%%%%%%%%%%%%%%%%%%%%%%%%%%%%%%%%%%%%
\subsection{Introduction to DWH}

\subsubsection{Motivation to Data Warehouse (DWH)}
\begin{frame}
\frametitle{Motivation to Data Warehouse (DWH)}
	\begin{itemize}[<+->]
		\item Data could be a product for some companies.
		\item It could be decision support for other products or businesses.
		\item Reporting the results after passing the data life-cycle will be from storage (Database).
		\item There are some challenges facing the people who work on data management backend:
			\begin{itemize}[<+->]
				\item Performance.
				\item Integration.
				\item Applying analytical functions. %Moving average
			\end{itemize}
		\item Vendors who are working to solve the above challenges creating their own product of DWH and their ultimate work is to optimize the above points.
	\end{itemize}
\end{frame}
%%%%%%%%%%%%%%%%%%%%%%%%%%%%%%%%%%%%%%%%%%%%%%%%%%%%%%
\begin{frame}[c]
\frametitle{Motivation to Data Warehouse (DWH)}

\begin{definition}[What is Data Warehousing?] A DWH is defined as a technique for collecting and managing data from varied sources to \textbf{provide meaningful business insights}. It is a blend of technologies and components which aids the strategic use of data.%\footnotemark
\end{definition}

%REF
The real concept was given by Inmon Bill. He was considered as a father of the DWH. He had written about a variety of topics for building, usage, and maintenance of the warehouse \& the Corporate Information Factory

%\footnotetext{The definition mentioned in this slides copied from  \href{https://www.guru99.com/data-warehousing.html\#2}{guru99.com} }

\end{frame}

%%%%%%%%%%%%%%%%%%%%%%%%%%%%%%%%%%%%%%%%%%%%%%%%%%%%%%

%%%%%%%%%%%%%%%%%%%%%%%%%%%%%%%%%%%%%%%%%%%%%%%%%%%%%%
\begin{frame}
\frametitle{Motivation to Data Warehouse (DWH)}

\begin{itemize}[<+->]
	\item The DWH is not a product but an environment.
	\item It is a process of transforming data into information and make it available to users in a \textbf{timely manner} to make a difference.
	\item It is an architectural construct of an information system which provides users with current and historical decision support information which is difficult to access or present in the traditional operational data store.
	\item The DWH is the core of the BI system which is built for data analysis and reporting.
\end{itemize}

\end{frame}

%%%%%%%%%%%%%%%%%%%%%%%%%%%%%%%%%%%%%%%%%%%%%%%%%%%%%%

\begin{frame}
\frametitle{Motivation to Data Warehouse}

Data warehouse system is also known by the following names:


\begin{wideitemize}
\item Decision Support System (DSS).
\item Business Intelligence Solution.
\item Executive Information System.
\item Management Information System.
\item Analytic Application.
\item Data Warehouse.

\end{wideitemize}


\end{frame}


%%%%%%%%%%%%%%%%%%%%%%%%%%%%%%%%%%%%%%%%%%%%%%%%%%%%%%
\subsubsection{Differences Between DWH and Operational DB}
\begin{frame}
	\frametitle{DWH vs Operational databases}
	
	
	\begin{table}[t]
		\centering	
		\resizebox{\columnwidth}{!}{%
			
			%		\centering
			\begin{tabular}{|c | c | c|}
				\hline
				\textbf{Metric}  & \textbf{Transactions DB}& \textbf{DWH} \\
				\hline
				Volume & GB/TB & TB/PB \\
				Historical  & Short-term & Long-Term\\
				rows & <1000M &  1000M>\\
				Orientation & Product & Subject or multi products\\
				Business Units & Product team & Multi organizational units\\
				Normalization & Normalized %due to storage and performance limitation and its design
				&  Not required (De-normalized in many use cases)\\
				Data Model & Relational & Star Schema or Multi-dim\\
				Intelligence&Reporting & Advanced reporting and Machine Learning\\
				Use cases& Online transactions \& operations & Centeralized storage (360\textdegree)\\
				\hline
			\end{tabular}
			%		\caption{Data Representation Combination Matrix}\label{Tab:Data_Representation_Matrix}
		}
	\end{table}
\end{frame}


%%%%%%%%%%%%%%%%%%%%%%%%%%%%%%%%%%%%%%%%%%%%%%%%%%%%%%

\begin{frame}
\frametitle{Transnational DB Use cases}
\begin{figure}[ht]
	
	\centering
	\includegraphics[width=\linewidth]{./Figures/chapter-01/baby-01.jpg}
	%		\includegraphics[width=\linewidth,height=\textheight]{./Figures/chapter-01/baby-02.jpg}
	%	\caption{}
\end{figure}
\end{frame}


%%%%%%%%%%%%%%%%%%%%%%%%%%%%%%%%%%%%%%%%%%%%%%%%%%%%%%
\begin{frame}
\frametitle{Transnational DB Use cases}
\begin{figure}[ht]

\centering
%	\includegraphics[width=\linewidth]{./Figures/chapter-01/baby-01.jpg}
\includegraphics[width=\linewidth]{./Figures/chapter-01/baby-02.jpg}
%	\caption{}
\end{figure}
\end{frame}


%%%%%%%%%%%%%%%%%%%%%%%%%%%%%%%%%%%%%%%%%%%%%%%%%%%%%%
\begin{frame}
\frametitle{DWH Use cases}
\begin{figure}[ht]

\centering
\includegraphics[width=\linewidth,height=.8\textheight]{./Figures/chapter-01/Marvel-03.jpg}
%	\caption{}
\end{figure}
\end{frame}

%%%%%%%%%%%%%%%%%%%%%%%%%%%%%%%%%%%%%%%%%%%%%%%%%%%%%%
\begin{frame}
\frametitle{DWH Use cases}
\begin{figure}[ht]

\centering
\includegraphics[width=\linewidth,height=.8\textheight]{./Figures/chapter-01/Marvel-02.jpg}
%	\caption{}
\end{figure}
\end{frame}

%%%%%%%%%%%%%%%%%%%%%%%%%%%%%%%%%%%%%%%%%%%%%%%%%%%%%%
\begin{frame}
\frametitle{DWH Use cases}
\begin{figure}[ht]

\centering
\includegraphics[width=\linewidth,height=.8\textheight]{./Figures/chapter-01/Marvel-01.jpg}
%	\caption{}
\end{figure}
\end{frame}
%%%%%%%%%%%%%%%%%%%%%%%%%%%%%%%%%%%%%%%%%%%%%%%%%%%%%%


%%%%%%%%%%%%%%%%%%%%%%%%%%%%%%%%%%%%%%%%%%%%%%%%%%%%%%
\subsubsection{Types of DWH}
\begin{frame}
\frametitle{Motivation to Data Warehouse}
Types of Data Warehouse
\begin{description}
\item [\textbf{Enterprise Data Warehouse (EDWH)}] It provides decision support service across the enterprise. It offers a unified approach for organizing and representing data (DWH Model). It offers data classifications according to the subject with privileges policy.
\item [\textbf{Operational Data Store (ODS):}] is a central database that provides an up-to-date (real-time) data from multiple transnational systems for operational reporting into a single DWH.

%% for real time questions and answers. call ODS using intermidiate data store. DWH is day -1 (billing & subscribtions). Oracle (Loading or headache) && CDC capture change interest column not row
\item [\textbf{Data Mart:}] A data mart is a subset of the data warehouse. It specially designed for a particular line of business, such as sales, finance, sales or finance. In an independent data mart, data can collect directly from sources.
\end{description}

\end{frame}

%%%%%%%%%%%%%%%%%%%%%%%%%%%%%%%%%%%%%%%%%%%%%%%%%%%%%%

\begin{frame}
\frametitle{DWH vs ODS vs Data Mart}


\begin{table}[t]
\centering	
\resizebox{\columnwidth}{!}{%

%		\centering
\begin{tabular}{|c | c | c| c |}
\hline
\textbf{Metric}  & \textbf{DWH}& \textbf{ODS} & \textbf{Data Mart} \\
\hline
Latency & Day -1  & Real-time & Day -1 \\			
Data level  & Transnational & Transnational & Summary \\
Historical  & Long-term & Snapshot & Aggregated Long-Term \\
Size & TB/PB & GB & GB/TB\\
Orientation & Multi sources & Multi sources & Product\\
Business Units & Multi organizational units & Product team & Business team \\
\hline
\end{tabular}
%		\caption{Data Representation Combination Matrix}\label{Tab:Data_Representation_Matrix}
}
\end{table}
\end{frame}


%%%%%%%%%%%%%%%%%%%%%%%%%%%%%%%%%%%%%%%%%%%%%%%%%%%%%%%%%%%%%%%%%%%%%%%%
\subsubsection{Use Cases of Operational DB vs DWH}

\begin{frame}
\frametitle{Use case (Operational DB)}

\begin{itemize}[<+->]

\item A telecommunication company named \textbf{XTec}.
\newline
\item They have lots of systems. One of this systems is a CRM system as example of operational DB.
\begin{itemize}[<+->]

\item The CRM system handles the customer activities with the company including (sales, change in customer plans, and other activities).
\item This system has a backend database (MySQL).
\item CRM team can report their sales and customer activities from their database.
\item Product owner can take a decision based on their system backend reports.

\end{itemize}

\end{itemize}

\end{frame}

%%%%%%%%%%%%%%%%%%%%%%%%%%%%%%%%%%%%%%%%%%%%%%%%%%%%%%

\begin{frame}
\frametitle{Use case (DWH)}

\begin{itemize}[<+->]

\item What is the need for DWH?		
\begin{itemize}[<+->]
\item This company has other systems for example: billing, charging, signaling.	
\item They need to report information related to the CRM, billing, and signaling source systems in one report.
\item So, they need to ingest (transfer) the data from the source systems to one single database.
\item The decision from the DHW is a \textbf{global and strategical decision.}
\item If the company needs to build a machine learning model which needs data from different sources. They need to load the data from a centralized database rather than read each source alone.
\end{itemize}

\end{itemize}

\end{frame}
%%%%%%%%%%%%%%%%%%%%%%%%%%%%%%%%%%%%%%%%%%%%%%%%%%%%%%


\begin{frame}
\frametitle{Use case (DWH)}
\centering
The Full picture required a DWH. However, we still need the other operational databases for product development perspective.


\end{frame}
%%%%%%%%%%%%%%%%%%%%%%%%%%%%%%%%%%%%%%%%%%%%%%%%%%%%%%

\begin{frame}
\frametitle{Use case (ODS)}
\centering

\begin{itemize}[<+->]
\item Why do we need the ODS?
\item 	How does it fit in our system?
\end{itemize}


\end{frame}
%%%%%%%%%%%%%%%%%%%%%%%%%%%%%%%%%%%%%%%%%%%%%%%%%%%%%%


%%%%%%%%%%%%%%%%%%%%%%%%%%%%%%%%%%%%%%%%%%%%%%%%%%%%%%

\begin{frame}
\frametitle{Use case (ODS)}
\centering
\textbf{XTec} has a call center system which handles the customer inquiries. This system requires the some data related to usage, customer information, billing details to be calculated and accumulated in \textbf{real-time} to be able to give the customer the right answer for his inquires.

\end{frame}
%%%%%%%%%%%%%%%%%%%%%%%%%%%%%%%%%%%%%%%%%%%%%%%%%%%%%%
%%%%%%%%%%%%%%%%%%%%%%%%%%%%%%%%%%%%%%%%%%%%%%%%%%%%%%
\begin{frame}
\frametitle{Use case (ODS)}

	\begin{itemize}[<+->]
		\item So, What is the challenge for this system?
			\begin{itemize}[<+->]		
				\item It needs specific information from different source systems.
				\item It requires to track the source system database changes or update in real-time.
				\item It's functionality is based on the aggregate data not the transactions for example (It needs the total outgoing calls till time or it needs the total charging amounts from prepaid or the available limits from billing if it is postpaid).
			\end{itemize}
	\end{itemize}

\end{frame}
%%%%%%%%%%%%%%%%%%%%%%%%%%%%%%%%%%%%%%%%%%%%%%%%%%%%%%
\begin{frame}
\frametitle{Use case (ODS)}

	\begin{itemize}[<+->]
		\item ODS is based on change data capture (CDC). This approach used to determine the data change and apply action based on this change.
		\item ODS uses the real-time aggregations to support the online systems from different source systems.
	\end{itemize}
\end{frame}
%%%%%%%%%%%%%%%%%%%%%%%%%%%%%%%%%%%%%%%%%%%%%%%%%%%%%%
\subsection{DWH Characteristics}
\begin{frame}
\frametitle{DWH Characteristics}
%these is not a definitions, we just show the meaning and the understanding.
\begin{wideitemize}
\item The characteristics of DWH:
	\begin{wideitemize}
		\item Integrated: \textit{DWH is an integrated environment which allows us to integrate different source systems. Data are modeled (organized) into a unified manner.} %regardless of the original source
	
		\item Time-Variant: \textit{Data modeled (organized) based on time periods (hourly, daily, weekly, monthly, quarterly, yearly, etc.)}
	
		\item Subject-oriented: \textit{DWH main target is to support business needs for the whole organization including (decision makers, departments, and specific user requirements)}.
	
		\item Non-Volatile: \textit{It refers to the data will not erased or deleted (It could be archived and retrieved when needed). Data can be accumulated daily the new snapshots (refreshed at based on the source system interval. For example, It could be updated daily, weekly, and monthly).}
	\end{wideitemize}	
\end{wideitemize}
%https://www.guru99.com/data-warehouse-architecture.html

\end{frame}

\subsection{Hot vs Cold Storage}
\begin{frame}
\frametitle{Hot vs Cold Storage}
SOME DETAILS HERE
\end{frame}

%%%%%%%%%%%%%%%%%%%%%%%%%%%%%%%%%%%%%%%%%%%%%%%%%%%%%%
\subsection{DWH Architecture}
\begin{frame}
\frametitle{DWH Architecture Layers}

\begin{wideitemize}
	\item DWH Architecture contains the following layers:
	\begin{itemize}[<+->]
		\item Source system layer.
		\item Extraction layer.
		\item Staging Area.
		\item Data Modeling.
		\item ETL layer.
		\item Storage layer.
		\item Logical layer.
		\item Reporting (UI) layer.
		\item Metadata layer.
		\item System operations layer.
	\end{itemize}	
\end{wideitemize}

\end{frame}

%%%%%%%%%%%%%%%%%%%%%%%%%%%%%%%%%%%%%%%%%%%%%%%%%%%%%%
\begin{frame}
\frametitle{DWH Architecture Overview}
\begin{figure}[ht]
	%@@@ TODO: add the link to resources
	\centering
	\includegraphics[width=.9\linewidth,height=.8\textheight]{./Figures/chapter-01/Datawarehouse_reference_architecture.jpg}
	%https://commons.wikimedia.org/wiki/File:Datawarehouse_reference_architecture.jpg
	%		\includegraphics[width=\linewidth,height=\textheight]{./Figures/chapter-01/baby-02.jpg}
	\caption{taken from https://commons.wikimedia.org/wiki/File:Datawarehouse\_reference\_architecture.jpg}
\end{figure}
\end{frame}

%%%%%%%%%%%%%%%%%%%%%%%%%%%%%%%%%%%%%%%%%%%%%%%%%%%%%%
\subsubsection{Source System Integration Process}
\begin{frame}
\frametitle{Source System Integration Process}
\begin{itemize}[<+->]
\item In some companies they hire or dedicate a team for this part (business analyst, system analyst, data analyst, or demand team).
\item Before we start, ALL communications from start till the end should be documented into any format.
	\begin{itemize}
		\item  Conflounce page, Word, or Excel sheet.
		\item  Make the discussion online and put comments to make the history available always.
		\item  All tasks should be clear what is the expected output for example (analysis means to document data structure, format, column names, etc..).
	\end{itemize}
\end{itemize}

\end{frame}

%%%%%%%%%%%%%%%%%%%%%%%%%%%%%%%%%%%%%%%%%%%%%%%%%%%%%%
\begin{frame}
\frametitle{Source System Integration Process}
\begin{itemize}[<+->]
	\item  Requirements gathering. % or business need (It could be DWH unification).
	\item  Identify the stakeholders (Data owner(s)).
	\item  Data Analysis includes but not only (format, latency, and column definitions).
	\item  Connectivity analysis and security (assessment).
	\item  Technical discussion about the best way to ingest the data.
	\item  Data Ingestion method and format.
	\item  Sign or confirmation for every point between the stakeholders.
	\item  \blue{This layer deliver a data analysis (Source system interface ) document}.
\end{itemize}

\end{frame}


%%%%%%%%%%%%%%%%%%%%%%%%%%%%%%%%%%%%%%%%%%%%%%%%%%%%%%

\subsubsection{Extraction Layer}

\begin{frame}
\frametitle{Extraction Layer}
\begin{itemize}[<+->]
	\item In some companies they hire or dedicate a team for this part (extraction or ingestion team) but in other companies it is part of the data engineering team.
	\item This layer take the output analysis and decisions from the previous layer (source system analysis) and implement the extraction \blue{(quality from the previous team output highly affect this team)}.
	\item There are many consideration this team need to take care about or deal with but we can summarize it in the following:
		\begin{itemize}			
			\item  Data latency will affect the tool and the methodology (stream or batch).
			\item  Data extraction method (push or pull).
			\item  Data size and format compared with the available resources for this project.
		\end{itemize}
	\item \blue{This layer output is a minimal data cleansing (no transformation) into the staging/landing layer}.

\end{itemize}

\end{frame}

%%%%%%%%%%%%%%%%%%%%%%%%%%%%%%%%%%%%%%%%%%%%%%%%%%%%%%

\subsubsection{Staging Layer}

\begin{frame}
\frametitle{Staging Layer}
\begin{itemize}[<+->]
	\item This layer handled by the same team who own the \blue{storage part} in most of the organizations. 
	\item Segregation of this layer if it uses different storage type or multi-teams access this layer for a different purpose (ex: Kafka)\\ \red{\textit{\faBullhorn Kafka is not a storage layer but it could be landing layer}}.
	\item All the ETL layers are working on top of this layer.
	\item The decision of the storage type is based on the use case and the data.
	
\end{itemize}

\end{frame}
%%%%%%%%%%%%%%%%%%%%%%%%%%%%%%%%%%%%%%%%%%%%%%%%%%%%%%

\subsubsection{Data Modeling}


\begin{frame}
\frametitle{Data Modeling Objective}
\begin{itemize}[<+->]
	\item Understanding the data modeling and its roles.
	\item Be aware about its importance.
	\item Explore different types of data modeling.
	\item \blue{We will not go in details about how to design in this part (we will explain it later and in the appendix)}.

\end{itemize}

\end{frame}

%%%%%%%%%%%%%%%%%%%%%%%%%%%%%%%%%%%%%%%%%%%%%%%%%%%%%%

\begin{frame}
\frametitle{What is data model?}
Data model is
\begin{itemize}[<+->]
	\item An abstract model that organizes elements of data.
	\item It describes the objects, entities and data structure properties, semantic, and constraint.
	\item It formalizes the relationship between entities.
	\item It describes how application (report) API data manipulation.
	\item It describes the conceptual design of a business or an application with its flow, logic, semantic information (rules), and how things are done.
	\item It refers to a set of concepts used in defining such as entities, attributes, relations, or tables.
\end{itemize}
\end{frame}

%%%%%%%%%%%%%%%%%%%%%%%%%%%%%%%%%%%%%%%%%%%%%%%%%%%%%%
\begin{frame}
\frametitle{What is data model?}

\begin{columns}

\column{0.4\textwidth}
Data model is not	
\begin{itemize}[<+->]
	\item a science.
	\item a static design for each organization.
	\item a type of database.
	\item a new invention which needs to be done for each project. %ex: sldm teradata model
\end{itemize}


\column{0.4\textwidth}
Data model is		
\begin{itemize}[<+->]
	\item a general concepts which lead to build full architecture.
	\item an engineering design practices.
	\item different based on the use case and the database type.
	\item customizable and we can utilize some of ready built architecture.
	\item affecting the information reporting performance and ways.
\end{itemize}

\column{0.2\textwidth}
\end{columns}

\end{frame}

%%%%%%%%%%%%%%%%%%%%%%%%%%%%%%%%%%%%%%%%%%%%%%%%%%%%%%

\begin{frame}
\frametitle{What is data model?}
Data model is
\begin{itemize}[<+->]
	\item The initial part before starting integration with any new source system.
	\item It the connection layer between the business requirements and the technical design.
	\item It is also the translation between logical and physical layer.
	\item It is unified across the all systems and has the same patterns and practices.
	\item It could be engage with any source systems integration from early stages.
	\item \blue{This stage output is data model design document or mapping sheet}.
\end{itemize}
\end{frame}

%%%%%%%%%%%%%%%%%%%%%%%%%%%%%%%%%%%%%%%%%%%%%%%%%%%%%%

\begin{frame}
\frametitle{Why does data models are important?}
\begin{wideitemize}	
\item Data models are currently affecting software design. 
\item It decides how engineers will think about the problem they are solving.
\end{wideitemize}
\end{frame}

%%%%%%%%%%%%%%%%%%%%%%%%%%%%%%%%%%%%%%%%%%%%%%%%%%%%%%
\begin{frame}
\frametitle{Data Model Design vs Implementation}
\red{REVIEW THIS EXAMPLE}
\begin{itemize}[<+->]
\item You need to build a home. So, how do we design this home?
\begin{itemize}[<+->]
\item Determine if the home is one level or multi-level and decide man bedrooms and bathrooms for each floor. (User needs)
\item Hire an architect to put the architecture in more detailed way for example, the size for each room, the distribution of the wireds, where the plumbing fixtures will be placed, etc. (Architecture phase)
\item Decide the decorations, colors for each room, carpets, etc. 
\end{itemize}
\item What do we do for the implementation?
\begin{itemize}[<+->]
\item Hire a contractor to build (implement the design) the home. 
\item This phase will implement the design but it also include some detail related to the actual way to build the tools and the material. (Physical Design)
\end{itemize}		
\end{itemize}
\end{frame}
%%%%%%%%%%%%%%%%%%%%%%%%%%%%%%%%%%%%%%%%%%%%%%%%%%%%%%


\begin{frame}
\frametitle{Data Model Design Principle }
\red{Decide what is the limitation of this part what is in and what is out to be part of the appendix}\\
%http://infogoal.com/datawarehousing/data_modeling_basics.htm
%https://www.guru99.com/star-snowflake-data-warehousing.html#3
- facts, start schema, dimensional modeling techniques.\\
- Fact Tables and Dimension Tables. \\
- Multidimensional Model(Star, Snowflake, and Galaxy Schema).\\
- Support Roll Up, Drill Down, and Pivot Analysis\\
- Time Phased / Temporal Data \\
- Operational Logical and Physical Data Models\\
- Normalization and Denormalization\\
- Model Granularity : Level of Detail\\


\end{frame}

%%%%%%%%%%%%%%%%%%%%%%%%%%%%%%%%%%%%%%%%%%%%%%%%%%%%%%
\subsubsection{ETL Process}

\begin{frame}
\frametitle{What is ETL?}

\begin{itemize}[<+->]
	\item The ETL (Extraction, Transformation, Loading) is main core function for any data engineering (DWH) team.
	\item This team takes the delivered output from the previous stage (data modeling) and start to implement the mapping.
	\item The implementation of the ETL preferred to be unified across the team members and the organization unless there is a special case of license of capacity.
	
\end{itemize}		


\end{frame}
%%%%%%%%%%%%%%%%%%%%%%%%%%%%%%%%%%%%%%%%%%%%%%%%%%%%%%

\begin{frame}
\frametitle{ETL Characteristics}
%%https://www.timmitchell.net/etl-best-practices/
\begin{itemize}[<+->]
	\item Successful ETL design have the following characteristics:
		\begin{itemize}[<+->]
			\item [\faCheckSquareO] Maintainable.
			\item [\faCheckSquareO] Reusable.
			\item [\faCheckSquareO] Well-Performed.
			\item [\faCheckSquareO] Reliable.
			\item [\faCheckSquareO] Resilient.
			\item [\faCheckSquareO] Secure.
		\end{itemize}
\end{itemize}
\end{frame}
%%%%%%%%%%%%%%%%%%%%%%%%%%%%%%%%%%%%%%%%%%%%%%%%%%%%%%
%%%%%%%%%%%%%%%%%%%%%%%%%%%%%%%%%%%%%%%%%%%%%%%%%%%%%%

\begin{frame}
\frametitle{ETL Best Practice}
%%https://www.timmitchell.net/etl-best-practices/
\begin{itemize}[<+->]
	\item To implement the previous characteristics you need to have the following:
	\begin{itemize}[<+->]
		\item [\faCheckSquareO] Logging.
		
		\item [\faCheckSquareO] Auditing.
		
		\item [\faCheckSquareO] Data Lineage.
		
		\item [\faCheckSquareO] Modularity.
		
		\item [\faCheckSquareO] Atomicity.
		
		\item [\faCheckSquareO] Error Handling.
		
		\item [\faCheckSquareO] Managing Bad Data (Rejection Handling).		
	\end{itemize}
\end{itemize}
\end{frame}

%%%%%%%%%%%%%%%%%%%%%%%%%%%%%%%%%%%%%%%%%%%%%%%%%%%%%%
%%%%%%%%%%%%%%%%%%%%%%%%%%%%%%%%%%%%%%%%%%%%%%%%%%%%%%

\begin{frame}
\frametitle{ETL Logging}
%%https://www.timmitchell.net/etl-best-practices/
\begin{itemize}[<+->]
	\item Logging
	\begin{itemize}[<+->]
		\item  Logging.
		\item  Logging.
		\item  Logging.
		

	\end{itemize}
\end{itemize}
\end{frame}

%%%%%%%%%%%%%%%%%%%%%%%%%%%%%%%%%%%%%%%%%%%%%%%%%%%%%%
%%%%%%%%%%%%%%%%%%%%%%%%%%%%%%%%%%%%%%%%%%%%%%%%%%%%%%

\begin{frame}
\frametitle{ETL Auditing}
%%https://www.timmitchell.net/etl-best-practices/
\begin{itemize}[<+->]
	\item Logging
	\begin{itemize}[<+->]
		\item  Logging.
		\item  Logging.
		\item  Logging.
		
		
	\end{itemize}
\end{itemize}
\end{frame}

%%%%%%%%%%%%%%%%%%%%%%%%%%%%%%%%%%%%%%%%%%%%%%%%%%%%%%
%%%%%%%%%%%%%%%%%%%%%%%%%%%%%%%%%%%%%%%%%%%%%%%%%%%%%%

\begin{frame}
\frametitle{ETL Data Lineage}
%%https://www.timmitchell.net/etl-best-practices/
\begin{itemize}[<+->]
	\item Logging
	\begin{itemize}[<+->]
		\item  Logging.
		\item  Logging.
		\item  Logging.
		
		
	\end{itemize}
\end{itemize}
\end{frame}

%%%%%%%%%%%%%%%%%%%%%%%%%%%%%%%%%%%%%%%%%%%%%%%%%%%%%%
%%%%%%%%%%%%%%%%%%%%%%%%%%%%%%%%%%%%%%%%%%%%%%%%%%%%%%

\begin{frame}
\frametitle{ETL Modularity}
%%https://www.timmitchell.net/etl-best-practices/
\begin{itemize}[<+->]
	\item Logging
	\begin{itemize}[<+->]
		\item  Logging.
		\item  Logging.
		\item  Logging.
		
		
	\end{itemize}
\end{itemize}
\end{frame}

%%%%%%%%%%%%%%%%%%%%%%%%%%%%%%%%%%%%%%%%%%%%%%%%%%%%%%
%%%%%%%%%%%%%%%%%%%%%%%%%%%%%%%%%%%%%%%%%%%%%%%%%%%%%%

\begin{frame}
\frametitle{ETL Atomicity}
%%https://www.timmitchell.net/etl-best-practices/
\begin{itemize}[<+->]
	\item Logging
	\begin{itemize}[<+->]
		\item  Logging.
		\item  Logging.
		\item  Logging.
		
		
	\end{itemize}
\end{itemize}
\end{frame}

%%%%%%%%%%%%%%%%%%%%%%%%%%%%%%%%%%%%%%%%%%%%%%%%%%%%%%
%%%%%%%%%%%%%%%%%%%%%%%%%%%%%%%%%%%%%%%%%%%%%%%%%%%%%%

\begin{frame}
\frametitle{ETL Error Handling}
%%https://www.timmitchell.net/etl-best-practices/
\begin{itemize}[<+->]
	\item Logging
	\begin{itemize}[<+->]
		\item  Logging.
		\item  Logging.
		\item  Logging.
		
		
	\end{itemize}
\end{itemize}
\end{frame}

%%%%%%%%%%%%%%%%%%%%%%%%%%%%%%%%%%%%%%%%%%%%%%%%%%%%%%
%%%%%%%%%%%%%%%%%%%%%%%%%%%%%%%%%%%%%%%%%%%%%%%%%%%%%%

\begin{frame}
\frametitle{ETL Rejection Handling}
%%https://www.timmitchell.net/etl-best-practices/
\begin{itemize}[<+->]
	\item Logging
	\begin{itemize}[<+->]
		\item  Logging.
		\item  Logging.
		\item  Logging.
		
		
	\end{itemize}
\end{itemize}
\end{frame}

%%%%%%%%%%%%%%%%%%%%%%%%%%%%%%%%%%%%%%%%%%%%%%%%%%%%%%

\begin{frame}
\frametitle{ETL vs ELT When? Why?}
\end{frame}
%%%%%%%%%%%%%%%%%%%%%%%%%%%%%%%%%%%%%%%%%%%%%%%%%%%%%%

%%%%%%%%%%%%%%%%%%%%%%%%%%%%%%%%%%%%%%%%%%%%%%%%%%%%%%
\subsubsection{Storage layer}

\begin{frame}
\frametitle{Storage layer}
\end{frame}
%%%%%%%%%%%%%%%%%%%%%%%%%%%%%%%%%%%%%%%%%%%%%%%%%%%%%%
%%%%%%%%%%%%%%%%%%%%%%%%%%%%%%%%%%%%%%%%%%%%%%%%%%%%%%
\subsubsection{Logical layer}

\begin{frame}
\frametitle{Logical layer}
\end{frame}
%%%%%%%%%%%%%%%%%%%%%%%%%%%%%%%%%%%%%%%%%%%%%%%%%%%%%%

%%%%%%%%%%%%%%%%%%%%%%%%%%%%%%%%%%%%%%%%%%%%%%%%%%%%%%
\subsubsection{Reporting (UI) layer}

\begin{frame}
\frametitle{Reporting (UI) layer}
\end{frame}
%%%%%%%%%%%%%%%%%%%%%%%%%%%%%%%%%%%%%%%%%%%%%%%%%%%%%%

%%%%%%%%%%%%%%%%%%%%%%%%%%%%%%%%%%%%%%%%%%%%%%%%%%%%%%
\subsubsection{Metadata layer}

\begin{frame}
\frametitle{Metadata layer}
\end{frame}
%%%%%%%%%%%%%%%%%%%%%%%%%%%%%%%%%%%%%%%%%%%%%%%%%%%%%%

%%%%%%%%%%%%%%%%%%%%%%%%%%%%%%%%%%%%%%%%%%%%%%%%%%%%%%
\subsubsection{System operations layer}

\begin{frame}
\frametitle{System operations layer}
\end{frame}
%%%%%%%%%%%%%%%%%%%%%%%%%%%%%%%%%%%%%%%%%%%%%%%%%%%%%%



%%%%%%%%%%%%%%%%%%%%%%%%%%%%%%%%%%%%%%%%%%%%%%%%%%%%%%
\begin{frame}
\frametitle{DWH Architecture Overview}
There are mainly three types of Datawarehouse Architectures: -
%https://www.guru99.com/data-warehouse-architecture.html
\begin{wideitemize}
	\item Single-tier architecture.
	\item Two-tier architecture.
	\item Three-tier architecture.
\end{wideitemize}

\end{frame}


%%%%%%%%%%%%%%%%%%%%%%%%%%%%%%%%%%%%%%%%%%%%%%%%%%%%%%
%%%%%%%%%%%%%%%%%%%%%%%%%%%%%%%%%%%%%%%%%%%%%%%%%%%%%%
\subsection{File Formats}

\begin{frame}
\frametitle{\subsecname}
\begin{itemize}[<+->]
	\item Any Big Data solution working based distributed systems.
	\item What is distributed systems in brief?
\end{itemize}
\end{frame}
%%%%%%%%%%%%%%%%%%%%%%%%%%%%%%%%%%%%%%%%%%%%%%%%%%%%%%
\subsection{Data Encoding and Formats}

\begin{frame}
\frametitle{\subsecname}
\begin{itemize}[<+->]
	\item Any Big Data solution working based distributed systems.
	\item What is distributed systems in brief?
\end{itemize}
\end{frame}
%%%%%%%%%%%%%%%%%%%%%%%%%%%%%%%%%%%%%%%%%%%%%%%%%%%%%%
\subsection{Data Compression Technique}

\begin{frame}
\frametitle{\subsecname}
\begin{itemize}[<+->]
\item Any Big Data solution working based distributed systems.
\item What is distributed systems in brief?
\end{itemize}
\end{frame}
%%%%%%%%%%%%%%%%%%%%%%%%%%%%%%%%%%%%%%%%%%%%%%%%%%%%%%%%%%%%%%%%%%%%%%%%%%%

\subsection{Data Archiving and Retention}
\begin{frame}
\frametitle{\subsecname}
\begin{itemize}[<+->]
	\item some details about hot vs cold storage,
\end{itemize}
\end{frame}
%%%%%%%%%%%%%%%%%%%%%%%%%%%%%%%%%%%%%%%%%%%%%%%%%%%%%%%%%%%%%%%%%%%%%%%%%%%

%%%%%%%%%%%%%%%%%%%%%%%%%%%%%%%%%%%%%%%%%%%%%%%%%%%%%%

\subsection{DWH On Cloud}



\subsection{Further Readings and Assignment}


%%%%%%%%%%%%%%%%%%%%%%%%%%%%%%%%%%%%%%%%%%%%%%%%%%%%%%%%%%%%%%%%%%%%%%%%%%%
%%% Local Variables:
%%% mode: latex
%%% TeX-master: "../main"
%%% TeX-engine: xetex
%%% End:
