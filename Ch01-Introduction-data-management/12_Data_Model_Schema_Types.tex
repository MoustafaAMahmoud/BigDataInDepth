%%%%%%%%%%%%%%%%%%%%%%%%%%%%%%%%%%%%%%%%%%%%%%%%%%%%%%
\VideoClassification[column=1, colour=blue]
%%%%%%%%%%%%%%%%%%%%%%%%%%%%%%%%%%%%%%%%%%%%%%%%%%%%%%

\midTitle{Schema Types}
\begin{frame}
\frametitle{Schema Types}
	\begin{itemize}[<+->]
		\item Star schema.
		\item Snowflake schema.
	\end{itemize}
	
\end{frame}
%%%%%%%%%%%%%%%%%%%%%%%%%%%%%%%%%%%%%%%%%%%%%%%%%%%%%%
%%%%%%%%%%%%%%%%%%%%%%%%%%%%%%%%%%%%%%%%%%%%%%%%%%%%%%
\midTitle{Schema Types: Star schema}
\begin{frame}
    \frametitle{Schema Types: Star schema}
			\begin{itemize}
				\item \textbf{Star Schema} the center of the star can have one fact table and a number of associated dimension tables. It is known as star schema as its structure resembles a star. The star schema is the simplest type of Data Warehouse schema. It is also known as Star Join Schema and is optimized for querying large data sets.
			\end{itemize}
%     primary dim & secondary 2 level only 
    %snawflake multi dim nationality for example egypt
    %it will scan the nationality first and it will be fortign key with forign key inner join
\end{frame}
%%%%%%%%%%%%%%%%%%%%%%%%%%%%%%%%%%%%%%%%%%%%%%%%%%%%%%
\begin{frame}
\frametitle{Schema Types: Star schema}
\begin{itemize}
	\item Characteristics of Star Schema:
	\begin{itemize}
		\item Every dimension in a star schema is represented with the only one-dimension table.
		\item The dimension table should contain the set of attributes.
		\item The dimension table is joined to the fact table using a foreign key
		\item The dimension table are not joined to each other
		\item Fact table would contain key and measure
		\item The Star schema is easy to understand and provides optimal disk usage.
		\item The dimension tables are not normalized. For instance, in the above figure, Country\_ID does not have Country lookup table as an OLTP design would have.
		\item The schema is widely supported by BI Tools
	\end{itemize}
\end{itemize}
\end{frame}
%%%%%%%%%%%%%%%%%%%%%%%%%%%%%%%%%%%%%%%%%%%%%%%%%%%%%%
\begin{frame}
\frametitle{Schema Types: Star schema (Example)}


\resizebox{\columnwidth}{!}{%
\begin{tikzpicture}[every node/.style={font=\ttfamily}, node distance=1.4in,scale=.75, every node/.style={scale=0.75}]
%https://tex.stackexchange.com/questions/133754/creating-crows-foot-style-e-r-diagrams-rather-than-chen-style-ones
\matrix  [entity=Usage, entity anchor=Usage-id]  {
	\properties{
		id,
		cust-id (FK),
		cal-id (FK), 
		loc-id (FK),
		promo-id (FK),
		date-id (FK),
		TotalInCalls (agg),
		TotalOutCalls (agg),
		TotalAmount (agg)
	}
};


\matrix  [entity=CellLookup, above left=of Usage-id, entity anchor=CellLookup-id]  {
	\properties{
		id,
		celltype,
		vendorname,
		street,
		city,
		state,
		zip
	}
};
\matrix  [entity=Promotion, below left=of Usage-id,yshift=10ex, entity anchor=Promotion-id]  {
	\properties{
		id,
		promotype,
		promodesc,
		value,
		startdate,
		enddate
	}
};

\matrix [entity=CustomerProfile, below right=of Usage-id,yshift=10ex, entity anchor=CustomerProfile-id]  {
	\properties{
		id, 
		gender, 
		age, 
		nationality,
		firstname,
		lastname
	}
};


\matrix  [entity=Calendar, above right=of Usage-id, entity anchor=Calendar-id]  {
	\properties{
		id,
		date,
		day,
		week,
		month,
		qtr,
		year
	}
};

\draw [one to one] (Usage-id)  to (CustomerProfile-id);
\draw [one to one] (Usage-id)  to (Calendar-id);
\draw [one to one] (Usage-id)  to (CellLookup-id);
\draw [one to one] (Usage-id)  to (Promotion-id);

\end{tikzpicture}
}
%%%%%%%%%%%%%%%%%%%%%%%%%%%%%%%%%%%%%%%%%%%%%%%%%%%%%%%%%%%%%%%%%%%%%%%%%%%
%%% Local Variables:
%%% mode: latex
%%% TeX-master: "../../main.tex"
% !TeX root = ../../main.tex
%%% TeX-engine: xetex
%%% End:

\end{frame}
%%%%%%%%%%%%%%%%%%%%%%%%%%%%%%%%%%%%%%%%%%%%%%%%%%%%%%%
\midTitle{Schema Types: Snowflake  schema}
\begin{frame}
\frametitle{Schema Types: Snowflake  schema}
	\begin{itemize}
		\item \textbf{Snowflake Schema} is an extension of a Star Schema, and it adds additional dimensions. It is called snowflake because its diagram resembles a Snowflake. The dimension tables are normalized which splits data into additional tables. In the following example, Country is further normalized into an individual table.
	\end{itemize}
\end{frame}

%%%%%%%%%%%%%%%%%%%%%%%%%%%%%%%%%%%%%%%%%%%%%%%%%%%%%%%
\begin{frame}
\frametitle{Schema Types: Snowflake  schema}
\begin{itemize}
	\item Characteristics of Snowflake Schema:
	\begin{itemize}
		\item The main benefit of the snowflake schema it uses smaller disk space.
		\item Easier to implement a dimension is added to the Schema
		\item Due to multiple tables query performance is reduced
		\item The primary challenge that you will face while using the snowflake Schema is that you need to perform more maintenance efforts because of the more lookup tables.
	\end{itemize}
\end{itemize}
\end{frame}

%%%%%%%%%%%%%%%%%%%%%%%%%%%%%%%%%%%%%%%%%%%%%%%%%%%%%%
\begin{frame}
	\frametitle{Schema Types: Star schema (Example)}
	
	\begin{tikzpicture}[every node/.style={font=\ttfamily}, node distance=1.4in,scale=.75, every node/.style={scale=0.75}]
%https://tex.stackexchange.com/questions/133754/creating-crows-foot-style-e-r-diagrams-rather-than-chen-style-ones
\matrix  [entity=Usage, entity anchor=Usage-id]  {
	\properties{
		id,
		cust-id (FK),
		cal-id (FK), 
		loc-id (FK),
		promo-id (FK),
		date-id (FK),
		TotalInCalls (agg),
		TotalOutCalls (agg),
		TotalAmount (agg)
	}
};


\matrix [entity=CustomerProfile, below right=of Usage-id,yshift=10ex, entity anchor=CustomerProfile-id]  {
	\properties{
		id, 
		gender, 
		age, 
		NationalityID,
		firstname,
		lastname
	}
};

\matrix [entity=Nationality, below right=of CustomerProfile-id,yshift=10ex, entity anchor=Nationality-id]  {
	\properties{
		id, 
		name
	}
};


\matrix  [entity=Calendar, above right=of Usage-id, entity anchor=Calendar-id]  {
	\properties{
		id,
		date,
		day,
		week,
		month,
		qtr,
		year
	}
};

\draw [one to one] (Usage-id)  to (CustomerProfile-id);
\draw [one to one] (Usage-id)  to (Calendar-id);
\draw [one to one] (CustomerProfile-id)  to (Nationality-id);

\end{tikzpicture}
%%%%%%%%%%%%%%%%%%%%%%%%%%%%%%%%%%%%%%%%%%%%%%%%%%%%%%%%%%%%%%%%%%%%%%%%%%%
%%% Local Variables:
%%% mode: latex
%%% TeX-master: "../../main.tex"
% !TeX root = ../../main.tex
%%% TeX-engine: xetex
%%% End:

\end{frame}

%%%%%%%%%%%%%%%%%%%%%%%%%%%%%%%%%%%%%%%%%%%%%%%%%%%%%%%
\begin{frame}
\frametitle{Star Vs Snowflake Schema}

	\begin{tabular}{| l | l |}
		\hline
		Star & Snowflake\\
		\hline
		 \makecell{Hierarchies of the dimensions\\are stored in the dimensional table} &  \makecell{Hierarchies are divided\\ into separate tables}\\
		 		\hline
		\makecell{Fact table surrounded\\ by dimension tables} & 
		\makecell{One fact table surrounded by\\multi-level of dimension tables} \\
				\hline
		 \makecell{Single join creates the\\relation between fact \& dimensions}
		 & \makecell{Requires many joins\\ to fetch the data}\\
 		\hline
		Simple DB Design & Very Complex DB Design\\
		\hline
		De-normalized Data structure & Normalized Data Structure\\
		\hline
		High level of Data redundancy & Very low-level data redundancy\\
		\hline
		Cube processing is faster & Cube processing is slow\\% (complex join)
		\hline
	\end{tabular}


\end{frame}





%%%%%%%%%%%%%%%%%%%%%%%%%%%%%%%%%%%%%%%%%%%%%%%%%%%%%%%%%%%%%%%%%%%%%%%%%%%%
%%% Local Variables:
%%% mode: latex
%%% TeX-master: "../main"
% !TeX root = ../main.tex
%%% TeX-engine: xetex
%%% End: