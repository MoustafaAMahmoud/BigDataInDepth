%%%%%%%%%%%%%%%%%%%%%%%%%%%%%%%%%%%%%%%%%%%%%%%%%%%%%%%
\VideoClassification[column=1, colour=blue]
%%%%%%%%%%%%%%%%%%%%%%%%%%%%%%%%%%%%%%%%%%%%%%%%%%%%%%
\midTitle{Dimensions Types: Shrunken Rollup Dimensions}
\begin{frame}
    \frametitle{Shrunken Rollup Dimensions}%منكمش
    \begin{itemize}[<+->]    	
		\item Shrunken Rollup dimension is used for developing aggregate (higher level of summary) fact tables. 
		\item It required that the data model has a lower level of granularity.
	\end{itemize}
		\begin{example}
		    \begin{itemize}[<+->]    	
			\item We have a daily usage fact table, and we need to have a higher level of monthly usage. So, we use the monthly dimension to get a summary of the daily.
			\item We have a daily usage fact table aggregated on area-id, and we need to create another summary table aggregated based on city id. So, the new grain level here is the new dimension for the city.
			\end{itemize}
		\end{example}
\end{frame}
%%%%%%%%%%%%%%%%%%%%%%%%%%%%%%%%%%%%%%%%%%%%%%%%%%%%%%
\begin{frame}
	\frametitle{Shrunken Rollup Dimensions Cont.}

	\begin{table}[t]
		\centering
		\sffamily
		\begin{tabular}{|l | a | l |}
			\hline
			OrderDate & AreaID & TotalOrders\\
			\hline
			\hline
			123456789 & 123  & 20\\
			123456789 & 123  & 30\\
			\hline
			\hline
			123456789 & 678  & 10\\
			123456789 & 678  & 12\\
			\hline
		\end{tabular}

	\end{table}
	
	\begin{table}[t]
		\centering
		\sffamily
		\begin{tabular}{|a | l | l |}
			\hline
			AreaID & AreaName & CityID\\
			\hline
			\hline			
			123 & Al-Matareya  & 1\\
			678 & Ain shams    & 1\\
			\hline
		\end{tabular}
		\quad
		\begin{tabular}{|a | l|}
			\hline
			CityID & CityName\\
			\hline
			\hline			
			1 & Cairo \\
			\hline
		\end{tabular}
	\end{table}

	\begin{table}[t]
	\centering
	\sffamily
		\begin{tabular}{|l | a | l |}
			\hline
			OrderDate & CityID & TotalOrders\\
			\hline
			\hline		
			123456789 & 1  & 72\\
			\hline
		\end{tabular}
	\end{table}
\end{frame}




%%%%%%%%%%%%%%%%%%%%%%%%%%%%%%%%%%%%%%%%%%%%%%%%%%%%%%%%%%%%%%%%%%%%%%%%%%%%
%%% Local Variables:
%%% mode: latex
%%% TeX-master: "../../main"
% !TeX root = ../../main.tex
%%% TeX-engine: xetex
%%% End: