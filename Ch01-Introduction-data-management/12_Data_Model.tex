\VideoClassification[column=1, colour=red]
\subsubsection{Data Modeling}


\begin{frame}
    \frametitle{Data Modeling Objective}
    \begin{itemize}[<+->]
        \item Explain what data modeling is and its roles?
        \item Be aware of its importance.
        \item Explore different types of data modeling.
        \item \blue{We target to explain the main components and types, and for more details, it could be found in the appendix videos.}.
    \end{itemize}

\end{frame}

%%%%%%%%%%%%%%%%%%%%%%%%%%%%%%%%%%%%%%%%%%%%%%%%%%%%%%

\begin{frame}
    \frametitle{What is data model?}
	The data model
    \begin{itemize}[<+->]
        \item is An abstract model that organizes elements of data.
        \item It describes the objects, entities, and data structure properties, semantic, and constraint.
        \item It formalizes the relationship between entities.
        \item It describes how the application (report) API data manipulation.
        \item It describes the conceptual design of a business or an application with its flow, logic, semantic information (rules), and how things are done.
        \item It refers to a set of concepts used in defining such as entities, attributes, relations, or tables.
    \end{itemize}
\end{frame}

%%%%%%%%%%%%%%%%%%%%%%%%%%%%%%%%%%%%%%%%%%%%%%%%%%%%%%
\begin{frame}
    \frametitle{What is data model?}

    \begin{columns}

        \column{0.4\textwidth}
        Data model is not
        \begin{itemize}[<+->]
            \item a science.
            \item a static design for each organization.
            \item a type of database.
            \item a new invention which needs to be done for each project.
            %ex: sldm teradata model
        \end{itemize}


        \column{0.45\textwidth}
        Data model is
        \begin{itemize}[<+->]
            \item a general concept that leads to build full architecture.
            \item an engineering design practices.
            \item different based on the use case and the database type.
            \item customizable, and we can utilize some of the ready built architecture.
            \item affecting information reporting performance.
        \end{itemize}

        \column{0.2\textwidth}
    \end{columns}

\end{frame}

%%%%%%%%%%%%%%%%%%%%%%%%%%%%%%%%%%%%%%%%%%%%%%%%%%%%%%

\begin{frame}
    \frametitle{What is data model?}
    The data model is
    \begin{itemize}[<+->]
        \item The first part before starting integration with any new source system.
        \item The connection layer between business requirements and technical design.
        \item It is also the translation between logical and physical layer.
        \item It is unified across all systems and has the same patterns and practices.
        \item It engaged with any source systems integration from the early stages.
        \item \blue{This stage output is a data model design document or mapping sheet}.
    \end{itemize}
\end{frame}

%%%%%%%%%%%%%%%%%%%%%%%%%%%%%%%%%%%%%%%%%%%%%%%%%%%%%%

\begin{frame}
    \frametitle{Why does the data model are important?}
    \begin{wideitemize}
        \item Data models are currently affecting software design.
        \item It decides how engineers think about the problem they are solving.
    \end{wideitemize}
\end{frame}

%%%%%%%%%%%%%%%%%%%%%%%%%%%%%%%%%%%%%%%%%%%%%%%%%%%%%%
\begin{frame}
    \frametitle{Data Model Design vs Implementation}
    %Replace it by photo
    \begin{itemize}[<+->]
        \item If you need to build a home, so, how do we design this home?
        \begin{itemize}[<+->]
            \item Determine if the home is one level or multi-level and decide main bedrooms and bathrooms for each floor. (User needs)
            \item Hire an architect to put the architecture in more detailed way \forexample the size for each room, the distribution of the wires, where the plumbing fixtures will be placed, etc. (Architecture phase)
            \item Decide the decorations, colors for each room, carpets, etc.
        \end{itemize}
        \item What do we do for the implementation?
        \begin{itemize}[<+->]
            \item Hire a contractor to build (implement the design) the home.
            \item This phase implement the design, but it also includes some detail related to the real way to build the tools and the material (Physical Design).
        \end{itemize}
    \end{itemize}
\end{frame}

%%%%%%%%%%%%%%%%%%%%%%%%%%%%%%%%%%%%%%%%%%%%%%%%%%%%%%
\midTitle{Data Model: Elements of Data Model}
\begin{frame}
    \frametitle{Elements of Data Model}
    \begin{description}[<+->]
        \item[Facts] are the measurements/metrics or facts from the business process \forexample (Telecom industry, measurement would be the count of daily/hourly usage per customer). We could consider facts as the source of reporting for the business.
        %measure and ids when,where
        \item[Dimensions] provide the context surrounding a business process event. In simple terms, they give who, what, where the fact, \forexample (Telecom industry, for the fact daily usage, dimensions would be customer\_id, location\_id).
        
        \item[Attributes] are the various characteristics of the dimension. In the previous examples, the attributes can be customer details (from customer\_id get the gender, age, nationality, etc.).
        %Attributes are used to search, filter, or classify facts. Dimension Tables contain Attributes
        %	Facts include dimensions, dimensions include attributes
    \end{description}
\end{frame}

%%%%%%%%%%%%%%%%%%%%%%%%%%%%%%%%%%%%%%%%%%%%%%%%%%%%%%
\begin{frame}
    \frametitle{Elements of Dimensional Data Model}
    \begin{description}[<+->]
        \item[Fact Table] is a primary table in a dimensional model. A Fact Table contains (Measurements/facts and Foreign key to \textit{dimension table}). It located at the center of a star or snowflake schema and surrounded by dimensions.
        %	contains only ids customer, date, location ids  to get this ids go to the diem
        %granuality for levels / agg up down
        \item[Dimension table] contains dimensions of a fact and business reference data. They are joined to fact table via a foreign key. Dimension tables are de-normalized tables. It connected to the fact table and located at the edges of the star or snowflake schema.

    \end{description}
\end{frame}

%%%%%%%%%%%%%%%%%%%%%%%%%%%%%%%%%%%%%%%%%%%%%%%%%%%%%%
\begin{frame}
    \frametitle{Example of Data Model}

    
\resizebox{\columnwidth}{!}{%
\begin{tikzpicture}[every node/.style={font=\ttfamily}, node distance=1.4in,scale=.75, every node/.style={scale=0.75}]
%https://tex.stackexchange.com/questions/133754/creating-crows-foot-style-e-r-diagrams-rather-than-chen-style-ones
\matrix  [entity=Usage, entity anchor=Usage-id]  {
	\properties{
		id,
		cust-id (FK),
		cal-id (FK), 
		loc-id (FK),
		promo-id (FK),
		date-id (FK),
		TotalInCalls (agg),
		TotalOutCalls (agg),
		TotalAmount (agg)
	}
};


\matrix  [entity=CellLookup, above left=of Usage-id, entity anchor=CellLookup-id]  {
	\properties{
		id,
		celltype,
		vendorname,
		street,
		city,
		state,
		zip
	}
};
\matrix  [entity=Promotion, below left=of Usage-id,yshift=10ex, entity anchor=Promotion-id]  {
	\properties{
		id,
		promotype,
		promodesc,
		value,
		startdate,
		enddate
	}
};

\matrix [entity=CustomerProfile, below right=of Usage-id,yshift=10ex, entity anchor=CustomerProfile-id]  {
	\properties{
		id, 
		gender, 
		age, 
		nationality,
		firstname,
		lastname
	}
};


\matrix  [entity=Calendar, above right=of Usage-id, entity anchor=Calendar-id]  {
	\properties{
		id,
		date,
		day,
		week,
		month,
		qtr,
		year
	}
};

\draw [one to one] (Usage-id)  to (CustomerProfile-id);
\draw [one to one] (Usage-id)  to (Calendar-id);
\draw [one to one] (Usage-id)  to (CellLookup-id);
\draw [one to one] (Usage-id)  to (Promotion-id);

\end{tikzpicture}
}
%%%%%%%%%%%%%%%%%%%%%%%%%%%%%%%%%%%%%%%%%%%%%%%%%%%%%%%%%%%%%%%%%%%%%%%%%%%
%%% Local Variables:
%%% mode: latex
%%% TeX-master: "../../main.tex"
% !TeX root = ../../main.tex
%%% TeX-engine: xetex
%%% End:

\end{frame}

%%%%%%%%%%%%%%%%%%%%%%%%%%%%%%%%%%%%%%%%%%%%%%%%%%%%%%
\midTitle{Data Model: Elements of Data Model}
\begin{frame}
	\frametitle{Elements of Data Model}
		Dimensional model life cycle:
	    \begin{itemize}[<+->]
			\item Gathering Requirements (Source Driven, Business/User Driven).
			\item Identify granularity of the facts
			\item Identify the dimensions
			\item Identify the facts
	    \end{itemize}	
\end{frame}
%%%%%%%%%%%%%%%%%%%%%%%%%%%%%%%%%%%%%%%%%%%%%%%%%%%%%%

\begin{frame}
\frametitle{Dimensions Types}
	\begin{enumerate}[<+->]
		\item Conformed Dimension.
		\item Degenerate Dimension.
		\item Junk Dimension (Garbage Dimension).
		\item Role-Playing Dimension.
		\item Outrigger Dimension.
		\item Snowflake Dimension.
		\item Shrunken Rollup Dimension.
		\item Swappable Dimension.
		\item Slowly changing Dimension.
		\item Fast Changing Dimension (Mini Dimension).
		\item Heterogenous Dimensions
		\item Multi-valued dimensions
	\end{enumerate}
\end{frame}





%%%%%%%%%%%%%%%%%%%%%%%%%%%%%%%%%%%%%%%%%%%%%%%%%%%%%%%%%%%%%%%%%%%%%%%%%%%%
%%% Local Variables:
%%% mode: latex
%%% TeX-master: "../main"
% !TeX root = ../main.tex
%%% TeX-engine: xetex
%%% End: