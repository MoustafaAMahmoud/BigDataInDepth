\documentclass{beamer}
%%%%%%%%%%%%%%%%%%%%%%%%%%%%%%%%%%%%%%%%%%%%%%%%%%%%%%%%%%%%%%%%%%%%%%%%%%%%%%%%%%%%%%%%%%%%%%%%
%
% Packages Definition
%
%%%%%%%%%%%%%%%%%%%%%%%%%%%%%%%%%%%%%%%%%%%%%%%%%%%%%%%%%%%%%%%%%%%%%%%%%%%%%%%%%%%%%%%%%%%%%%%%
\usepackage[toc,page]{appendix}

\usepackage{listings}
\usepackage{hyperref}
\usepackage{pgf,pgfpages}
\usepackage{graphicx}
\usepackage{units}
\usepackage[utf8]{inputenc}
\usepackage{pstricks}
\usepackage{multicol}
\usepackage{xspace}

%\usepackage{minted}
%\usepackage{scrhack}

\usepackage[utf8]{inputenc}
\usepackage{utopia} %font utopia imported
\usetheme{Madrid}
\usecolortheme{default}


%%%%%%%%%%%%%%%%%%%%%%%%%%%%%%%%%%%%%%%%%%%%%%%%%%%%%%%%%%%%%%%%%%%%%%%%%%%%%%%%%%%%%%%%%%%%%%%%
%
% To Add hyperlink setup
%
%%%%%%%%%%%%%%%%%%%%%%%%%%%%%%%%%%%%%%%%%%%%%%%%%%%%%%%%%%%%%%%%%%%%%%%%%%%%%%%%%%%%%%%%%%%%%%%%

%\hypersetup{colorlinks=false,
%	linkbordercolor=red,
%	linkcolor=green,
%	pdfborderstyle={/S/B/W 1}
%}


%%%%%%%%%%%%%%%%%%%%%%%%%%%%%%%%%%%%%%%%%%%%%%%%%%%%%%%%%%%%%%%%%%%%%%%%%%%%%%%%%%%%%%%%%%%%%%%%
% 
% To add sections numbers to table of content
%
%%%%%%%%%%%%%%%%%%%%%%%%%%%%%%%%%%%%%%%%%%%%%%%%%%%%%%%%%%%%%%%%%%%%%%%%%%%%%%%%%%%%%%%%%%%%%%%%


%\setbeamertemplate{section in toc}[sections numbered]
%\setbeamercolor{section in toc}{fg=blue}
%
%\setbeamertemplate{subsection in toc}[subsections numbered]
%\setbeamercolor{subsection in toc}{fg=blue}
%
%\defbeamertemplate{subsubsection in toc}{subsubsections numbered}
%{\leavevmode\leftskip=3em%
%	\rlap{\hskip-3em\inserttocsectionnumber.\inserttocsubsectionnumber.\inserttocsubsubsectionnumber}%
%	\inserttocsubsubsection\par}
%
%\setbeamertemplate{subsubsection in toc}[subsubsections numbered]
%\setbeamercolor{subsubsection in toc}{fg=blue}


%%%%%%%%%%%%%%%%%%%%%%%%%%%%%%%%%%%%%%%%%%%%%%%%%%%%%%%%%%%%%%%%%%%%%%%%%%%%%%%%%%%%%%%%%%%%%%%%
% 
% To define sime custom colors
%
%%%%%%%%%%%%%%%%%%%%%%%%%%%%%%%%%%%%%%%%%%%%%%%%%%%%%%%%%%%%%%%%%%%%%%%%%%%%%%%%%%%%%%%%%%%%%%%%


%\definecolor{uibred}{RGB}{170, 0, 0}
%\definecolor{uibblue}{RGB}{0, 84, 115}
%\definecolor{uibgreen}{RGB}{119, 175, 0}
%\definecolor{uibgreen}{RGB}{50, 105, 0}
\definecolor{uiborange}{RGB}{217, 89, 0}
\definecolor{MyGray}{rgb}{.9, .9, .9}
\definecolor{myorange}{rgb}{1.0,0.4,0}
\definecolor{mygreen}{rgb}{0,0.8,0.6}
\definecolor{dkgreen}{rgb}{0,0.6,0}
\definecolor{gray}{rgb}{0.5,0.5,0.5}
\definecolor{mauve}{rgb}{0.58,0,0.82}



%%%%%%%%%%%%%%%%%%%%%%%%%%%%%%%%%%%%%%%%%%%%%%%%%%%%%%%%%%%%%%%%%%%%%%%%%%%%%%%%%%%%%%%%%%%%%%%%



\beamertemplatenavigationsymbolsempty

%====================================================
%-------------Macro definitions go here--------------
%====================================================

%
% Differentials
%
\newcommand{\tdiff}[2]{\ensuremath{\frac{\mathrm{d}#2}}{\mathrm{d}{#1}}}
\newcommand{\tdifforder}[3]{\ensuremath{\frac{\mathrm{d}^{#2}#3}{\mathrm{d}{#1}^{#2}}}}
\newcommand{\pdiff}[2]{\ensuremath{\frac{\partial#2 }{\partial#1}}}
\newcommand{\pdifforder}[3]{\ensuremath{\frac{\partial^{#2}#3}{\partial{#1}^{#2}}}}

% bracket
\newcommand{\bra}[1]{\ensuremath{\left<#1\right|}}
\newcommand{\ket}[1]{\ensuremath{\left|#1\right>}}
\newcommand{\bracket}[2]{\ensuremath{ \left\langle #1 | #2 \right\rangle}}
\newcommand{\matelem}[3]{\ensuremath{ \left\langle #1 | #2 | #3 \right\rangle}}
\newcommand{\matr}[1]{\ensuremath{\mathbf{#1}}}
\newcommand{\vect}[1]{\ensuremath{\mathbf{#1}}}
\newcommand{\expectationvalue}[1]{\ensuremath{\left\langle #1 \right\rangle}}

\renewcommand{\imath}{\ensuremath{\mathrm{i}}}

%
% Linear algebra
%
\renewcommand{\vec}[1]{\ensuremath{\mathbf{#1}}}
\newcommand{\mat}[1]{\ensuremath{\mathbf{#1}}}
\newcommand{\tildemat}[1]{\ensuremath{\widetilde{\mat{#1}}}}

% Numerical analysis
\newcommand{\bigo}{\ensuremath{\mathcal{O}}}
\renewcommand{\Re}{\ensuremath{\mathrm{Re}}}
\renewcommand{\Im}{\ensuremath{\mathrm{Im}}}
\newcommand{\mathcol}[2]{{\color{#1}#2}}
%\newcommand{\red}[1]{\mathcol{uibred}{#1}}
%\newcommand{\blue}[1]{\mathcol{uibblue}{#1}}
%\newcommand{\green}[1]{\mathcol{uibgreen}{#1}}
\newcommand{\orange}[1]{\mathcol{uiborange}{#1}}
%
% Misc macros
%
\newcommand{\eref}[1]{~(\ref{#1})}
\renewcommand{\equiv}[0]{\ensuremath{:=}}
\newcommand{\etal}{\textit{et al. }}
\newcommand{\paperheader}[2]{\noindent\textbf{Paper #1}: \textit{#2}\\}
\newcommand{\paperitem}[3]{\noindent\textbf{Paper #1}: \textit{#2}\vspace{1em}\\\noindent #3\vspace{2em}}
\newcommand{\tfinal}{\ensuremath{T_{\text{f}}}}
\newcommand{\papernum}[1]{\textbf{#1}}
%\newcommand{\note}[1]{\colorbox{yellow}{#1}}
\newcommand{\paperref}[1]{Paper~\textbf{#1}}

%
% Code
%
\newcommand{\inlinename}[1]{\lstinline[basicstyle=\ttfamily,language=bash]{#1}}



%\Includeonlyframes{current}

\newcommand{\ShowGraphics}

%%%%%%%%%%%%%%%%%%%%%%%%%%%%%%%%%%%%%%%%%%%%%%%%%%%%%%%%%%%%%%%%%%%%%%%%%%%%%%%%%%%%%%%%%%%%%%%%
% 
% To Pass graphics paths
%
%%%%%%%%%%%%%%%%%%%%%%%%%%%%%%%%%%%%%%%%%%%%%%%%%%%%%%%%%%%%%%%%%%%%%%%%%%%%%%%%%%%%%%%%%%%%%%%%

\graphicspath{{./Ch00-CourseOverview/}}


%\defbeamertemplate{enumerate item}{mycircle}
%{
%  %\usebeamerfont*{item projected}%
%  %\usebeamercolor[bg]{item projected}%
%  \begin{pgfpicture}{0ex}{0ex}{1.5ex}{0ex}
%	%\pgfcircle[fill]{\pgfpoint{0pt}{.75ex}}{1.25ex}
%    \pgfbox[center,base]{\color{uibblue}\insertenumlabel.}
%  \end{pgfpicture}%
%}




%%%%%%%%%%%%%%%%%%%%%%%%%%%%%%%%%%%%%%%%%%%%%%%%%%%%%%%%%%%%%%%%%%%%%%%%%%%%%%%%%%%%%%%%%%%%%%%%
%
% To define Code Syntax Matlab,Scala
%
%%%%%%%%%%%%%%%%%%%%%%%%%%%%%%%%%%%%%%%%%%%%%%%%%%%%%%%%%%%%%%%%%%%%%%%%%%%%%%%%%%%%%%%%%%%%%%%%


%%%%%%%%%%%%%%%%%%%%%%%%%%%%%%%%%%%%%%%%%%%%%%%%%%%%%%%%%%%%%%%%%%%%%%%%%%%%%%%%%%%%%%%%%%%%%%%%
%
% Matlab
%
%%%%%%%%%%%%%%%%%%%%%%%%%%%%%%%%%%%%%%%%%%%%%%%%%%%%%%%%%%%%%%%%%%%%%%%%%%%%%%%%%%%%%%%%%%%%%%%%


%\lstloadlanguages{Matlab}
%\lstset{language=Matlab, xleftmargin=0pt, numberstyle=\tiny, numbersep=0pt, breaklines=true, breakindent=20pt, commentstyle=\scriptsize, frame=single, frameround=tttt}
%\renewcommand{\lstlistingname}{
%\textbf{\underline{Matlab Code}}}

%%%%%%%%%%%%%%%%%%%%%%%%%%%%%%%%%%%%%%%%%%%%%%%%%%%%%%%%%%%%%%%%%%%%%%%%%%%%%%%%%%%%%%%%%%%%%%%%
%
% Scala
%
%%%%%%%%%%%%%%%%%%%%%%%%%%%%%%%%%%%%%%%%%%%%%%%%%%%%%%%%%%%%%%%%%%%%%%%%%%%%%%%%%%%%%%%%%%%%%%%%


\lstdefinestyle{myScalastyle}{
  frame=tb,
  language=scala,
  aboveskip=3mm,
  belowskip=3mm,
  showstringspaces=false,
  columns=flexible,
  numbers=left,                   % where to put the line-numbers
  numberstyle=\tiny\color{gray},  % the style that is used for the line-numbers
  stepnumber=1,                   % the step between two line-numbers. If it's 1, each line will be numbered
  numbersep=5pt,                  % how far the line-numbers are from the code
  backgroundcolor=\color{white},  % choose the background color. You must add 
  keywordstyle=\color{blue},
  commentstyle=\color{mygreen},
%  stringstyle=\color{mauve},
  stringstyle=\color{myorange},
  frame=single,
  breaklines=true,
  breakatwhitespace=true,
  breakindent=20pt,
  tabsize=3,
  frameround=tttt,
  escapeinside={\%*}{*)},        % to add a comment within your code
  emph={count,take,textFile,filter,first,collect,mkString}, % Scala functions
  emphstyle={\color{mauve}},
  morekeywords ={val,sc},        % to add more keywords to the set  
}
\renewcommand{\lstlistingname}{\textbf{\underline{Scala Code}}}
\newcommand{\myhy}[2]{\hyperref[#1]{\color{green}\setulcolor{red}\ul{#2}}}

%\hypersetup{
%  urlbordercolor = red,
%  linkbordercolor = red,
%  linkbordercolor={0 0 1},
%  colorlinks   = true, %Colours links instead of ugly boxes
%  urlcolor     = blue, %Colour for external hyperlinks
%  linkcolor    = blue, %Colour of internal links
%  citecolor   = red %Colour of citations,
%}


%%%%%%%%%%%%%%%%%%%%%%%%%%%%%%%%%%%%%%%%%%%%%%%%%%%%%%%%%%%%%%%%%%%%%%%%%%%%%%%%%%%%%%%%%%%%%%%%



%------------------------------------------------------------
%This block of code defines the information to appear in the
%Title page
\title[Big Data Engineering In details] %optional
{Big Data Engineering In details}

\subtitle{From Beginner to Professional}

\author[Mostafa Alaa] % (optional)
%{Mostafa Alaa Mohamed\\ Senior Big Data Engineer \\ Email: \href{mailto: mustafa.alaa.mohamed@gmail.com}{mustafa.alaa.mohamed@gmail.com} \\ Linkedin: \href{https://www.linkedin.com/in/mostafa-alaa-5120615b/}{Mostafa Alaa} \inst{1}}
{Mostafa Alaa Mohamed \newline Senior Big Data Engineer \newline Email: \href{mailto: mustafa.alaa.mohamed@gmail.com}{mustafa.alaa.mohamed@gmail.com} \newline Linkedin: \href{https://www.linkedin.com/in/mostafa-alaa-5120615b/}{Mostafa Alaa} \inst{1}}


\institute[] % (optional)
{
  \inst{1}%
  Big Data \& Analytics Department, Epam Systems
}

\date[\today] % (optional)
{The Definitive Guide to Big Data Engineering Tasks}

%\logo{\includegraphics[height=1.5cm]{lion-logo.png}}

%End of title page configuration block
%------------------------------------------------------------



%------------------------------------------------------------
%The next block of commands puts the table of contents at the 
%beginning of each section and highlights the current section:
%
%\AtBeginSection[]
%{
%  \begin{frame}[allowframebreaks]
%    \frametitle{Table of Contents}
%    \tableofcontents[currentsection]
%  \end{frame}
%}
%
%\AtBeginSection[]
%  {
%    \ifnum \value{framenumber}>1
%      \begin{frame}[allowframebreaks]
%      \frametitle{Section Content}
%      \tableofcontents[currentsection]
%      \end{frame}
%    \else
%    \fi
%  }

\AtBeginSection[]{
  \begin{frame}
  \vfill
  \centering
  \begin{beamercolorbox}[sep=8pt,center,shadow=true,rounded=true]{title}
    \usebeamerfont{title}\secname\par%
  \end{beamercolorbox}
  \vfill
  \end{frame}
}

%------------------------------------------------------------


\begin{document}

%The next statement creates the title page.
\frame{\titlepage}


%---------------------------------------------------------
%This block of code is for the table of contents after
%the title page
\begin{frame}[allowframebreaks]
\frametitle{Table of Contents}
\tableofcontents
\end{frame}
%---------------------------------------------------------

%%%%%%%%%%%%%%%%%%%%%%%%%%%%%%%%%%%%%%%%%%%%%%%%%%%%%%%%%%%%%%%%%%%%%%%%%%%%%%%%%%%%%%%%%%%%%%%%%%
%
\section{Spark Basics}

%\begin{frame}
%\frametitle{Sample frame title}
%This is a text in second frame. For the sake of showing an example.
%
%\begin{itemize}
%    \item<1-> Text visible on slide 1
%    \item<2-> Text visible on slide 2
%    \item<3> Text visible on slides 3
%    \item<4-> Text visible on slide 4
%\end{itemize}
%\end{frame}

%%---------------------------------------------------------
%%Example of the \pause command
%\begin{frame}
%In this slide \pause
%
%the text will be partially visible \pause
%
%And finally everything will be there
%\end{frame}

%\section{Second section}
%
%%---------------------------------------------------------
%%Highlighting text
%\begin{frame}
%\frametitle{Sample frame title}
%
%In this slide, some important text will be
%\alert{highlighted} beause it's important.
%Please, don't abuse it.
%
%\begin{block}{Remark}
%Sample text
%\end{block}
%
%\begin{alertblock}{Important theorem}
%Sample text in red box
%\end{alertblock}
%
%\begin{examples}
%Sample text in green box. "Examples" is fixed as block title.
%\end{examples}
%\end{frame}
%%---------------------------------------------------------
%
%
%%---------------------------------------------------------
%%Two columns
%\begin{frame}
%\frametitle{Two-column slide}
%
%\begin{columns}
%
%\column{0.5\textwidth}
%This is a text in first column.
%$$E=mc^2$$
%\begin{itemize}
%\item First item
%\item Second item
%\end{itemize}
%
%\column{0.5\textwidth}
%This text will be in the second column
%and on a second tought this is a nice looking
%layout in some cases.
%\end{columns}
%\end{frame}
%%---------------------------------------------------------



\subsection{Introduction to Spark}
%%%%%%%%%%%%%%%%%%%%%%%%%%%%%%%%%%%%%%%%%%%%%%%%%%%%%%%%%%%%%%%%%%%%%%%%%%%
\begin{frame}
  \frametitle{\secname : \subsecname}
	\begin{itemize}[<+->]
		\item Any Big Data solution working based distributed systems.
		\item What is distributed systems in brief?
	\end{itemize}
\end{frame}

%%%%%%%%%%%%%%%%%%%%%%%%%%%%%%%%%%%%%%%%%%%%%%%%%%%%%%%%%%%%%%%%%%%%%%%%%%%
\subsection{Map-Reduce using Spark}
\begin{frame}
  \frametitle{\subsecname}
	\begin{itemize}[<+->]
		\item Any Big Data solution working based distributed systems.
		\item What is distributed systems in brief?
	\end{itemize}
\end{frame}

%%%%%%%%%%%%%%%%%%%%%%%%%%%%%%%%%%%%%%%%%%%%%%%%%%%%%%%%%%%%%%%%%%%%%%%%%%%
\section{Spark Programming using RDDs}
\subsection{Spark RDD}

\begin{frame}
  \frametitle{\subsecname}
	\begin{itemize}[<+->]
		\item Any Big Data solution working based distributed systems.
		\item What is distributed systems in brief?
	\end{itemize}
\end{frame}
%%%%%%%%%%%%%%%%%%%%%%%%%%%%%%%%%%%%%%%%%%%%%%%%%%%%%%%%%%%%%%%%%%%%%%%%%%%

\subsection{Spark Working With Key/Value Pairs}

\begin{frame}
  \frametitle{\subsecname}
	\begin{itemize}[<+->]
		\item Any Big Data solution working based distributed systems.
		\item What is distributed systems in brief?
	\end{itemize}
\end{frame}

%%%%%%%%%%%%%%%%%%%%%%%%%%%%%%%%%%%%%%%%%%%%%%%%%%%%%%%%%%%%%%%%%%%%%%%%%%%

\subsection{Why we need RDD in Real World Applications}

\begin{frame}
  \frametitle{\subsecname}
	\begin{itemize}[<+->]
		\item Any Big Data solution working based distributed systems.
		\item What is distributed systems in brief?
	\end{itemize}
\end{frame}


%%%%%%%%%%%%%%%%%%%%%%%%%%%%%%%%%%%%%%%%%%%%%%%%%%%%%%%%%%%%%%%%%%%%%%%%%%%
\section{Spark Datasets/Dataframe}

\subsection{Introduction to Datasets/Dataframe}

\begin{frame}
  \frametitle{\subsecname}
	\begin{itemize}[<+->]
		\item Any Big Data solution working based distributed systems.
		\item What is distributed systems in brief?
	\end{itemize}
\end{frame}


%%%%%%%%%%%%%%%%%%%%%%%%%%%%%%%%%%%%%%%%%%%%%%%%%%%%%%%%%%%%%%%%%%%%%%%%%%%

\subsection{Spark SQL}
\begin{frame}
  \frametitle{\subsecname}
	\begin{itemize}[<+->]
		\item Any Big Data solution working based distributed systems.
		\item What is distributed systems in brief?
	\end{itemize}
\end{frame}

%%%%%%%%%%%%%%%%%%%%%%%%%%%%%%%%%%%%%%%%%%%%%%%%%%%%%%%%%%%%%%%%%%%%%%%%%%%

\subsection{Dataframes/Datasets vs. RDDs}

\begin{frame}
  \frametitle{\subsecname}
	\begin{itemize}[<+->]
		\item Any Big Data solution working based distributed systems.
		\item What is distributed systems in brief?
	\end{itemize}
\end{frame}

%%%%%%%%%%%%%%%%%%%%%%%%%%%%%%%%%%%%%%%%%%%%%%%%%%%%%%%%%%%%%%%%%%%%%%%%%%%
\section{Spark on Production}

\subsection{Go Production Part 1}
\begin{frame}
  \frametitle{\subsecname}
	\begin{itemize}[<+->]
		\item Any Big Data solution working based distributed systems.
		\item What is distributed systems in brief?
	\end{itemize}
\end{frame}

%%%%%%%%%%%%%%%%%%%%%%%%%%%%%%%%%%%%%%%%%%%%%%%%%%%%%%%%%%%%%%%%%%%%%%%%%%%
\subsection{Go Production Part 2}

\begin{frame}
  \frametitle{\subsecname}
	\begin{itemize}[<+->]
		\item Any Big Data solution working based distributed systems.
		\item What is distributed systems in brief?
	\end{itemize}
\end{frame}


%%%%%%%%%%%%%%%%%%%%%%%%%%%%%%%%%%%%%%%%%%%%%%%%%%%%%%%%%%%%%%%%%%%%%%%%%%%
\section{Spark For Data Warehouse}
\subsection{ETL pipeline End-to-End into Spark}

\begin{frame}
  \frametitle{\subsecname}
	\begin{itemize}[<+->]
		\item Any Big Data solution working based distributed systems.
		\item What is distributed systems in brief?
	\end{itemize}
\end{frame}

%%%%%%%%%%%%%%%%%%%%%%%%%%%%%%%%%%%%%%%%%%%%%%%%%%%%%%%%%%%%%%%%%%%%%%%%%%%

%%%%%%%%%%%%%%%%%%%%%%%%%%%%%%%%%%%%%%%%%%%%%%%%%%%%%%%%%%%%%%%%%%%%%%%%%%%
\section{Spark Streaming}

\subsection{Introduction to Spark Streaming}

\begin{frame}
  \frametitle{\subsecname}
	\begin{itemize}[<+->]
		\item Any Big Data solution working based distributed systems.
		\item What is distributed systems in brief?
	\end{itemize}
\end{frame}

%%%%%%%%%%%%%%%%%%%%%%%%%%%%%%%%%%%%%%%%%%%%%%%%%%%%%%%%%%%%%%%%%%%%%%%%%%%

%%%%%%%%%%%%%%%%%%%%%%%%%%%%%%%%%%%%%%%%%%%%%%%%%%%%%%%%%%%%%%%%%%%%%%%%%%%
\subsection{Spark Streaming with Kafka}

\begin{frame}
  \frametitle{\subsecname}
	\begin{itemize}[<+->]
		\item Any Big Data solution working based distributed systems.
		\item What is distributed systems in brief?
	\end{itemize}
\end{frame}

%%%%%%%%%%%%%%%%%%%%%%%%%%%%%%%%%%%%%%%%%%%%%%%%%%%%%%%%%%%%%%%%%%%%%%%%%%%


%%%%%%%%%%%%%%%%%%%%%%%%%%%%%%%%%%%%%%%%%%%%%%%%%%%%%%%%%%%%%%%%%%%%%%%%%%%
\subsection{Spark Structure Streaming}

\begin{frame}
  \frametitle{\subsecname}
	\begin{itemize}[<+->]
		\item Any Big Data solution working based distributed systems.
		\item What is distributed systems in brief?
	\end{itemize}
\end{frame}

%%%%%%%%%%%%%%%%%%%%%%%%%%%%%%%%%%%%%%%%%%%%%%%%%%%%%%%%%%%%%%%%%%%%%%%%%%%


%%%%%%%%%%%%%%%%%%%%%%%%%%%%%%%%%%%%%%%%%%%%%%%%%%%%%%%%%%%%%%%%%%%%%%%%%%%
\subsection{Spark Streaming Applications}

\begin{frame}
  \frametitle{\subsecname}
	\begin{itemize}[<+->]
		\item Any Big Data solution working based distributed systems.
		\item What is distributed systems in brief?
	\end{itemize}
\end{frame}

%%%%%%%%%%%%%%%%%%%%%%%%%%%%%%%%%%%%%%%%%%%%%%%%%%%%%%%%%%%%%%%%%%%%%%%%%%%

\section{Spark using other Programming Languages}

%%%%%%%%%%%%%%%%%%%%%%%%%%%%%%%%%%%%%%%%%%%%%%%%%%%%%%%%%%%%%%%%%%%%%%%%%%%
\subsection{PySpsark for Python Geeks}

\begin{frame}
  \frametitle{\subsecname}
	\begin{itemize}[<+->]
		\item Any Big Data solution working based distributed systems.
		\item What is distributed systems in brief?
	\end{itemize}
\end{frame}

%%%%%%%%%%%%%%%%%%%%%%%%%%%%%%%%%%%%%%%%%%%%%%%%%%%%%%%%%%%%%%%%%%%%%%%%%%%
\subsection{PySpsark Utilizing Python Features}

\begin{frame}
  \frametitle{\subsecname}
	\begin{itemize}[<+->]
		\item Any Big Data solution working based distributed systems.
		\item What is distributed systems in brief?
	\end{itemize}
\end{frame}

%%%%%%%%%%%%%%%%%%%%%%%%%%%%%%%%%%%%%%%%%%%%%%%%%%%%%%%%%%%%%%%%%%%%%%%%%%%
\subsection{RSpark for R Geeks}

\begin{frame}
  \frametitle{\subsecname}
	\begin{itemize}[<+->]
		\item Any Big Data solution working based distributed systems.
		\item What is distributed systems in brief?
	\end{itemize}
\end{frame}

%%%%%%%%%%%%%%%%%%%%%%%%%%%%%%%%%%%%%%%%%%%%%%%%%%%%%%%%%%%%%%%%%%%%%%%%%%%

\section{Spark For Data Scientist}

%%%%%%%%%%%%%%%%%%%%%%%%%%%%%%%%%%%%%%%%%%%%%%%%%%%%%%%%%%%%%%%%%%%%%%%%%%%


\subsection{Data Analysis using Spark}

\begin{frame}
  \frametitle{\subsecname}
	\begin{itemize}[<+->]
		\item Any Big Data solution working based distributed systems.
		\item What is distributed systems in brief?
	\end{itemize}
\end{frame}

%%%%%%%%%%%%%%%%%%%%%%%%%%%%%%%%%%%%%%%%%%%%%%%%%%%%%%%%%%%%%%%%%%%%%%%%%%%
\subsection{Machine Leaning using Spark }

\begin{frame}
  \frametitle{\subsecname}
	\begin{itemize}[<+->]
		\item Any Big Data solution working based distributed systems.
		\item What is distributed systems in brief?
	\end{itemize}
\end{frame}

%%%%%%%%%%%%%%%%%%%%%%%%%%%%%%%%%%%%%%%%%%%%%%%%%%%%%%%%%%%%%%%%%%%%%%%%%%%

%%%%%%%%%%%%%%%%%%%%%%%%%%%%%%%%%%%%%%%%%%%%%%%%%%%%%%%%%%%%%%%%%%%%%%%%%%%
\subsection{Deep Learning using Spark}

\begin{frame}
  \frametitle{\subsecname}
	\begin{itemize}[<+->]
		\item Any Big Data solution working based distributed systems.
		\item What is distributed systems in brief?
	\end{itemize}
\end{frame}

%%%%%%%%%%%%%%%%%%%%%%%%%%%%%%%%%%%%%%%%%%%%%%%%%%%%%%%%%%%%%%%%%%%%%%%%%%%

\section{Spark Graph Dataframe/Graphx}
%%%%%%%%%%%%%%%%%%%%%%%%%%%%%%%%%%%%%%%%%%%%%%%%%%%%%%%%%%%%%%%%%%%%%%%%%%%
\subsection{Introduction to Spark Graph Dataframe/GraphX}

\begin{frame}
  \frametitle{\subsecname}
	\begin{itemize}[<+->]
		\item Any Big Data solution working based distributed systems.
		\item What is distributed systems in brief?
	\end{itemize}
\end{frame}

%%%%%%%%%%%%%%%%%%%%%%%%%%%%%%%%%%%%%%%%%%%%%%%%%%%%%%%%%%%%%%%%%%%%%%%%%%%
\subsection{Graph Applications using Spark}

\begin{frame}
  \frametitle{\subsecname}
	\begin{itemize}[<+->]
		\item Any Big Data solution working based distributed systems.
		\item What is distributed systems in brief?
	\end{itemize}
\end{frame}


%%%%%%%%%%%%%%%%%%%%%%%%%%%%%%%%%%%%%%%%%%%%%%%%%%%%%%%%%%%%%%%%%%%%%%%%%%%

\section{Tuning your Spark Jobs}

%%%%%%%%%%%%%%%%%%%%%%%%%%%%%%%%%%%%%%%%%%%%%%%%%%%%%%%%%%%%%%%%%%%%%%%%%%%
\subsection{Tuning Spark Jobs Part 1}

\begin{frame}
  \frametitle{\subsecname}
	\begin{itemize}[<+->]
		\item Any Big Data solution working based distributed systems.
		\item What is distributed systems in brief?
	\end{itemize}
\end{frame}

%%%%%%%%%%%%%%%%%%%%%%%%%%%%%%%%%%%%%%%%%%%%%%%%%%%%%%%%%%%%%%%%%%%%%%%%%%%
\subsection{Tuning Spark Jobs Part 2}

\begin{frame}
  \frametitle{\subsecname}
	\begin{itemize}[<+->]
		\item Any Big Data solution working based distributed systems.
		\item What is distributed systems in brief?
	\end{itemize}
\end{frame}

%%%%%%%%%%%%%%%%%%%%%%%%%%%%%%%%%%%%%%%%%%%%%%%%%%%%%%%%%%%%%%%%%%%%%%%%%%%
%\begin{appendices}
\section{Appendix}


\begin{frame}
  \frametitle{Appendix A: Hadoop Map-Reduce}
	\begin{itemize}[<+->]
		\item Any Big Data solution working based distributed systems.
		\item What is distributed systems in brief?
	\end{itemize}
\end{frame}

%%%%%%%%%%%%%%%%%%%%%%%%%%%%%%%%%%%%%%%%%%%%%%%%%%%%%%%%%%%%%%%%%%%%%%%%%%%

%%%%%%%%%%%%%%%%%%%%%%%%%%%%%%%%%%%%%%%%%%%%%%%%%%%%%%%%%%%%%%%%%%%%%%%%%%%
\begin{frame}
  \frametitle{Appendix B: Introduction to Scala}
	\begin{itemize}[<+->]
		\item Any Big Data solution working based distributed systems.
		\item What is distributed systems in brief?
	\end{itemize}
\end{frame}

%%%%%%%%%%%%%%%%%%%%%%%%%%%%%%%%%%%%%%%%%%%%%%%%%%%%%%%%%%%%%%%%%%%%%%%%%%%

%%%%%%%%%%%%%%%%%%%%%%%%%%%%%%%%%%%%%%%%%%%%%%%%%%%%%%%%%%%%%%%%%%%%%%%%%%%
\begin{frame}
  \frametitle{Appendix C: SQL Syntax and Tips for Data Science}
	\begin{itemize}[<+->]
		\item Any Big Data solution working based distributed systems.
		\item What is distributed systems in brief?
	\end{itemize}
\end{frame}

%%%%%%%%%%%%%%%%%%%%%%%%%%%%%%%%%%%%%%%%%%%%%%%%%%%%%%%%%%%%%%%%%%%%%%%%%%%

%%%%%%%%%%%%%%%%%%%%%%%%%%%%%%%%%%%%%%%%%%%%%%%%%%%%%%%%%%%%%%%%%%%%%%%%%%%
\begin{frame}
  \frametitle{Appendix D: DWH Concepts and Modeling}
	\begin{itemize}[<+->]
		\item Any Big Data solution working based distributed systems.
		\item What is distributed systems in brief?
	\end{itemize}
\end{frame}

%%%%%%%%%%%%%%%%%%%%%%%%%%%%%%%%%%%%%%%%%%%%%%%%%%%%%%%%%%%%%%%%%%%%%%%%%%%

%%%%%%%%%%%%%%%%%%%%%%%%%%%%%%%%%%%%%%%%%%%%%%%%%%%%%%%%%%%%%%%%%%%%%%%%%%%
\begin{frame}
  \frametitle{Appendix E: Maven for Automation Cycle}
	\begin{itemize}[<+->]
		\item Any Big Data solution working based distributed systems.
		\item What is distributed systems in brief?
	\end{itemize}
\end{frame}

%%%%%%%%%%%%%%%%%%%%%%%%%%%%%%%%%%%%%%%%%%%%%%%%%%%%%%%%%%%%%%%%%%%%%%%%%%%

%%%%%%%%%%%%%%%%%%%%%%%%%%%%%%%%%%%%%%%%%%%%%%%%%%%%%%%%%%%%%%%%%%%%%%%%%%%
\begin{frame}
  \frametitle{Appendix F: How to use Shell Scripts to Automate your tasks}
	\begin{itemize}[<+->]
		\item Any Big Data solution working based distributed systems.
		\item What is distributed systems in brief?
	\end{itemize}
\end{frame}

%%%%%%%%%%%%%%%%%%%%%%%%%%%%%%%%%%%%%%%%%%%%%%%%%%%%%%%%%%%%%%%%%%%%%%%%%%%


%%%%%%%%%%%%%%%%%%%%%%%%%%%%%%%%%%%%%%%%%%%%%%%%%%%%%%%%%%%%%%%%%%%%%%%%%%%
\begin{frame}
  \frametitle{Appendix G: Introduction to Oozie for Big Data Development}
	\begin{itemize}[<+->]
		\item Any Big Data solution working based distributed systems.
		\item What is distributed systems in brief?
	\end{itemize}
\end{frame}

%%%%%%%%%%%%%%%%%%%%%%%%%%%%%%%%%%%%%%%%%%%%%%%%%%%%%%%%%%%%%%%%%%%%%%%%%%%

%%%%%%%%%%%%%%%%%%%%%%%%%%%%%%%%%%%%%%%%%%%%%%%%%%%%%%%%%%%%%%%%%%%%%%%%%%%
\begin{frame}
  \frametitle{Appendix H: Introduction to Deep Learning}
	\begin{itemize}[<+->]
		\item Any Big Data solution working based distributed systems.
		\item What is distributed systems in brief?
	\end{itemize}
\end{frame}
%\end{appendices}
%%%%%%%%%%%%%%%%%%%%%%%%%%%%%%%%%%%%%%%%%%%%%%%%%%%%%%%%%%%%%%%%%%%%%%%%%%%
%%% Local Variables:
%%% mode: latex
%%% TeX-master: "../main"
%%% TeX-engine: xetex
%%% End:

%
%%%%%%%%%%%%%%%%%%%%%%%%%%%%%%%%%%%%%%%%%%%%%%%%%%%%%%%%%%%%%%%%%%%%%%%%%%%%%%%%%%%%%%%%%%%%%%%%%%


%\section{First section}
%
%%---------------------------------------------------------
%%Changing visivility of the text
%\begin{frame}
%\frametitle{Sample frame title}
%This is a text in second frame. For the sake of showing an example.
%
%\begin{itemize}
%    \item<1-> Text visible on slide 1
%    \item<2-> Text visible on slide 2
%    \item<3> Text visible on slides 3
%    \item<4-> Text visible on slide 4
%\end{itemize}
%\end{frame}
%
%%---------------------------------------------------------
%
%
%%---------------------------------------------------------
%%Example of the \pause command
%\begin{frame}
%In this slide \pause
%
%the text will be partially visible \pause
%
%And finally everything will be there
%\end{frame}
%%---------------------------------------------------------
%
%\section{Second section}
%
%%---------------------------------------------------------
%%Highlighting text
%\begin{frame}
%\frametitle{Sample frame title}
%
%In this slide, some important text will be
%\alert{highlighted} beause it's important.
%Please, don't abuse it.
%
%\begin{block}{Remark}
%Sample text
%\end{block}
%
%\begin{alertblock}{Important theorem}
%Sample text in red box
%\end{alertblock}
%
%\begin{examples}
%Sample text in green box. "Examples" is fixed as block title.
%\end{examples}
%\end{frame}
%%---------------------------------------------------------
%
%
%%---------------------------------------------------------
%%Two columns
%\begin{frame}
%\frametitle{Two-column slide}
%
%\begin{columns}
%
%\column{0.5\textwidth}
%This is a text in first column.
%$$E=mc^2$$
%\begin{itemize}
%\item First item
%\item Second item
%\end{itemize}
%
%\column{0.5\textwidth}
%This text will be in the second column
%and on a second tought this is a nice looking
%layout in some cases.
%\end{columns}
%\end{frame}
%%---------------------------------------------------------


\end{document}

%%% Local Variables:
%%% mode: latex
%%% TeX-master: t
%%% TeX-engine: xetex
%%% End: